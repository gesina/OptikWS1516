%%----------------------------------------------
%%
%%            Mitschrift der Vorlesung
%%   Experimentalphysik III: Wellen und Quanten 
%%       im WS2015/16 an der Uni Regensburg
%%
%% Anhang 2: Fourier-Transformation
%%
%%
%% gesetzt von Gesina und Hedwig
%%
%%----------------------------------------------

\chapter{Fouriertransformation}\label{fouriertrafo}

% --------------
% VL 10.12.2015
% --------------
\VL{10.12.2015}

Mit der Fouriertransformation ist die Fourierreihe für periodische
Funktionen verwandt.

\section{Fourierreihe}
Sei $f(t)$ eine komplexwertige periodische Funktion mit Periode
$T=\frac{2\pi}{\omegao}$, dann lässt sie sich als Summe wie folgt darstellen
\begin{align*}
  f(t) &= \sum_{k=-\infty}^{\infty} F_k\cdot e^{ik\omegao t}\\
  \Rightarrow\quad 
  f(t) &= \sum_0^{\infty} A_k\cos(\omega_k t)+B_k\sin(\omega_k t)
\end{align*}
wobei $B_0=0$ und $\omega_k=k\omegao=\frac{2\pi k}{T}$ die
Kreisfrequenz der periodischen Funktion ist. Die Koeffizienten lassen
sich berechnen, indem man beide Seiten mit $\exp(-ikt\omegao)$
multipliziert und über eine Periode integriert
\begin{align*}
  A_k &= \frac{2}{T} \int_{-\frac{T}{2}}^{\frac{T}{2}}
        f(t)\cos(\omega_k t)\dd t
  & k&\neq 0\\
  A_0 &= \fracone{T}\int_{-\frac{T}{2}}^{\frac{T}{2}} f(t)\dd t\\
  B_k &= \frac{2}{T}\int_{-\frac{T}{2}}^{\frac{T}{2}}
        f(t)\sin(\omega_k t)\dd t\\
\end{align*}
Das ergibt für die Summen-Koeffizienten $F_k$ der
Fouriertransformierten
\begin{align*}
  F_k &= \fracone{T}\int_{-\frac{T}{2}}^{\frac{T}{2}}
        f(t) e^{-k\omegao t}\dd t
  &-\infty &\leq k\leq\infty\\
  \text{also}\quad 
  f(t) &= A_0 + \sum_{1}^{\infty}(A_k\cos(\omega_k t) + B_k\sin(\omega_k t)\\
      &= F_0 + \sum_{1}^{\infty}e^{i\omega_kt} 
        \left( \frac{A_k-iB_k}{2} \right)
        + e^{-i\omega_kt}\left( \frac{A_k+iB_k}{2} \right)\\
      &= \sum_{-\infty}^{\infty} F_k e^{i\omega_k t}
\end{align*}

\Beispiel{periodische Rechtecksfunktion}
\begin{align}\label{rechtecksfunktion}
  f(t)&=\left\{
        \begin{array}{ll}
          0&\text{für $-\frac{T}{2}\leq t\leq -\frac{b}{2}$}\\
          a&\text{für $-\frac{b}{2}\leq t\leq \frac{b}{2}$}\\
          0&\text{für $\phantom{-}\frac{b}{2}\leq t\leq \frac{T}{2}$}
        \end{array}
             \right.
\end{align}
Der nullte Koeffizient lautet dann
\begin{align*}
  F_0 &= \frac{1}{T} \int_{-\frac{T}{2}}^{\frac{T}{2}}
        f(t)e^{-i\omegao t}\dd t
        = \frac{ab}{T}
\end{align*}
Die weiteren Koeffizienten lauten:
\begin{align*}
  F_k &= \fracone{T} \int_{-\frac{b}{2}}^{\frac{b}{2}} 
        ae^{i\omegao t}\dd t
        = \frac{a}{k\pi}\sin(k\omegao\frac{b}{2})
\end{align*}
Im Spezialfall $b=\frac{b}{2}$ (\SI{50}{\percent} Duty-Cycle) gilt:
\begin{align*}
  F_0 &= \frac{a}{2}
  &F_{\pm 2n} &= 0
  &F_{\pm (2n+1)} &= \frac{a}{(2n+1)\pi}
\end{align*}

%------

\section[Übergang zu nicht-periodischen Funktionen]%
{Übergang zu nicht-periodischen Funktionen}
Betrachte nun wieder Rechtecksfunktionen (Gleichung
\eqref{rechtecksfunktion}) mit der Breite $b$, 
verändere nun aber die Periode bzw. den Abstand zwischen
Rechtecksimpulsen.
Wenn die Periodendauer gegen unendlich geht
($T\longrightarrow\infty$), geht der Abstand der Fourierkoeffizienten
gegen null. Dabei beschreiben die Fourierkoeffizienten immer noch die
gleiche Funktion $f(\omega)$. Führt man nun eine Grenze $T_0$ ein kann
man oberhalb davon, also bei $\vert t\vert >T_0$, $f(t)$
vernachlässigen. Die Koeffizienten werden also durch $\omega=k\omegao$
und $\fracone{T}$ bestimmt. 
\begin{align*}
  F_k &= \fracone{T} \int_{-\frac{T}{2}}^{\frac{T}{2}} 
        f(t) e^{-ik\omegao t}\dd t
\end{align*}
Führe nun eine Funktion $F(\omega)$ ein, deren Funktionswerte bei
$\omega=\omegao k$ die Fourierkoeffizienten $F_k$ bestimmen.
\begin{align*}
  F(\omega) &= \int_{-\frac{T}{2}}^{\frac{T}{2}}
              f(t)e^{-i\omega t}\dd t\\
  F_k &= \fracone{T} F(\omega_k)        
        =\fracone{T}F(k\omega)
\end{align*}
Wenn man dies nun in $f(t)$ einsetzt erhält man
\begin{align*}
  f(t) &= \sum_{k=-\infty}^{\infty}
         \fracone{T} F(k\omegao)e^{ik\omegao t}
\end{align*}
Da $\fracone{T}$ direkt proportional zum Frequenzabstand 
$\dd\omega=\omegao$ zwischen zwei Fourierkomponenten ist, folgt daraus
\begin{align*}
  f(t) &= \sum_{k=-\infty}^{\infty}
         \frac{\dd \omega}{2\pi} F(k\omegao) e^{ik\omegao t}
\end{align*}
Wenn nun die Periodendauer gegen unendlich geht, 
wird $\dd \omega$ infinitesimal klein und man erhält die
Fouriertransformation $f(t)$ mit ihrer Fouriertransformierten
$F(\omega)$.
\begin{align*}
  f(t) &= \int_{-\infty}^{\infty} 
         F(\omega)e^{i\omega t}\frac{\dd\omega}{2\pi}\\
  F(\omega) &= \int_{-\infty}^{\infty} 
              f(t)e^{-i\omega t}\dd t
\end{align*}
Die Abbildung $f(t)\longrightarrow F(\omega)$ ergibt das
Frequenzspektrum der Funktion $f(t)$.

\minisec{Anwendung bei Messungen} 
Bei Messungen wird meist nur die Intensität in
Abhängigkeit von Zeit und Frequenz gemessen. Die Intensitäten sind
aber nicht über eine Fouriertransformation miteinander verknüpft,
deshalb muss man $I\propto E^2$ ausnutzen, denn die $E$-Felder in
Abhängigkeit zur Zeit und Frequenz sind wieder über eine
Fouriertransformation miteinander verknüpft.

%------

\section{Rechenregeln und Spezialfälle}
\subsection{Der Faltungssatz}
Eine Faltung ist mathematisch definiert als 
\begin{align*}
  f(t)\otimes g(t)
  &\coloneqq \int_{-\infty}^{\infty} f(\zeta)g(t-\zeta)\dd\zeta
\end{align*}
Die Fouriertransformation einer Faltung sieht wie folgt aus ($F$ FT
von $f$, $G$ FT von $g$):
\begin{align*}
  x(t) &\coloneqq g(t)\otimes f(t)
  &\text{FT: } X(\omega) &= G(\omega)F(\omega)\\
  \Rightarrow\quad 
  x(t) &= FT^{-1}(FT(g(t))\cdot FT(f(t))
\end{align*}
Eine Faltung lässt sich berechnen, indem die zu faltenden Funktionen
$g(t)$ und $f(t)$ fouriertransformiert, dann miteinander
multipliziert werden und schließlich das Produkt wieder in den
ursprünglichen Raum rücktransformiert wird.
Hierbei ist die Fouriertransformation (FT) als Operation zwischen
Funktionsräumen aufgefasst (Funktionen $\to$ Funktionen).

%---

\subsection{Spezielle Funktionen}
\subsubsection{Konstante Funktionen und Delta-Funktion}
Betrachte die Delta-Funktion $\delta(x)$ bzw. $\delta(\omega)$.
$\delta(t)$. $\delta(t)$ ist überall identisch Null außer bei $t=0$. 
\begin{align*}
  \delta(t) &= \begin{Bmatrix}
    0 &\text{für} &t\neq 0\\
    \infty &\text{für} &t=0
  \end{Bmatrix}\\
  \delta(\omega) &= \fracone{2\pi}\int_{-\infty}^{\infty}
                   e^{-i\omega t}\dd t
\end{align*}
Eine konstante Funktion besitzt also als Fouriertransformierte die
$\delta(\omega)$-Funktion.
\begin{align*}
  \int_{x-a}^{x+a} f(t)\delta(t-\omega)\dd t
  &= f(\omega)\\
  \delta(ax) &= \fracone{a}\delta(x)\\
  \delta(g(x)) &= \sum_{i=1}^{n}\fracone{g\prime (x_i)}\delta(x-x_i)
             &\text{mit } g(x_i)&0&g\prime(x_i)&\neq 0
\end{align*}

%---

\subsubsection{Gauß-Funktion} 
Eine (breite) Gauß-Funktion transformiert
immer in eine (schmale) Gauß-Funktion.
\begin{align*}
  F(\omega) &= \fracone{\sigma\sqrt{2\pi}}\int_{-\infty}^{\infty}
              e^{-\frac{t^2}{2\sigma^2}} e^{-i\omega t} \dd t
              = e^{-\frac{\sigma^2\omega^2}{2}}
\end{align*}
Die Gauß-Funktion ist eine Darstellung der $\delta$-Funktion für
$\sigma\longrightarrow 0$.

%---

\subsubsection{Exponentialfunktion} 
Eine (breite) Exponentialfunktion
transformiert in eine (schmale) Lorentzfunktion.
\begin{align*}
  f(t) &= e^{-\frac{\vert t\vert}{\tau}}\\
  F(\omega) &= \frac{2\tau}{1+\omega^2\tau}
\end{align*}

%---

\subsubsection{Rechteckfunktion} Eine Rechteckfunktion transformiert in
eine sinc-Funktion.
\begin{gather*}
  F(\omega) 
  = 2\int_{0}^{\frac{T}{2}}\cos(\omega t)\dd t
  \propto -\frac{\sin(x)}{x} x
\end{gather*}

% Bei der Funktion stimmt iwas noch nicht und sie kommt vollständig noch in der Übung dran

% --------------
% VL 14.12.2015
% --------------
\VL{14.12.2015}


\subsection{Parseval'sche Formel und Verschiebungssatz}
Es gilt die \emph{Parseval'sche Formel}\index{Parseval'sche Formel}
\begin{gather*}
  \int_{-\infty}^{\infty} |f(t)|^2 \dd t 
  = \int_{-\infty}^{\infty} |F(\omega)|^2 \dd\omega
\end{gather*}
und der \emph{Verschiebungssatz}\index{Verschiebungssatz}
\begin{align*}
  FT(f(t-t_0)) = FT(f(x)\otimes\delta(t-t_0))
  = \underbrace{FT(f(t))}_{F(\omega)}e^{-i\omegao t_0}
\end{align*}


%%% Local Variables:
%%% mode: latex
%%% TeX-master: "../OptikSkriptWS1516"
%%% End:
