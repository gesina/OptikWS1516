\chapter{Wellenoptik}
In der geometrischen Optik wird angenommen, dass wir immer einen
Fokus im Punkt erreichen können, d.\,h. unendlich gute Ortsauflösung.
In der Realität liegt der mögliche Fokus im Bereich der Wellenlänge.

Weiterhin betrachte das Gedankenexperiment, bei dem ein durch eine Blende
erzeugter, rechteckiger Lichtstrahl auf einem Schirm aufgefangen wird.
%% IMAGE MISSING
Das Bild ist kein exaktes Rechteck, denn durch Interferenz zeigt sich
eine Modulation der Intensitätsverteilung. Im Randbereich sorgt die
abnehmende Intensität dafür, dass es keinen scharfen Rand gibt, das
Rechteck wird zu einem unscharfen Fleck.
Allgemein beobachtet man bei Schattenwurf mit monochromatischem Licht
an einer scharfen Kante Licht an Orten, wo nach klassischer
geometrischer Optik keines sein sollte.
Daraus sieht man, dass die Wellennatur des Lichts zur
Abweichung von der geradlinigen Ausbreitung (\emph{Beugung} an
der Blende) und zu Intensitätsmodulationen am Schirm
(\emph{Interferenz}) führt.

%%
%Vorlesung vom 10.12
%%
\section{Fourier-Transformation}
Mit der Fouriertransformation ist die Fourierreihe für periodisce Funktionen verwandt.
\minisec{Fourierreihe}
$f(t)$ ist eine komplexwertige periodische Funktion mit Periode $T=\frac{2\pi}{\omegao}$ und lässt sich als Summe darstellen.
\begin{align*}
	f(t)&=\sum_{k=-\infty}^{\infty} F_k\cdot e^{ik\omegao t}\\
	=>\quad f(t)&=\sum_0^{\infty} A_k\cos(\omega_k t)+B_k\sin(\omega_k t)
	\text{wobei: }\omega_k&=k\omegao=\frac{2\pi k}{T}&B_0&=0
\end{align*}
Wobei $\omegao$ die Kreisfrequenz der periodischen Funktion ist.Die Koeffizienten lassen sich berechnen, indem man beide Seiten mit $\exp(-ikt\omegao)$ multipliziert und über eine Periode integriert.
\begin{align*}
	A_k&=\frac{2}{T}\int_{-\frac{T}{2}}^{\frac{T}{2}}f(t)\cos(\omega_k t)\dd t&k&\neq 0\\
	A_0&=\fracone{T}\int_{-\frac{T}{2}}^{\frac{T}{2}}f(t)\dd t\\
	B_k&=\frac{2}{T}\int_{-\frac{T}{2}}^{\frac{T}{2}}f(t)\sin(\omega_k t)\dd t\\
	\text{bzw.}\quad F_k&=\fracone{T}\int_{-\frac{T}{2}}^{\frac{T}{2}}f(t) e^{-k\omegao t}\dd t&-\infty&\leq k\leq\infty\\
	\text{also}\quad f(t)&=A_0+\sum_{1}^{\infty}(A_k\cos(\omega_k t)+B_k\sin(\omega_k t)\\
	&=F_0+\sum_{1}^{\infty}e^{i\omega_kt}\left(\frac{A_k-iB_k}{2}\right)+e^{-i\omega_kt}\left(\frac{A_k+iB_k}{2}\right)\\
	&=\sum_{-\infty}^{\infty}F_ke^{i\omega_k t}
\end{align*}
\paragraph{Beispiel 1:} periodische Rechtecksfunktion
\begin{align*}
	f(t)&=\begin{Bmatrix}
	0&\text{für}&-\frac{T}{2}\leq t\leq -\frac{b}{2}\\
	a&\text{für}&-\frac{b}{2}\leq t\leq \phantom{-}\frac{b}{2}\\
	0&\text{für}&\phantom{-} \frac{b}{2}\leq t\leq \phantom{-}\frac{T}{2}\\
	\end{Bmatrix}
\end{align*}
Der nullte Koeffizient lautet dann:
\begin{align*}
	F_0&=\frac{1}{T}\int_{-\frac{T}{2}}^{\frac{T}{2}}f(t)e^{-i\omegao 0 t}\dd t= \frac{ab}{T}
\end{align*}
Die weiteren Koeffizienten lauten:
\begin{align*}
	F_k&=\fracone{T}\int_{-\frac{b}{2}}^{\frac{b}{2}} ae^{i\omegao t}\dd t=\frac{a}{k\pi}\sin(k\omegao \frac{b}{2})
\end{align*}
Im Spezialfall $b=\frac{b}{2}$ (50\% Duty-Cycle) gilt:
\begin{align*}
	F_0&=\frac{a}{2}& F_{\pm 2n}&=0&F_{\pm (2n+1)}&=\frac{a}{(2n+1)\pi}
\end{align*}
\paragraph{Beispiel 2:}Übergang zu nicht- periodischen Funktionen. Betrrachte nun wieder Rechteckfunktionen mit der Breite b, verändere nun aber die Periode bzw. den Abstand zwischen Rechteckimpulsen.
Wenn die Periodendauer gegen unendlich geht ($T\longrightarrow\infty$), geht der Abstand der Fourierkoeffizienten gegen null. Dabei beschreiben die Fourierkoeffizienten immer noch die gleiche Funktion $f(\omega)$. Führt man nun eine Grenze $T_0$ ein kann man oberhalb davon, also bei $\vert t\vert >T_0$, $f(t)$ vernachlässigen. Die Koeffizienten werden also durch $\omega=k\omegao$ und $\fracone{T}$ bestimmt.
\begin{align*}
	F_k&=\fracone{T} \int_{-\frac{T}{2}}^{\frac{T}{2}} f(t) e^{-ik\omegao t}\dd t
\end{align*}
Führe nun eine Funktion $F(\omega)$ ein, deren Funktionswerte bei $\omega=\omegao k$ die Fourierkoeffizienten $F_k$ bestimmen.
\begin{align*}
	F(\omega)&=\int_{-\frac{T}{2}}^{\frac{T}{2}}f(t)e^{-i\omega t}\dd t\\
	F_k&=\fracone{T}F(\omega_k)=\fracone{T}F(k\omega)
\end{align*}
Wenn man dies nun in $f(t)$ einsetzt erhält man:
\begin{align*}
	f(t)&=\sum_{k=-\infty}^{\infty}\fracone{T} F(k\omegao)e^{ik\omegao t}
\end{align*}
Da $\fracone{T}$ direkt proportional zum Frequenzabstand $\dd\omega =\omegao$ zwischen zwei Fourierkomponenten ist, folgt daraus:
\begin{align*}
	f(t)&=\sum_{k=-\infty}^{\infty}\frac{\dd \omega}{2\pi} F(k\omegao)e^{ik\omegao t}
\end{align*}
Wenn nun die Periodendauer gegen unendlich geht, wird $\dd \omega$ infitissimal klein und man erhält die Fouriertransformation $f(t)$ mit ihrer Fouriertransformierten $F(\omega)$.
\begin{align*}
	f(t)&=\int_{-\infty}^{\infty} F(\omega)e^{i\omega t}\frac{\dd\omega}{2\pi}\\
	F(\omega)&=\int_{-\infty}^{\infty} f(t)e^{-i\omega t}\dd t
\end{align*}
Die Abbildung $f(t)\longrightarrow F(\omega)$ ergibt das Frequenzspektrum der Funktion $f(t)$.
\paragraph{Messungen:} Bei Messungen wird meist nur die Intensität in Abhängigkeit von Zeit und Frequenz gemessen. Die Intensitäten sind aber nicht über eine Fouriertransformation miteinander verknüpft, deshalb muss man $I\propto E^2$ ausnutzen, denn die E-Felder in Abhängigkeit zur Zeit und Frequenz sind wieder über eine Fouriertransformation miteinander verknüpft.

\minisec{Rechenregeln und Spezialfälle}
Betrachte nun die Delta-Funktion $\delta(x)$ bzw. $\delta(\omega), \delta(t)$.$\delta(t)$ ist überall identisch Null außer bei $t=0$.
\begin{align*}
	\delta(t)&=\begin{Bmatrix}
	 0&\text{für}&t\neq 0\\
	 \infty&\text{für}&t=0
	\end{Bmatrix}\\
	\delta(\omega)&=\fracone{2\pi}\int_{-\infty}^{\infty}e^{-i\omega t}\dd t
\end{align*}
Eine konstante Funktion besitzt also als FT die $\delta(\omega)$-Funktion.
\begin{align*}
	\int_{x-a}^{x+a}f(t)\delta(t-\omega)\dd t&=f(\omega)\\
	\delta(ax)&=\fracone{a}\delta(x)\\
	\delta(g(x))&=\sum_{i=1}^{n}\fracone{g\prime (x_i)}\delta(x-x_i)&\text{mit:}\quad g(x_i)&0&g\prime(x_i)&\neq 0
\end{align*}
\minisec{Der Faltungssatz}
Eine Faltung ist mathematisch definiert als 
\begin{align*}
	f(t)\otimes g(t)&=\int_{-\infty}^{\infty}f(\zeta)g(t-\zeta)\dd\zeta
	x(t)&=g(t)\otimes f(t)& \text{FT:}\quad X(\omega)&=G(\omega)F(\omega)\\
	=>\quad x(t)&=FT^{-1}(FT(g(t))\cdot FT(f(t))
\end{align*}
Eine Faltung lässt sich berechnen, indem die zu faltenden Funktionen $g(t)$ und $f(t)$ fouriertransformiert werde, dann miteinander multipliziert werden und schließlich das Produkt wieder in den ursprünglichen Raum rücktransformiert wird.
\minisec{Spezielle Funktionen}
\paragraph{Gauß-Funktion} Eine (breite) Gauß-Funktion transformiert immer in eine (schmale) Gauß-Funktion.
\begin{align*}
	F(\omega)&=\fracone{\sigma\sqrt{2\pi}}\int_{-\infty}^{\infty}e^{-\frac{t^2}{2\sigma^2}}e^{-i\omega t}\dd t
	=e^{-\frac{\sigma^2\omega^2}{2}}
\end{align*}
Die Gauß-Funktion ist eine Darstellung der $\delta$-Funktion für $\sigma\longrightarrow 0$.
\paragraph{Exponentialfunktion} Eine (breite) Exponentialfunktion transformiert in eine (schmale) Lorentzfunktion.
\begin{align*}
	f(t)&=e^{-\frac{\vert t\vert}{\tau}}\\
	F(\omega)&=\frac{2\tau}{1+\omega^2\tau}
\end{align*}
\paragraph{Rechteckfunktion} Eine Rechteckfunktion transformiert in eine sinc- Funktion.
\begin{align*}
	F(\omega)&=2\int_{0}^{\frac{T}{2}}\cos(\omega t)\dd t&\propto-\frac{\sin(x)}{x}x
\end{align*}
%Bei der Funktion stimmt iwas noch nicht und sie kommt vollständig noch in der Übung dran

%%% mode: latex
%%% TeX-master: "../OptikSkriptWS1516"
%%% End:
