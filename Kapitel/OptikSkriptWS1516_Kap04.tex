%% -----------------------------------------

%% Vorlesungsmitschrift (Kapitel 4)

%% an der Uni Regensburg, gelesen von Christian Back

%% -----------------------------------------


\chapter{Elektromagnetische Wellen an Grenzflächen}
\section{Randbedingungen der elektromagnetischen Welle}
Wir wollen jetzt Wellenausbreitung in inhomogenen Medien beschreiben,
z.\,B. den Übergang von Medium~1 nach Medium~2, also einer
Grenzfläche.

\paragraph{Randbedingungen der MWGl.:} Die Tangentialkomponenten von
$\vecE$ und $\vecH=\fracone{\muo\mu_r}\vecB$ sind stetig. Die
Normalkomponenten von $\vecD=\epso\eps_r\vecE$ und $\vecB$ sind
ebenfalls stetig (Erinnerung: Für isotrope, isolierende, nicht
magnetische Medien gilt $\mu_r\eqqcolon\mu=1$).
\paragraph{einfachster Fall:} Wir betrachten zwei homogene Medien mit
Brechungsindex $n_e$ (einfallender Strahl) und $n_t$ (transmittierte
Welle).
% 
%% Skizze zu Reflexion
% 
Der Winkel $\alpha$ liegt zwischen $\veck_e$ und $\vece_y$
bzw. zwischen $\veck_t$ und $\vece_y$. Wir nehmen an, dass es eine
fest vorgegebene einlaufende Welle ist.
\begin{align*}
  \vecE_e &= \vecE_{0,e}\cos(\omega_et-\veck_e\vecr)
            =\vecE_{0,e}(\phi_e(\vecr,t))\\
  \vecE_r &= \vecE_{0,r}\cos(\omega_rt - \veck_r\vecr+\varphi_r)
            =\vecE_{0,r}(\phi_r(\vecr,t))\\
  \vecE_t &= \vecE_{0,t}\cos(\omega_tt - \veck_t\vecr + \varphi_t)
            =\vecE_{0,t}(\phi_t(\vecr,t))
\end{align*}
Die Wellenvektoren $\veck_e$, $\veck_r$ und $\veck_t$ müssen die%
\nomenclature{$\veck_e$}{Wellenvektor der einlaufenden Welle}%
\nomenclature{$\veck_t$}{Wellenvektor der transmittierten Welle}%
\nomenclature{$\veck_r$}{Wellenvektor der reflektierten Welle}
Dispersionsrelationen im jeweiligen Medium erfüllen. Die
\emph{Phasenfaktoren}\index{Phasenfaktoren}
$\varphi_r$, $\varphi_t$%
\nomenclature{$\varphi_r$}{Phasenfaktor der reflektierten Welle;
  bestimmt Phasenverschiebung zur einlaufenden Welle}%
\nomenclature{$\varphi_t$}{Phasenfaktor der transmittierten Welle;
  bestimmt Phasenverschiebung zur einlaufenden Welle} 
bestimmen die Phasenlage relativ zur einlaufenden Welle.

% -------------------

\section{Reflexions- und Brechungsgesetz}
Die Stetigkeit der Tangentialkomponente des elektrischen Feldes an der
Grenzfläche ist wichtig für den Wellenverlauf.
Daher muss man die Randbedingung annehmen, dass die $x$- und die
$z$-Komponente von $\vecE$ und $\vecH$ stetig sein müssen.
Z.\,B. ist für
\begin{align*}
  E_{0_{e_x}} \cos(\phi_e(\vecr,t)) 
  +  E_{0_{r_x}} \cos(\phi_r(\vecr,t)) 
  &=   E_{0_{t_x}} \cos(\phi_t(\vecr,t)) \\
  E_{0_{e_z}} \cos(\phi_e(\vecr,t)) 
  +  E_{0_{r_z}} \cos(\phi_r(\vecr,t)) 
  &=   E_{0_{t_z}} \cos(\phi_t(\vecr,t)) \\
\end{align*}
die notwendige Bedingung, dass für alle $t$ und alle $\vecr$
mit $y=0$ gilt
\begin{gather*}
  \phi_e(\vecr, t) = \phi_r(\vecr,t) = \phi_t(\vecr, t) 
  \qquad\text{bzw.}\\
  \omega_et- \veck_e\vecr = \omega_rt- \veck_r\vecr = \omega_tt- \veck_t\vecr
\end{gather*}
Diese ist nur erfüllbar, wenn folgendes gilt
\begin{itemize}
\item $\omega_e=\omega_r=\omega_t$.\\
  \red{Achtung:} Es ist eine Wellenlängenänderung möglich.
\item Die Ebenengleichungen
  \begin{align*}
    \veck_e\vecr &= \veck_r\vecr \quad\text{bzw.} \quad(\veck_e-\veck_r)\vecr=0\\
    \veck_e\vecr &= \veck_t\vecr \quad\text{bzw.} \quad(\veck_e-\veck_t)\vecr=0
  \end{align*}
  D.\,h. die Vektoren $(\veck_e-\veck_r)$
  und $(\veck_e-\veck_t)$ müssen senkrecht auf der Ebene $y=0$ stehen.
\end{itemize}
Daraus folgt, dass die Komponenten $\veck_e$ und $\veck_r$ parallel
zur Grenzfläche gleich sein müssen:
\begin{gather*}
  \veck_e = \veck_r
\end{gather*}
Setzt man hier nun die Dispersionsrelation 
$\veck^2 = \frac{n^2\omega^2}{c^2}$ ein, erhält man
$k_{e_G} = \frac{\omega n_e}{c}\sin\alpha = k_{r_G} 
= \frac{\omega n_e}{c}\sin\alpha'$, umgeschrieben das
\emph{Reflexionsgesetz}\index{Reflexionsgesetz}
\begin{gather*}
  \sin\alpha = \sin\alpha'
\end{gather*}
Die Oberflächennormalen $\vece_y$ und $\veck_e$ spannen die
Einfallsebene auf, in der auch $\veck_r$ liegen muss (d.\,h. keine
Seitwärtsreflexion).

\paragraph{Transmittierter Strahl} Für den transmittierten Strahl
erhält man analog $\veck_{e_r} = \veck_{t_G}$ und daraus wieder mit
der Dispersionsrelation
$k_{e_G} = \frac{\omega n_e}{c}\sin\alpha 
= k_{t_G} =  \frac{\omega n_t}{c}\sin\beta$, also das sog.
\emph{Snellins'sche  Brechungsgesetz}%
\index{Brechungsgesetz!Snellins'sches Brechungsgesetz}
\begin{gather}
  n_e\sin\alpha = n_t\sin\beta
  \label{brechungsgesetz}
\end{gather}
%% IMAGE MISSING

Es gilt anschaulich
\begin{description}
\item[$n_e<n_t$] zum Lot hin gebrochen
\item[$n_e>n_t$] vom Lot weg gebrochen
\end{description}
und mit relativem Brechungsindex $n_{et}=\frac{n_e}{n_t}$ lässt sich
\eqref{brechungsgesetz} umformen zu
\begin{gather*}
  \sin\alpha = n_{et} \sin\beta
\end{gather*}

% ---------------

\section{Die Fresnel'schen Formeln}
Die in diesem Kapitel behandelten 
\emph{Fresnel'schen Formeln}\index{Fresnel'sche Formeln}
beschreiben den Reflexionsgrad einer Grenzfläche.
Dieser hängt von der \emph{Polarisation} ab.

\subsection{Linear Polarisiertes Licht}%
\index{Polarisation!linear polarisiertes Licht}
Es sei $\vecE\bot\vecB$ und $\vecE,\vecB\bot\veck$.
Definiere $\veck= (0,0,k_z)$, dann liegt $\vecE$ in der $x$-$y$-Ebene
mit
\begin{gather*}
  \vecE = \vecE_x + \vecE_y 
  =\begin{pmatrix}
    E_{x,0} \cos(k_z z - \omega t)\phantom{~+ \varphi}\\
    E_{y,0} \cos(k_z z - \omega t + \varphi)\\
    0
  \end{pmatrix}
\end{gather*}
Man sieht, dass eine Phasendifferenz zwischen $E_x$ und $E_y$
möglich ist. Für $\phi=0$ bzw. $\phi=\pm m\pi$ ($m\in\mathds{N}$) gilt
\begin{gather*}
  \vecE = \begin{pmatrix} E_{x,0}\\E_{y,0}\\0 \end{pmatrix}
  \cos(k_zz - \omega t)
  = \vecE_0  \cos(k_zz - \omega t)
\end{gather*}
Die Richtung von $\vecE_0$ ist nicht zeitabhängig.

% ---

\subsection{Zirkular polarisiertes Licht}%
\index{Polarisation!zirkular polarisiertes Licht}
Zirkular polarisiertes Licht ist der Spezialfall $E_{x,0} = E_{y,0} = E_0$
und $\varphi = \frac{\pi}{2} + m\pi$ (wieder $m\in\mathds{N}$), in
Formel
\begin{gather*}
  \vecE = E_0\begin{pmatrix} 
    \cos(k_zz - \omega t)\phantom{~+ \frac{\pi}{2} + m\pi}\\
    \cos(k_zz - \omega t + \frac{\pi}{2} + m\pi)\\
    0 
  \end{pmatrix}
  = E_0\begin{pmatrix} 
    \phantom{\pm}\cos(k_zz - \omega t)\\
    \pm\sin(k_zz - \omega t)\\
    0 
  \end{pmatrix}
\end{gather*}
Hier ist der Betrag der Feldstärke zeitlich konstant und der
$\vecE$-Vektor beschreibt eine Helixbahn (s. Folien), während $m$ den
Drehsinn bestimmt. Die Unterscheidung zwischen rechts und links zirkular ist:
\begin{description}
\item[rechts zirkulares Licht] Blick zur Lichtquelle,
  $\vecE$ rotiert im Uhrzeigersinn
\item[links zirkulares Licht] Blick zur Lichtquelle,
  $\vecE$ rotiert gegen den Uhrzeigersinn
\end{description}

% ---

\subsection{Elliptisch polarisiertes Licht}%
\index{Polarisation!elliptisch polarisiertes Licht}
Hier gilt $E_{x,0} \neq E_{y,0}$ und $\varphi$ ist beliebig.
Dann Beschreibt der $\vecE$-Vektor eine Ellipse in $x$-$y$-Richtung.

\subsection{Fresnel'sche Formeln}
Bisher wurden nur die Phase und damit die Ausbreitungsvektoren
betrachtet.
Jetzt betrachten wir die Amplituden.
Dazu spaltet man die Felder in Komponenten parallel und senkrecht zur
Einfallsebene auf:
\begin{description}
\item[$E_s$ bzw. $E_\bot$] Feldvektor schwingt senkrecht zur
  Einfallsebene; $E_s$ ist automatisch tangential zur Grenzfläche
\item[$E_p$ bzw. $E_\parallel$] Feldvektor schwingt in der Einfallsebene
\end{description}

% ---

\subsection{Senkrechter Lichteinfall}
Wir nehmen für senkrechten Lichteinfall folgende Randbedingungen für
das elektrische und magnetische Feld an
\begin{align}
  \vecE_{0_e} + \vecE_{0_r} = \vecE_{0_t} \label{randbed1}\\
  \vecH_{0_e} + \vecH_{0_r} = \vecH_{0_t} \label{randbed2}
\end{align}
Verwendet man die MWGl
$\vecna\times\vecE = -\dif[\vecB]{t} = -\muo\mu\dif[\vecH]{t}$
und setzt alles in die ebene Welle ein, erhält man
\begin{gather*}
  \vecE(\vecr,t) = \vecE_0 e^{i(\veck\vecr-\omega t)} 
  \qquad{und}\qquad
  \vecB(\vecr,t) = \vecB_0 e^{i(\veck\vecr-\omega t)} 
\end{gather*}
sowie
\begin{align*}
  \begin{pmatrix}
    \dif[E_{0,z}e^{i(\veck\vecr-\omega t)}]{y}
    - \dif[E_{0,y}e^{i(\veck\vecr-\omega t)}]{z}\\
    \dif[E_{0,x}e^{i(\veck\vecr-\omega t)}]{z}
    - \dif[E_{0,z}e^{i(\veck\vecr-\omega t)}]{x}\\
    \dif[E_{0,y}e^{i(\veck\vecr-\omega t)}]{x}
    - \dif[E_{0,x}e^{i(\veck\vecr-\omega t)}]{y}\\
  \end{pmatrix}
&= - \dif[\vecB_0e^{i(\veck\vecr-\omega t)}]{t}\\
  \Longleftrightarrow
  \begin{pmatrix}
    E_{0,z}\cdot ik_y - E_{0,y} \cdot ik_z\\
    E_{0,x}\cdot ik_z - E_{0,z} \cdot ik_x\\
    E_{0,y}\cdot ik_x - E_{0,x} \cdot ik_y
  \end{pmatrix}
& = -\vecB_0\cdot (-i\omega)
\end{align*}
Also
\begin{align}\nonumber
  \veck\times\vecE_0 &= \omega\vecB \qquad\qquad\text{bzw.}\\
  \vecB_0 &= \frac{1}{\omega}(\veck\times\vecE_0) \label{randbed3}
\end{align}
Setze \eqref{randbed3} in \eqref{randbed2} ein und erhalte
\begin{gather*}
  \frac{1}{\omega}(\veck_e\times\vecE_{0,e}) 
  +   \frac{1}{\omega}(\veck_r\times\vecE_{0,r})
  = \frac{1}{\omega}(\veck_t\times\vecE_{0,t}) 
\end{gather*}
da $\veck\bot\vecE$ und $\veck_e = -\veck_r$, 
$\veck_t = \frac{n_t}{n_e}\veck_e$. Daraus erhält man
\begin{gather*}
  n_e\vecE_{0,e} -   n_r\vecE_{0,r} =   n_t\vecE_{0,t}
\end{gather*}
Eliminiere $\vecE_{0,t}$ durch Einsetzen von \eqref{randbed1}
\begin{align*}
  \vecE_{0,r} &= \frac{n_e-n_t}{n_e+n_t} \vecE_{0,e} = r\vecE_{0,e}
  &r &= \frac{n_e-n_t}{n_e+n_t} 
  &&\text{(Reflexionskoeffizient)}\\
  \vecE_{0,t} &= \frac{2n_e}{n_e+n_t} \vecE_{0,e} = t\vecE_{0,e}
  &r &= \frac{2n_e}{n_e+n_t} 
  &&\text{(Transmissionskoeffizient)}
\end{align*}%
\index{Reflexionskoeffizient}%
\nomenclature{$r$}{Reflexionskoeffizient;
  $r=\frac{n_e-n_t}{n_e+n_t}$}%
\index{Transmissionskoeffizient}%
\nomenclature{$t$}{Transmissionskoeffizient; $t=\frac{2n_e}{n_e+n_t}$}%

% 04.11.2015 MISSING

% 09.11.2015

%% Anmerkung letzte Vorlesung
% \begin{align*}
    %     E_{eT} &= E_e\cos\alpha\\
    %     E_{tT} &= E_t\cos\beta\\
    %     E_{rT} &= \mathbf{-} E_r\cos\alpha
                   %   \end{align*}
                   %                    \begin{align*}
                   %                    E_{eN} &= E_e\sin\alpha
                                                 %                  &E_{tN} &= E_t\sin\beta
                                                                              %                                                                     &E_{rN} &= E_r\sin\alpha
                                                                                                                                                              %   \end{align*}
                                                                                                                                                              %                                                                                                                                                               \begin{align*}
                                                                                                                                                              %                                                                                                                                                               E_{eT} + E_{rT} &= E_{rT}\\
    %     \cos\alpha(E_{e}-E_t) &= E_t\cos\beta
                                  %   \end{align*}

                                  %                                   2.
\paragraph{Reflexionsgrad bei Lichteinfall aus optisch dichteren
  Medien}
Wir betrachten $n_e>n_t$ und beobachten, dass ab einem
Einfallswinkel $\alpha=\alpha_T<\ang{90}$ das Reflexionsvermögen
100\% erreicht, d.h. \emph{Totalreflexion}\index{Totalreflexion}.
Aus Snell erhalten wir $\sin\beta = 1$, also für den Winkel
$\alpha_T$%
\nomenclature{$\alpha_T$}{Winkel für Totalreflexion}
der Totalreflexion
\begin{align*}
  \sin\alpha_t &= \frac{n_t}{n_e} \quad\text{bzw.}\\
  \alpha_T &= \arcsin(\frac{n_t}{n_e})
\end{align*}
Für Winkel $\alpha>\alpha_T$ gibt es nach Snell keine Lösung
für $\beta$. Der transmittierte Wellenvektor $\veck_t$ besitzt
keine reelle Komponente senkrecht zur Grenzfläche.
Die imaginäre Komponente führt zu Absorption! Wir erhalten eine
\emph{evaneszente Welle}.

Die Fresnel'schen Formeln können weiterhin verwendet
werden. Ersetze hierfür $\cos\beta = \sqrt{1-\sin^2\beta}$ durch
die Snell'sche Formel
\begin{gather*}
  \cos\beta = \sqrt{
    1 - \left(\frac{n_e}{n_t}\right)^2 \sin\alpha
  }
\end{gather*}
Für $\alpha>\alpha_T$ wird dieser Ausdruck rein imaginär.

Der Reflexionsgrad (für Intensitäten) ist $R_\bot=R_\parallel=1$
Die Koeffizienten $r_\parallel$ und $r_\bot$ werden ebenfalls komplex.
Diese komplexen Amplitudenkoeffizienten verursachen eine
\emph{Phasenverschiebung $\varphi_r$}
bei Totalreflexion, die polarisationsabhängig ist.

Anwendungen sind z.\,B.
\begin{itemize}
\item Fresnel-Rhombus (Polarisationsdreher)
\item Umlenkprisma: Einlaufendes linear polarisiertes Licht wird
  nach zweimaliger Reflexion zirkular polarisiert.
  %% IMAGE MISSING
  Der Grund dafür ist, dass die relative Phase von s- und
  p-polarisierten Komponenten des $\vecE$-Feldes sich ändert.
\end{itemize}

% 3.
\paragraph{Lichtwellenleiter}\index{Lichtwellenleiter}
(s. Folien)\\
Lichtwellenleiter sind wichtig für die Telekommunikation. 
Anwendungen sind u.\,a.
\begin{itemize}
\item Mono-Moden Fasern
\item Multi-Moden Fasern
\item Bildübertragung
\end{itemize}

%---

\subsection{Totalreflexion und evaneszente Welle}
%% IMAGE MISSING
Trifft eine ebene Welle auf eine Grenzschicht,
erhält man ein Interferenzbild durch Überlagerung von einlaufender und
reflektierter Welle im ersten Medium (aus dem die einlaufende Welle kommt).
Im zweiten Medium beobachtet man eine stetige Abnahme der Feldstärke
in $y$-Richtung. Um dies zu beschreiben, verwende
\begin{itemize}
\item $\veck_e$ hat die Komponenten 
  \begin{align*}
    k_{ex} &= k_e\sin\alpha 
    &k_{ey} &= k_e\cos\alpha
  \end{align*}
\item Für $\veck_t$ gilt
  \begin{gather*}
    k_{tG} = k_{eG} = \frac{\omega n_e}{c}\sin\alpha
  \end{gather*}
  Mit der Dispersionsrelation folgt
  \begin{gather*}
    k_t = \frac{\omega n_t}{c} 
    = \sqrt{k_{tG}^2 + k_{ty}^2}
  \end{gather*}
\end{itemize}
Hiermit lässt sich die Komponente $k_{ty}=k_{t,\bot}$ berechnen als
\begin{gather*}
  k_{t,\bot}^2 = \left(\frac{\omega n_t}{c}\right)^2 - k_{tG}^2
  = \frac{\omega^2}{c^2} \left( n_t^2 - n_e^2\sin^2\alpha \right)
\end{gather*}
Für $\alpha>\alpha_T$ gilt wegen Snell ($n_e\sin\alpha > n_t$),
dass $k_{t,\bot}$ rein imaginär wird:
\begin{gather*}
  k_{t,\bot} 
  = \pm ik_{t} \sqrt{\frac{n_e^2}{n_t^2}\sin^2\alpha - 1}
  = \pm i\beta
\end{gather*}
Die Oberflächenwelle wird dann beschrieben durch
\begin{gather*}
  \vecE(x,y,t) = 
  \underbrace{\vecE_{0,t}\exp(-\beta y)}_{\mathclap{\substack{
        \text{exponentiell gedämpft}\\\text{in $y$-Richtung}
      }}
  }
  \cdot 
  \underbrace{\exp(ik_{tG}x - i\omega t)}_{\mathclap{\substack{
        \text{ebene Welle}\\\text{in $x$-Richtung}
      }}
  }
\end{gather*}

Wir betrachten das Beispiel $n_e=1,5$, $n_t=1$, 
also $\alpha_T=\ang{41.8}$, für eine Welle mit Wellenlänge
$\lambda = \SI{600}{\nano\meter}$.
Hier erhält man $\beta = \SI{3.7e3}{\per\milli\meter}$
bzw. $\frac{1}{\beta}\approx\frac{\lambda}{2}$.
D.\,h. die evaneszente Welle klingt auf der Längenskala der Wellenlänge
ab (hier ca. $\SI{300}{\nano\meter}$).

%%% Local Variables:
%%% mode: latex
%%% TeX-master: "../OptikSkriptWS1516"
%%% End:
