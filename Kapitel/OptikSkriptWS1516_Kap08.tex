\chapter{Wellenoptik}
In der geometrischen Optik wird angenommen, dass wir immer einen
Fokus im Punkt erreichen können, d.\,h. unendlich gute Ortsauflösung.
In der Realität liegt der mögliche Fokus im Bereich der Wellenlänge.

Weiterhin betrachte das Gedankenexperiment, bei dem ein durch eine Blende
erzeugter, rechteckiger Lichtstrahl auf einem Schirm aufgefangen wird.
%% IMAGE MISSING
Das Bild ist kein exaktes Rechteck, denn durch Interferenz zeigt sich
eine Modulation der Intensitätsverteilung. Im Randbereich sorgt die
abnehmende Intensität dafür, dass es keinen scharfen Rand gibt, das
Rechteck wird zu einem unscharfen Fleck.
Allgemein beobachtet man bei Schattenwurf mit monochromatischem Licht
an einer scharfen Kante Licht an Orten, wo nach klassischer
geometrischer Optik keines sein sollte.
Daraus sieht man, dass die Wellennatur des Lichts zur
Abweichung von der geradlinigen Ausbreitung (\emph{Beugung} an
der Blende) und zu Intensitätsmodulationen am Schirm
(\emph{Interferenz}) führt.


%%% Local Variables:
%%% mode: latex
%%% TeX-master: "../OptikSkriptWS1516"
%%% End:
