\chapter{Geometrische Optik}
Geometrische Optik
Die \emph{geometrische Optik}\index{geometrische Optik}
bzw. \emph{Strahlenoptik}\index{Strahlenoptik} befasst sich mit der
Darstellung komplexer optischer Instrumente.
%% IMAGE MISSING
Es wird der Grenzfall betrachtet, dass die Dimension des Problems sehr
viel größer sein muss, als die Wellenlänge.
Zur Vereinfachung wird die Wellenfront einer punktförmigen Lichtquelle
betrachtet und ein Lichtbündel bzw. ein Lichtstrahl durch Einfügen
einer Blende erzeugt. Beugungseffekte werden vernachlässigt.
%% IMAGE MISSING

\section[Fermat'sches Prinzip]
{Grundgesetze der geometrischen Optik/Fermat'sches Prinzip}
Für die geometrische Optik werden weiterhin folgende Annahmen gemacht:
\begin{itemize}
\item geradlinige Ausbreitung von Licht in einem homogenen Medium
\item Reflexionsgesetz und Snellius'sches Brechungsgesetz gelten
\item Strahlengänge sind umkehrbar
\item sich durchdringende Strahlen beeinflussen sich nicht
\end{itemize}
Dann gilt das \emph{Fermat'sche Prinzip}\index{Fermat'sche Prinzip}
(auch Variationsprinzip\index{Variationsprinzip}):
Lichtausbreitung erfolgt so, dass der optische Weg $W(s)$%
\nomenclature{$W(s)$}{optischer Weg im Fermat'schen Prinzip;
  $W(s) = \int_{s(Q\to P)}n(\vecx)\dd s$}
(Produkt aus Brechungungsindex $n(\vecr)$ und zurückgelegter Strecke
$s$) einen Extremwert besitzt:
\begin{gather*}
  W(s) = \int_{s(Q\to P)}n(\vecx)\dd s
  \qquad\text{und}\qquad
  \left( \dif[W(s)]{s} \right)_{s_0} = 0
\end{gather*}%
\nomenclature{$s$}{Strecke (Pfad) zwischen zwei Punkten}%
für eine Strecke (einen Pfad) $s$ vom Punkt $Q$ zum Punkt $P$.

Bemerkung dazu:
\begin{itemize}
\item Das Fermat'sche Prinzip lässt sich aus der Wellentheorie des
  Lichts herleiten.
\item Aus der obigen Gleichung folgt die Umkehrbarkeit des Lichtweges.
\item Es kann ein Minimum \emph{oder} ein Maximum sein (fast immer ein
  Minimum)
\end{itemize}
%% IMAGE MISSING

%------------
% 26.11.2015
%------------

\minisec{Strahlablenkung durch ein Prisma}
Wir betrachten als Beispiel ein Glasprisma ($n_1=n_\text{Luft}=1$,
$n_2=n_\text{Glas}\approx 1,5$) und verwenden das Snellius'sche Gesetz
$n_1\sin\theta_1 = n_2\sin\theta_2$.
Die Fragestellung ist nun: Wie hängt die Ablenkung $\delta$ von
$\alpha_1$ ab?
%% IMAGE MISSING (Prisma Skizze)

Generell gilt $\delta = (\alpha_1-\beta_1)+(\alpha_2-\beta_2)$.
und $(\eps+(\ang{90}-\beta_1)+(\ang{90}-\beta_2))=\ang{180}$.
Aus Snellius folgt
\begin{gather*}
  \delta = \alpha_1 - \eps
  + \arcsin\left(
    \sin\eps\sqrt{n^2-\sin^2\alpha_1}-\sin\alpha_1\cos\eps
    \right)
\end{gather*}
Also, da $\beta_1$, $\beta_2$ durch den Schliff des Prismas gegeben
sind, ist $\delta=\delta(\alpha_1)$ von $\alpha_1$ (d.\,h. dem
Einfallswinkel) abhängig.

Es kann gezeigt werden, dass es einen minimalen Ablenkwinkel gibt, der
für Spektroskopie als Fixpunkt genutzt werden kann bei symmetrischem
Strahlengang. Er liegt bei (Kleinwinkelnäherung)
%% MISSING
\begin{align*}
  \delta_\text{min} 
  = 2\alpha_1-\eps 
  &= 2\arcsin\left(n\sin\left(\frac{\eps}{2}\right)\right)\\
  &\approx 2\arcsin\left(n\frac{\eps}{2}\right)\\
  &\approx n\cdot\eps - \eps \\
  &= \eps(n-1)
\end{align*}


%-----

\section{Die optische Abbildung}
\subsection{Paraxiale Objekte}
Eine Abbildung schickt mathematisch gesehen jeden Punkt im Objektraum
auf einen Punkt im Bildraum. Allerdings gibt es folgende Unterschiede
zwischen dieser Idealisierung und der Realität:
\begin{description}
\item[ideale opt. Abb.] Hier wird angenommen:
  \begin{itemize}
  \item maßstabsgetreu (z.\,B. 8:1)
  \item umkehrbar (Umkehrbarkeit der Lichtwege)
  \end{itemize}
\item[real phys. Abb.] Hier gilt
  \begin{itemize}
  \item Abbildungsfehler
  \item keine punktgenaue Abbildung möglich, d.\,h. das Bild eines
    Punkts verschmiert in ein Scheibchen, es gibt also eine
    Verbreiterung bzw. Unschärfe
  \end{itemize}
\end{description}
Die Näherung einer idealen optischen Abbildung trifft am ehesten zu,
wenn man sich in der Nähe der optischen Achse aufhält bzw. bei kleinen
Winkeln. Denn dann gilt die Näherung $\tan\theta=\sin\theta=\theta$.
Dieses Gebiet der Optik wird \emph{paraxiale Optik}\index{paraxiale
  Optik} oder \emph{Gaußsche Optik}\index{Gaußsche Optik} genannt.

\minisec{Reelle und virtuelle Abbildung}
Grundsätzlich soll alles Licht, das von einem Objektpunkt ausgeht, in
einem lagerichtigen Bildpunkt zusammengeführt werden.
Dazu gibt es zwei Möglichkeiten:
\begin{description}
\item[reelle Abbildung:] Die Abbildung kann direkt beobachtet und auf
  einen Schirm abgebildet werden.
%% IMAGE MISSING Bildpunkt
\item[virtuelles Bild:] Die Lichtstrahlen scheinen von einem Punkt zu
  kommen, können aber nicht auf einem Schirm abgebildet werden.
  Ein Beispiel ist ein Hohlspiegel.
%% IMAGE MISSING virtueller Bildpunkt
\end{description}


\subsection{Abbildungen mit Kugelspiegel}
siehe Übungsaufgaben

\subsection{Abbildung durch brechende Kugelflächen}
Die Fragestellung ist, wie sich eine Abbildung realisieren lässt.
%% IMAGE MISSING
Man kann sphärisch gekrümmte Grenzflächen zwischen zwei Bereichen mit
unterschiedlichen Brechungsindices $n_1$ und $n_2$ verwenden.
Wir betrachten die paraxiale Näherung, also kleine Winkel.
%% IMAGE MISSING Strahlengang bei sphärischer Grenzfläche
Damit wird das Snellius'sche Gesetz $n_1\sin\theta_e =
n_2\sin\theta_t$ zu $n_1\theta_e=n_2\theta_t$.
Ersetzen wir hier $\theta_e=\gamma+\alpha$ und $\theta_t=\alpha-\beta$%
\footnote{{} $\alpha$: Winkel zwischen optischer Achse und Lot auf die sphärische
Grenzfläche am Grenzpunkt des Lichtstrahls;\\
$\gamma$: Winkel zwischen optischer Achse und Ausbreitungsrichtung des
einfallenden Lichtstrahls;\\
$\beta$: Winkel zwischen optischer Achse und Ausbreitungsrichtung des
transmittierten Lichtstrahls;}
erhalten wir
\begin{align*}
  n_1\cdot(\gamma+\alpha) &= n_2\cdot(\alpha-\beta)\\
\end{align*}
und mit
$\gamma = \frac{h}{g}$,
$\alpha =\frac{h}{r}$ und
$\beta = \frac{h}{b}$%
\footnote{{} $h$: Höhe des Grenzpunktes über der optischen Achse;\\
$g$: Entfernung vom Objektpunkt (auf der opt. Achse) zum Lotpunkt des
Grenzpunkts auf der opt. Achse;\\
$b$: Entfernung vom Bildpunkt (auf der opt. Achse) zum Lotpunkt des
Grenzpunkts auf der opt. Achse;\\
$r$: Radius der sphärischen Krümmung}
dann insgesamt
\begin{gather*}
  n_1\left(\frac{h}{g} + \frac{h}{r}\right)
  = n_2\left(\frac{h}{r} - \frac{h}{b}\right)
\end{gather*}
Ein wenig umgeschrieben erhalten wir die \emph{Abbildungsgleichung für
  eine brechende Kugelfläche}\index{Abbildungsgleichung!Kugelfläche}
\begin{gather}
  \frac{n_1}{g} + \frac{n_2}{b} = \frac{n_2-n_1}{r}
  \label{abbgleichung}
\end{gather}

%------------
% 30.11.2015
%------------

\minisec{Brennweite}
Lässt man die Gegenstandsweite $g$\nomenclature{$g$}{Gegenstandsweite}
gegen unendlich gehen (also geht man von Einfall paralleler Strahlen
aus), wird Gleichung \eqref{abbgleichung} zu
\begin{gather*}
  \frac{n_2-n_1}{r}
  = \frac{n_1}{g} + \frac{n_2}{b}
  \overset{g\to\infty}{\longrightarrow}
  \underbrace{\frac{n_1}{\infty}}_{0} + \frac{n_2}{b} 
\end{gather*}
und wir erhalten durch Umformung die 
\emph{bildseitige Brennweite}\index{Brennweite!bildseitig}
$b$\nomenclature{$b$}{bildseitige Brennweite}
\begin{gather*}
  b = f_B = \frac{n_2\cdot r}{n_2-n_1}
\end{gather*}
Analog erhält man die
\emph{gegenstandsseitige Brennweite}%
\index{Brennweite!gegenstandsseitig}
$g$\nomenclature{$g$}{gegenstandsseitige Brennweite}
\begin{gather*}
  g = f_G = \frac{n_1\cdot r}{n_2-n_1}
\end{gather*}

Bisher haben wir nur eine halbe Linse betrachtet, also nur einen
sphärischen Übergang von zwei Materialien mit unterschiedlichem
Brechungsindex.
Für eine richtige Linse benötigt man einen Übergang von einem Medium
(Brechungsindex $n_1$) in die Linse (Brechungsindex $n_2$) und wieder
in ein anderes Medium (Brechungsindex $n_3$). 
Wir betrachten wieder sphärische Grenzflächen von Medium 1 und 3 zum
Linsenmedium. Wir erhalten:
\begin{align*}
  \frac{n_1}{g_1} + \frac{n_2}{b_1} &= \frac{n_2-n_1}{r_1}
  &&\text{1. Kugelfläche}\\
  \frac{n_2}{g_2} + \frac{n_3}{b_2} &= \frac{n_3-n_2}{r_2}
  &&\text{2. Kugelfläche}
\end{align*}
wobei $g_2 = -b_1$ für $d\ll r_1,r_2$ (d.\,h. der Abstand der beiden
Linsengrenzflächen ist wesentlich kleiner als die Grenzflächenradien).
Ineinander eingesetzt ergibt das
\begin{gather*}
  \frac{n_1}{g_1} + \frac{n_3}{b_3} 
  = \frac{n_2-n_1}{r_1} + \frac{n_3-n_2}{r_2}
\end{gather*}
Für $n_1=n_3=n_\text{Luft}=1$ und
mit $b\coloneqq b_2$ und $g\coloneqq g_1$ erhalten wir die
\emph{allgemeine Linsenmachergleichung}%
\index{Linsenmachergleichung!allgemein}
\begin{gather*}
  \frac{1}{g} + \frac{1}{b}
  = (n_2-1)\left( \frac{1}{r_1}-\frac{1}{r_2} \right)
  \eqqcolon \frac{1}{f}
\end{gather*}
$f$ ist die Brennweite der Linse (auf beiden Seiten gleich) und alle
zur optischen Achse parallele Strahlen verlaufen nach Durchgang durch
die Linse durch einen der zwei Brennpunkte
(Sammellinse: Brennpunkt auf der Austrittsseite, Zerstreuungslinse:
Brennpunkt auf der Eintrittsseite).

\red{Achtung:} Die Brennweite $f$ ist (unter den genannten Näherungen)
nur abhängig von $\left( \frac{1}{r_1}-\frac{1}{r_2} \right)$ und auf
beiden Linsenseiten gleich!

\section{Optische Geräte}
\subsection{Sammel- und Zerstreuungslinsen}
\minisec{Sammellinsen}
%% IMAGE MISSING verschiedene Gegenstands- und Bildweiten;
% Konstruktion von Bildebenen für geg. Brennweite
Bei gegebener Linse mit best. Brennweite ist die Bildweite (die Ebene,
in der Punkte scharf auf Punkte abgebildet werden) von der
Gegenstandsweite abhängig.
Je größer die Gegenstandsweite, desto kleiner die Bildweite.

%% IMAGE MISSING Unschärfe eines Bildes mit und ohne Blende
Der Grad der Unschärfe ist bestimmt durch die Größe des
Bildscheibchens, auf das ein Gegenstandspunkt abgebildet wird. Die
Bildscheibchengröße wiederum ist bestimmt durch den Winkel, den zwei
vom Gegenstandspunkt ausgehende Strahlen einschließen können.
Dieser Winkel kann durch das einsetzen einer Blende wesentlich
verringert werden.

\minisec{Zerstreuungslinsen}
Eine Zerstreuungslinse erhält man, wenn die Grenzflächen der Linse
nach innen sphärisch sind anstatt nach außen.
Dann durchläuft ein zur optischen Achse paralleler Strahl nach
Durchtritt durch die Linse nicht mehr den Brennpunkt auf der Austrittsseite,
sondern seine Verlängerung verläuft durch den Brennpunkt auf der Eintrittsseite.
Ein Betrachter sieht ein virtuelles Bild, das weiter als die
Gegenstandsweite weg zu sein scheint.

\subsection{Lupe}
Der Betrachtungswinkel $\eps_0$ (bzgl. der optischen Achse), den unser
Auge für einen Gegenstand der Größe $G$ wahrnimmt
bzw. der Größe $\frac{G}{2}$ von der opt. Achse aus gemessen,
lässt sich aus dem Winkel $\eps$ zwischen der optischen Achse und
einem Strahl von der Gegenstandsspitze durch die Linsenmitte
bestimmen.
Für $\eps$ gilt
\begin{gather*}
  \tan(\eps) = \frac{G}{2 }\cdot\frac{1}{f} =\frac{G}{2f} \approx
  \epso
\end{gather*}
Dann ist $\epso$ ohne Lupe (also nur die Linse des Auges) und mit
$s_0=f_\text{Auge}\approx \SI{25}{\centi\meter}$
\begin{gather*}
  \eps_0 = \frac{G}{2 s_0}
\end{gather*}
Damit ist der Vergrößerungsfaktor einer weiteren Sammellinse, die man
vor das Auge schiebt,
\begin{gather*}
  V = \frac{\eps}{\epso} = \frac{s_0}{f}
\end{gather*}

%------------------------
% 3.12.2015
%------------------------
\subsection{Das Teleskop bzw. Fernrohr}
Für weit entfernte Objekte kann man mit 2 Linsen gute Vergrößerungen erreichen. Der wesentliche Unterschied zur Lupe ist hierbei, dass die Strahlen nahezu parallel aufgrund der großen Entfernung auf die Linse treffen.
\minisec{Funktionsweise:}
Von einem Objekt wird ein verkleinertes auf dem Kopf stehendes reelles Bild auf die Brennebene der ersten Linse projeziert. Dieses wird mit der Lupe (2. Linse) betrachtet.
%% Image missing
Das erste Zwischenbild in der Brennebene hat die Größe:
\begin{align*}
	B=\frac{f_{obj}}{g}G
\end{align*}
B wird also mit $f_{obj}$ größer bzw. wenn man ein größeres Bild haben will, benötigt man eine größere Brennweite. Beispielsweise besitzen Teleobjektive Brennweiten von $\num{100}-\SI{1000}{\milli\meter}$.
Weiter kann man eine Lupe benutzen um dieses reelle Zwischenbild noch weiter zu vergrößern.\\
Nun betrachten wir die Vergrößerung zunächst ohne Lupe:
\begin{align*}
	\fracone{2}\eps&=\frac{B}{S_0}&\fracone{2}\epso&=\frac{G}{g}\\
	=>\quad V_{obj}&=\frac{f_{obj}}{S_0}
\end{align*}
Nun mit Lupe:
\begin{align*}
	V&=\frac{f_{obj}}{S_0}V_{Lupe}=\frac{f_{obj}}{S_0}\frac{S_0}{f_{Lupe}}=\frac{f_{obj}}{f_{Lupe}}\\
	V_{Tele}&=\frac{f_{obj}}{f_{Lupe}}
\end{align*}
Also erhalten wir höchste Vergrößerung, wenn $f_{obj}$ sehr groß und $f_{Lupe}$ sehr klein ist. Allerdings ist das Teleskop für terrestrische Anwendungen nicht geeignet, da das Bild invertiert wird und eine große Baulänge benötigt wird. Man müsste also mit Prismenglas zur Faltung des Strahls arbeiten und bei astronomischen Ferngläsern bekommt man Gewichtsprobleme.

\minisec{Alternativ: Galileisches Fernrohr}
Anstatt einer Sammellinse als Okular wird eine negative Linse, Zerstreuungslinse, mit negativer Brennweite ($f_{okular}<0$) verwendet und \emph{vor} das reelle Bild gebracht. Der Abstand der negativen Linse von der Bildebene soll gleich $\vert f_{okular}\vert$ sein.
Mit entspanntem Auge sieht man ein aufrecht stehendes Bild unter dem Sehwinkel:
\begin{align*}
	\eps&=\epso\frac{f_{obj}}{\vert f_{okular}\vert}
\end{align*}

\subsection{Mondaufgang}
%Image missing
Die Größe G eines Objekts wird durch den Beobachtungswinkel $\eps$ bestimmt.
Wir schätzen die Größe ab, indem wir die Entfernung $g$ zum Objekt schätzen.
\begin{align*}
	G&=g\sin\eps=g\eps
\end{align*}
Zum Beispiel der Mond, der beim Aufgang oberhalb von nahen Objekten zu stehen scheint.
\begin{align*}
	d_{Mond}&=\SI{3,6e6}{\meter}\\
	g&=\SI{3,8e8}{\meter}\\
	\eps&=\frac{d}{g}=\SI{9,05e-3}{\radian}
\end{align*}
Wenn man dies nun mit dem Sehwinkel eines $\SI{10}{\meter}$ hohen Baums in $\SI{1}{\kilo\meter}$ Entfernung vergleicht:
\begin{align*}
	\eps\prime=\SI{10e-2}{\radian}
\end{align*}
Daher sieht der Mensch den Mond in der selben Größe, wenn der Mond nun weiter aufsteigt wird er nicht kleiner, sondern unser Gehirn vergleicht ihn nun mit Objekten mit dem gleichen kleineren Sehwinkel. Beispielsweise einem Baum in $\SI{7}{\kilo\meter}$ Entfernung. Dann erhalten wir einen Durchmesser für den Mond von $d=\SI{64}{\meter}$.

\section{Abbildungsfehler}
\subsection{Chromatische Aberration}
Abbildungsgleichung für dünne Linsen:
\begin{align*}
	\fracone{g}+\fracone{b}=(n(\lambda)-1)\left(\fracone{r_1}-\fracone{r_2}\right)=\fracone{f}
\end{align*}
Also ist die Brennweite $f$ einer einfachen Linse eine monotone Funktion des Brechungsindex.
\begin{align*}
	D(\lambda)=\fracone{f(\lambda)}=(n(\lambda)-1)\rho
\end{align*}
Das bedeutet, dass bei einer dünneren Linse die Brennweite für rotes Licht immer größer ist als für blaues Licht (wegen normaler Dispersion von n).
%Image missing
Es gibt also keinen Ort an dem das\enquote{rote} und das \enquote{blaue} Bild scharf sind. Außerdem tritt die sogenannte \emph{laterale Aberation} auf, nämlich dass die Bilder unterschiedliche Größen haben.
\emph{Es gibt also keine einfachen Linsen ohne Farbfehler.}
Man kann dies durch Kombinationen von Linsen korrigieren. Zum Beispiel durch einen \emph{Achromat}, der aus zwei Linsen aus Glas mit unterschiedlichen Dispersionen besteht.

\subsection{Sphärische Aberration (Öffnungsfehler)}
Die Brennweite hängt vom Abstand der Strahlen von der optischen Achse ab. Der Brennpunkt verschiebt sich also für Strahlen, die weiter Entfernt von der optischen Achse sind, da die Kleinwinkelnäherung hier nicht mehr gültig ist.
Man nennt diese auch \emph{monochromatische Aberration}, da dieser Effekt auch bei monochromatischem Licht auftritt und unabhängig von der Wellenlänge ist.

%%% Local Variables:
%%% mode: latex
%%% TeX-master: "../OptikSkriptWS1516"
%%% End:
