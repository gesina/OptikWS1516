\chapter{Geometrische Optik}
Geometrische Optik
Die \emph{geometrische Optik}\index{geometrische Optik}
bzw. \emph{Strahlenoptik}\index{Strahlenoptik} befasst sich mit der
Darstellung komplexer optischer Instrumente.
%% IMAGE MISSING
Es wird der Grenzfall betrachtet, dass die Dimension des Problems sehr
viel größer sein muss, als die Wellenlänge.
Zur Vereinfachung wird die Wellenfront einer punktförmigen Lichtquelle
betrachtet und ein Lichtbündel bzw. ein Lichtstrahl durch Einfügen
einer Blende erzeugt. Beugungseffekte werden vernachlässigt.
%% IMAGE MISSING

\section[Fermat'sches Prinzip]
{Grundgesetze der geometrischen Optik/Fermat'sches Prinzip}
Für die geometrische Optik werden weiterhin folgende Annahmen gemacht:
\begin{itemize}
\item geradlinige Ausbreitung von Licht in einem homogenen Medium
\item Reflexionsgesetz und Snellius'sches Brechungsgesetz gelten
\item Strahlengänge sind umkehrbar
\item sich durchdringende Strahlen beeinflussen sich nicht
\end{itemize}
Dann gilt das \emph{Fermat'sche Prinzip}\index{Fermat'sche Prinzip}
(auch Variationsprinzip\index{Variationsprinzip}):
Lichtausbreitung erfolgt so, dass der optische Weg $W(s)$%
\nomenclature{$W(s)$}{optischer Weg im Fermat'schen Prinzip;
  $W(s) = \int_{s(Q\to P)}n(\vecx)\dd s$}
(Produkt aus Brechungungsindex $n(\vecr)$ und zurückgelegter Strecke
$s$) einen Extremwert besitzt:
\begin{gather*}
  W(s) = \int_{s(Q\to P)}n(\vecx)\dd s
  \qquad\text{und}\qquad
  \left( \dif[W(s)]{s} \right)_{s_0} = 0
\end{gather*}%
\nomenclature{$s$}{Strecke (Pfad) zwischen zwei Punkten}%
für eine Strecke (einen Pfad) $s$ vom Punkt $Q$ zum Punkt $P$.

Bemerkung dazu:
\begin{itemize}
\item Das Fermat'sche Prinzip lässt sich aus der Wellentheorie des
  Lichts herleiten.
\item Aus der obigen Gleichung folgt die Umkehrbarkeit des Lichtweges.
\item Es kann ein Minimum \emph{oder} ein Maximum sein (fast immer ein
  Minimum)
\end{itemize}
%% IMAGE MISSING

%------------
% 26.11.2015
%------------

\minisec{Strahlablenkung durch ein Prisma}
Wir betrachten als Beispiel ein Glasprisma ($n_1=n_\text{Luft}=1$,
$n_2=n_\text{Glas}\approx 1,5$) und verwenden das Snellius'sche Gesetz
$n_1\sin\theta_1 = n_2\sin\theta_2$.
Die Fragestellung ist nun: Wie hängt die Ablenkung $\delta$ von
$\alpha_1$ ab?
%% IMAGE MISSING (Prisma Skizze)

Generell gilt $\delta = (\alpha_1-\beta_1)+(\alpha_2-\beta_2)$.
und $(\eps+(\ang{90}-\beta_1)+(\ang{90}-\beta_2))=\ang{180}$.
Aus Snellius folgt
\begin{gather*}
  \delta = \alpha_1 - \eps
  + \arcsin\left(
    \sin\eps\sqrt{n^2-\sin^2\alpha_1}-\sin\alpha_1\cos\eps
    \right)
\end{gather*}
Also, da $\beta_1$, $\beta_2$ durch den Schliff des Prismas gegeben
sind, ist $\delta=\delta(\alpha_1)$ von $\alpha_1$ (d.\,h. dem
Einfallswinkel) abhängig.

Es kann gezeigt werden, dass es einen minimalen Ablenkwinkel gibt, der
für Spektroskopie als Fixpunkt genutzt werden kann bei symmetrischem
Strahlengang. Er liegt bei (Kleinwinkelnäherung)
%% MISSING
\begin{align*}
  \delta_\text{min} 
  = 2\alpha_1-\eps 
  &= 2\arcsin\left(n\sin\left(\frac{\eps}{2}\right)\right)\\
  &\approx 2\arcsin\left(n\frac{\eps}{2}\right)\\
  &\approx n\cdot\eps - \eps \\
  &= \eps(n-1)
\end{align*}


%-----

\section{Die optische Abbildung}
\subsection{Paraxiale Objekte}
Eine Abbildung schickt mathematisch gesehen jeden Punkt im Objektraum
auf einen Punkt im Bildraum. Allerdings gibt es folgende Unterschiede
zwischen dieser Idealisierung und der Realität:
\begin{description}
\item[ideale opt. Abb.] Hier wird angenommen:
  \begin{itemize}
  \item maßstabsgetreu (z.\,B. 8:1)
  \item umkehrbar (Umkehrbarkeit der Lichtwege)
  \end{itemize}
\item[real phys. Abb.] Hier gilt
  \begin{itemize}
  \item Abbildungsfehler
  \item keine punktgenaue Abbildung möglich, d.\,h. das Bild eines
    Punkts verschmiert in ein Scheibchen, es gibt also eine
    Verbreiterung bzw. Unschärfe
  \end{itemize}
\end{description}
Die Näherung einer idealen optischen Abbildung trifft am ehesten zu,
wenn man sich in der Nähe der optischen Achse aufhält bzw. bei kleinen
Winkeln. Denn dann gilt die Näherung $\tan\theta=\sin\theta=\theta$.
Dieses Gebiet der Optik wird \emph{paraxiale Optik}\index{paraxiale
  Optik} oder \emph{Gaußsche Optik}\index{Gaußsche Optik} genannt.

\minisec{Reelle und virtuelle Abbildung}
Grundsätzlich soll alles Licht, das von einem Objektpunkt ausgeht, in
einem lagerichtigen Bildpunkt zusammengeführt werden.
Dazu gibt es zwei Möglichkeiten:
\begin{description}
\item[reelle Abbildung:] Die Abbildung kann direkt beobachtet und auf
  einen Schirm abgebildet werden.
%% IMAGE MISSING Bildpunkt
\item[virtuelles Bild:] Die Lichtstrahlen scheinen von einem Punkt zu
  kommen, können aber nicht auf einem Schirm abgebildet werden.
  Ein Beispiel ist ein Hohlspiegel.
%% IMAGE MISSING virtueller Bildpunkt
\end{description}


\subsection{Abbildungen mit Kugelspiegel}
siehe Übungsaufgaben

\subsection{Abbildung durch brechende Kugelflächen}
Die Fragestellung ist, wie sich eine Abbildung realisieren lässt.
%% IMAGE MISSING
Man kann sphärisch gekrümmte Grenzflächen zwischen zwei Bereichen mit
unterschiedlichen Brechungsindices $n_1$ und $n_2$ verwenden.
Wir betrachten die paraxiale Näherung, also kleine Winkel.
%% IMAGE MISSING Strahlengang bei sphärischer Grenzfläche
Damit wird das Snellius'sche Gesetz $n_1\sin\theta_e =
n_2\sin\theta_t$ zu $n_1\theta_e=n_2\theta_t$.
Ersetzen wir hier $\theta_e=\gamma+\alpha$ und $\theta_t=\alpha-\beta$%
\footnote{{} $\alpha$: Winkel zwischen optischer Achse und Lot auf die sphärische
Grenzfläche am Grenzpunkt des Lichtstrahls;\\
$\gamma$: Winkel zwischen optischer Achse und Ausbreitungsrichtung des
einfallenden Lichtstrahls;\\
$\beta$: Winkel zwischen optischer Achse und Ausbreitungsrichtung des
transmittierten Lichtstrahls;}
erhalten wir
\begin{align*}
  n_1\cdot(\gamma+\alpha) &= n_2\cdot(\alpha-\beta)\\
\end{align*}
und mit
$\gamma = \frac{h}{g}$,
$\alpha =\frac{h}{r}$ und
$\beta = \frac{h}{b}$%
\footnote{{} $h$: Höhe des Grenzpunktes über der optischen Achse;\\
$g$: Entfernung vom Objektpunkt (auf der opt. Achse) zum Lotpunkt des
Grenzpunkts auf der opt. Achse;\\
$b$: Entfernung vom Bildpunkt (auf der opt. Achse) zum Lotpunkt des
Grenzpunkts auf der opt. Achse;\\
$r$: Radius der sphärischen Krümmung}
dann insgesamt
\begin{gather*}
  n_1\left(\frac{h}{g} + \frac{h}{r}\right)
  = n_2\left(\frac{h}{r} - \frac{h}{b}\right)
\end{gather*}


%%% Local Variables:
%%% mode: latex
%%% TeX-master: "../OptikSkriptWS1516"
%%% End:
