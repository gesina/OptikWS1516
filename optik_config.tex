\usepackage[utf8]{inputenc}
\usepackage[T1]{fontenc}
\usepackage{babel}
\usepackage{csquotes}
\usepackage{lmodern}
\usepackage{microtype}
\usepackage{makeidx}
\makeindex
% Nomenklatur, d.h. Symbolverzeichnis
% Eintrag wird erstellt durch:
% \nomenclature[<prefix for sorting>]{<symbol>}{<description>}
% !! keine neuen Zeilen im Befehl und alle außen herum auskommentieren !!
%
% Ausgabe mit:
% \printnomenclature
% 
% Erzeugt .nlo Datei mit dem Programmaufruf (Terminal):
% makeindex OptikSkriptWS1516.nlo -s nomencl.ist -o OptikSkriptWS1516.nls
\usepackage[intoc]{nomencl}
\makenomenclature
\usepackage[backend=biber]{biblatex}
\bibliography{optik.bib}

\usepackage{color}
\newcommand*{\red}[1]{\textcolor{red}{#1}}

\usepackage{amsmath}
\usepackage{amssymb}
\usepackage{mathtools}
\usepackage{amsthm}
\usepackage{siunitx}
\usepackage{booktabs}

\usepackage[%
pdftitle={Experimentalphysik III (Wellen und Quanten)},
pdfauthor={Hedwig Werner}]%
{hyperref}
\usepackage{tabularx}

\subject{Vorlesungsmitschrift}
\title{Experimentalphysik III\\ (Wellen und Quanten)}
\subtitle{im WS2015/16 bei Prof. Dr. Christian Back}
\date{Stand: \today}
\author{gesetzt von Hedwig Werner\and Gesina Schwalbe}


%%% Local Variables:
%%% mode: latex
%%% TeX-master: "OptikSkriptWS1516"
%%% End:
