%% ----------------------------------------------
%% 
%% Mitschrift der Vorlesung
%% Experimentalphysik III: Wellen und Quanten 
%% im WS2015/16 an der Uni Regensburg
%% 
%% Kapitel 9: Quantenphänomene: Welle-Teilchen-Dualismus
%% 
%% 
%% gesetzt von Gesina und Hedwig
%% 
%% ----------------------------------------------

\chapter[Quantenphänomene]%
{Quantenphänomene: Welle-Teilchen-Dualismus}

\section[Der Photoeffekt]{Licht als Teilchen: Der Photoeffekt}
Entscheidende Experimente bzw. theoretische Arbeiten zum Verständnis
der Teilchennatur von Licht sind der
\emph{thermische Strahler (Planck 1900)}\index{Thermischer Strahler} 
und der \emph{Photoeffekt (Einstein 1905)}\index{Photoeffekt}.

Der Photoeffekt ist ein qualitatives Experiment.
%% IMAGE MISSING

Erste Versuche zum Photoeffekt (Herz und Hallwachs 1887/88) hatten als
Aufbau eine Metallkathode, die von UV-Licht (Filterung über
Quarzkristall) bestrahlt wird. Bei Bestrahlung kann man einen Strom
zwischen Kathode und einer Anode messen, es werden also Ladungen
freigesetzt.
Das Experiment liefert, dass sichtbares Licht beliebiger Intensität
nicht ausreicht, um Ladungen freizusetzen, nur die Wellenlänge ist
entscheidend.
Daraus kann man schließen, dass das Licht in Quanten auftreten muss.

Die genaue Messung ergibt:
\begin{description}
\item[Variation von $\Phi$:] Mit wachsender Spannung $U$
  steigt der Strom $I(U)$ bis zu einem Sättigungsstrom $I_\text{max}$%
  \nomenclature{$I_\text{max}$}{Sättigungsstrom beim Photoeffekt}, 
  welcher mit $\Phi$ wächst;
  die minimale Spannung $U_0$, bei der Strom gemessen
  wird, bleibt gleich.
  %% IMAGE MISSING

\item[Variation von $\nu$:] Mit sinkender Frequenz $\nu$ steigt die
  minimale Spannung $U_0$; der Sättigungsstrom $I_\text{max}$ bleibt konstant. 
  %% IMAGE MISSING

\item[Variation des Kathodenmaterials:] Bei unterschiedlichem
  Kathodenmaterial ändert sich die minimale Spannung $U_0$.
  %% IMAGE MISSING
\end{description}

Daraus kann man schließen
\begin{itemize}
\item $U_0$ ist unabhängig von der Lichtintensität.
\item $I_\text{max}$ steigt mit der Lichtintensität.
\item $|U_0|$ steigt linear mit der Frequenz $\nu$ an.
\end{itemize}
Eine Erklärung ist, dass durch das einfallende Licht Elektronen aus
der Kathode ausgelöst werden. Diese besitzen eine gewisse kinetische
Energie $E_\text{kin}$, die zum Überwinden der Gegenspannung
eingesetzt wird.
Erst ab der Gegenspannung $U_0(\nu)$ \enquote{schaffen} es die
Elektronen die Anode zu erreichen
\begin{gather*}
  E_{\text{kin,max}}(\nu) = -e U_0(\nu)
\end{gather*}
Die \emph{Einsteingleichung}\index{Einsteingleichung} lautet
\begin{gather*}
  E_{\text{kin,max}}(\nu) = h\nu - A
\end{gather*}
wobei 
$A$\nomenclature{$A$}{Austrittsarbeit beim Photoeffekt; 
  Materialkonstante} 
die materialabhängige Konstante für die sog. \emph{Austrittsarbeit} und
$h=\SI{6.626e-34}{\J\s}$%
\nomenclature{$h$}{Plank'sches Wirkungsquantum; 
  $h=\SI{6.626e-34}{\Joule\second}$}
das Plank'sche Wirkungsquantum ist.

% --------------
% VL 21.01.2016
% --------------
\VL{21.01.2016}

%% IMAGE MISSING: Austrittsarbeit, Zinth, Abb. 5.8
Die Austrittsarbeit $A$ ist abhängig von der Frequenz $\omega$.
Elektronen können nur dann aus dem Festkörper ausgelöst werden, wenn
$A$ überwunden wird.
Der Energieübertrag aus dem Lichtfeld ist \enquote{gequantelt} und die
Intensität ist proportional zur Anzahl der \enquote{Energiequanten
  $h\nu$} (Photonen).
Typische Werte für die Austrittsarbeit sind:
\begin{description}
\item[Alkalimetalle] $A\approx \SI{1.2}{\electronvolt}$
\item[Edelmetalle] $A\approx \SI{5}{\electronvolt}$
\end{description}

% -----

\section{Nachweis von Photonen}
\cite[siehe][Kap. 5.1.4, Halbleitersensoren]{zinth}

% -----

\section{Eigenschaften von Photonen}
Licht ist ein Teilchenstrom von Photonen mit der Geschwindigkeit $c$
und der Ruhemasse $m_0=0$\nomenclature{$m_0$}{Ruhemasse}.

% ---

\subsection{Energie des Photons}\index{Photon!Energie}
Die Energie eines Photons ist
\begin{gather*}
  W_\text{Ph} = h\nu = \frac{hc}{\lambda} = \hbar\omega
\end{gather*}%
\nomenclature{$W_\text{Ph}$}{Energie eines Photons}%

\paragraph*{Beispiel}
Für $\lambda=\SI{600}{\nano\meter}$ gilt
$W_\text{Ph}=\SI{3.3e-18}{\joule}$
und für $\lambda=\SI{0.1}{\nano\meter}$ gilt
$W_\text{Ph}=\SI{12.407}{\kilo\electronvolt}$.

Die Photonenflussdichte $N$\nomenclature{$N$}{Photonenflussdichte}
eines Lichtbündels mit Intensität $I$ ist definiert über
\begin{gather*}
  I = Nh\nu
  \qquad\text{und}\qquad
  N \coloneqq 
  \frac{\text{Zahl der Photonen}}{\text{Fläche}\cdot\text{Zeit}}
\end{gather*}

\paragraph*{Beispiel}
Das Sonnenlicht hat eine Intensität von etwa
\SI{1000}{\watt\per\centi\meter\squared} und somit eine
Photonenflussdichte
$N=\SI{2.77e21}{{\text{Photonen}}\per\square\meter\second}$

% ---

\subsection{Impuls des Photons}\index{Photon!Impuls}
Ein Photon stößt wie ein Teilchen, was u.\,a. zum sogenannten
\emph{Compton-Effekt} führt.
Bei der Behandlung von Licht als eine Welle haben wir den
Photonenimpuls $P_\text{Ph}$ und den Strahlungsdruck $P_S$ kennen
gelernt:
\begin{alignat*}{2}
  P_\text{Ph} &= \frac{h\nu}{c} = \frac{h}{\lambda} = \hbar h\\
  P_S &= \frac{I}{c} = \frac{Nh\nu}{c} = N\cdot P_\text{ph}
\end{alignat*}
und wegen $W_\text{Ph}=h\nu$ ergibt sich die Darstellung
$W_\text{Ph} = P_\text{Ph}\cdot c$.


Die relativistische Beziehung zwischen Energie $W$ und Impuls $p$ für ein
allgemeines Teilchen lautet
\begin{gather*}
  W^2 = (m_0c^2)^2 + p^2
\end{gather*}
Da die tatsächlich Ruhemasse eines Photons verschwindet, muss für das
Photon die \enquote{Masse}
\begin{gather*}
  m_{0,\text{Ph}}= \frac{W_\text{Ph}}{c^2} = \frac{h\nu}{c^2}
\end{gather*}
gesetzt werden.

% ---

\subsection{Eigendrehimpuls des Photons}\index{Photon!Eigendrehimpuls}
Der Eigendrehimpuls des Photons (Spin) ist
\begin{gather*}
  |\vecS_\text{Ph}| = \hbar
\end{gather*}

% -----

\section{Welle-Teilchen-Dualismus}
Es gibt zwei unterschiedliche Bilder von Licht, das Wellen-~und das
Teilchenbild. (Das trifft auch für alle anderen Quantenteilchen zu.)
Welche Anschauung man verwendet, hängt von der Fragestellung ab, siehe
auch \autoref{tab:verwendung-welle-teilchen-bild}.
\begin{table}[tbh]
  \centering
  \captionabove[Anwerwendung des Welle- und des Teilchenbilds]
  {\label{tab:verwendung-welle-teilchen-bild}
    Typische Anwerwendung des Welle- und des Teilchenbilds}
  \begin{tabular}{ll}
    \toprule
    \emph{Fragestellung} & \emph{verwendetes Bild}\\\midrule[\heavyrulewidth]
    Lichtausbreitung, Beugung & Wellenbild \\\midrule
    Lichtabsorption/~-emission & Teilchenbild \\\midrule
    Lokalisierung & Wahrscheinlichkeitsinterpretation\\\bottomrule
  \end{tabular}
\end{table}

\subsection{Beispiel: Doppelspaltexperiment}
Misst man beim Doppelspaltexperiment mit einzelnen Photonen, so
entspricht die statistische Verteilung (also die
Aufenthaltswahrscheinlichkeit) der Intensitätsverteilung beim
normalen Doppelspaltversuch. (Trifft auf alle Quantenteilchen zu.)
Daraus kann man schließen, dass Quantenteilchen auch
Welleneigenschaften haben müssen und man kann einem Teilchen mit
Impuls $P_T$\nomenclature{$P_T$}{Teilchenimpuls} die sogenannte 
\emph{De-Broglie-Wellenlänge}\index{De-Broglie-Wellenlänge}
\begin{gather*}
  \lambda_B=\frac{h}{P_T}
\end{gather*}%
\nomenclature{$\lambda_B$}{De-Broglie-Wellenlänge}%
zuweisen.

% --------------
% VL 28.01.2016
% --------------
\VL{28.01.2016}

\section{Hohlraumstrahlung}
Die Problemstellung hier ist, ein Strahlungsfeld, das im
thermodynamischen Gleichgewicht mit der Umgebung steht, zu
quantifizieren.
%% IMAGE MISSING: Hohlraumstrahler, Zinth, Abb. 5.27

Ein \emph{Schwarzer Strahler}\index{Schwarzer Strahler} ist ein
beliebiger Körper, der im harmonischen Gleichgewicht mit seiner
Umgebung steht, d.\,h. es wird pro Zeiteinheit genauso viel Strahlung
absorbiert wie emittiert.

Allgemein gilt: Ein guter Absorber ist ein guter Emitter.

Ein \emph{Schwarzer Körper}\index{Schwarzer Körper} ist ein Körper,
der alle auf ihn treffende Strahlung vollständig absorbiert.

\subsection{Kirchhoff'sches Strahlungsgesetz}%
\index{Kirchhoff'sches Strahlungsgesetz}
Das \emph{Kirchhoff'sche Strahlungsgesetz} lautet
\begin{align*}
  \frac{\text{Emission}}{s} &= E_\nu(T) \dd\nu \dd\sigma \dd\theta\\
  \frac{\text{Absorption}}{s} &= A_\nu(T)u_\nu(T) \dd\nu \dd\sigma \dd\theta\\
\end{align*}
wobei
\begin{description}
\item[$T$] Temperatur
  \nomenclature{$T$}{Temperatur}
\item[$\dd\sigma$] Fläche
  \nomenclature{$\dd\sigma$}{Fläche}
\item[$\dd\theta$] Raumwinkelelement
  \nomenclature{$\dd\theta$}{Raumwinkelelement}
\item[$\dd\nu$] Frequenzwinkel
  \nomenclature{$\dd\nu$}{Frequenzwinkel}
\item[$u(\nu,T)\dd\nu$] spektrale Energiedichte
  \nomenclature{$u(\nu,T)\dd\nu$}{spektrale Energiedichte}
\item[$u(\nu,T)$] universelle Funktion von Frequenz und Temperatur
\item[$A(\nu,T)$] 
  $\frac{\text{absorbierte Strahlungsleistung}}
  {\text{einfallende Strahlungsleistung}}$ (bei Frequenz $\nu$ und
  Temperatur $T$)%
  \nomenclature{$A(\nu,T)$}{$\frac{\text{absorbierte Strahlungsleistung}}
    {\text{einfallende Strahlungsleistung}}$
    (bei Frequenz $\nu$ und Temperatur $T$)}
\item[$E(\nu,T)$] emittierte Strahlungsleistung
  \nomenclature{$E(\nu,T)$}{emittierte Strahlungsleistung 
    (bei Frequenz $\nu$ und Temperatur $T$)}
\end{description}
Ist ein Körper im thermischen Gleichgewicht, entspricht dies
\begin{gather*}
  E_\nu(T) = u_\nu(T)\cdot A_\nu(T)
\end{gather*}
Für einen Schwarzen Strahler gilt $A_\nu(T)\equiv 1$, alle anderen
Körper haben $A_\nu(T)<1$. Das Emissionsvermögen eines Schwarzen
Strahlers ist also $E_S(\nu,T) = u(\nu,T)$.

Im Allgemeinen ist $u(\nu,T)$ gesucht. Dazu gibt es verschiedene
Ansätze.

\paragraph{Rayleigh-Jeans-Gesetz (1898)}
Dieses Gesetz gilt für 
$h\nu\ll K_B T$\nomenclature{$K_B$}{Boltzmann-Konstante}
und lautet
\begin{gather*}
  u(\nu,T)\dd\nu = \frac{8\pi}{c^3}(K_BT)\nu^2\dd\nu
\end{gather*}
Es hätte eine UV-Katastrophe zur Folge.
Der Ansatz fußt auf der Thermodynamik: 
Die stehenden elektromagnetischen Wellen in einem Hohlraum (z.\,B. Würfel)
sind die Freiheitsgrade des Systems. Der Äquipartitionssatz der
Thermodynamik besagt, dass in jedem Freiheitsgrad die mittlere Energie
$\frac{1}{2}K_BT$ steckt.

\paragraph{Wien'sches Gesetz}
Dieses Gesetz gilt für hohe Frequenzen $h\nu\gg K_BT$ 
und lautet
\begin{gather*}
  u(\nu, T) = A\nu^3e^{-g\frac{\nu}{T}}
\end{gather*}
wobei $A=\frac{8\pi h}{c^3}$ und $g=\frac{h}{K_B}$ Konstanten sind
(die Werte wurden später ermittelt).
Dies beschreibt die Farbänderung der Emission eines glühenden Körpers bei
steigender Temperatur.

\paragraph{Wien'sches Verschiebungsgesetz (1893)}
Dieses Gesetz beschreibt die Lage des Maximums von $u$ als Funktion
der Temperatur
($b=\SI{0.29e-2}{\meter\kelvin}$%
\nomenclature{$b$}{Konstante im Wien'schen Verschiebungsgesetz;
  $b=\SI{0.29e-2}{\meter\kelvin}$}
Konstante):
\begin{gather*}
  \lambda_\text{max}\cdot T = b
\end{gather*}
Z.\,B. hat die Sonnenoberfläche eine Temperatur von
ca. \SI{6000}{\kelvin} und man erhält ein Maximum von $u$ bei
$\lambda=\SI{480}{\nano\meter}$.

\paragraph{Stefan-Boltzmann-Gesetz}
Es gilt für die über alle Frequenzen integrierte Strahlungsleistung
$\widetilde u$\nomenclature{$\widetilde u$}{über alle Frequenzen
  integrierte Strahlungsleistung}
($\sigma = \SI{5.67051e-8}{\watt\per\square\meter\kelvin^{-4}}$%
\nomenclature{$\sigma$}{Stefan-Boltzmann-Konstante; 
  $\sigma = \SI{5.67051e-8}{\watt\per\square\meter\kelvin^{-4}}$}
Stefan-Boltzmann-Konstante):
\begin{gather*}
  \widetilde u(T) = \int u(\nu,T)\dd\nu = \sigma T^4
\end{gather*}

\paragraph{Planck'sches Strahlungsgesetz}
Dieses Gesetz liefert, dass die Photonenenergie quantisiert ist mit
$E=h\nu$. Im folgenden sind die groben Schritte der Herleitung.

Nach Planck gilt die (zunächst empirische) Formel
\begin{gather}
  u(\nu, T) 
  = \frac{8\pi\nu^2}{c^2}
  \cdot \frac{h\nu}{\exp(\frac{h\nu}{kT}}
  \label{planck}
\end{gather}

% --------------
% VL 04.02.2016
% --------------
\VL{04.02.2016}

\subparagraph{Elektromagnetischer Schritt}
Betrachte die Energie des s.\,g. harmonischen Oszillator mit
Eigenfrequenz $\nu$ im harmonischen elektromagnetischen Feld
als kohärente Überlagerung von harmonischen Feldern variabler
Frequenz.
Berechne die Gleichgewichtsenergie $U$ des Oszillators%
\nomenclature{$U$}{Gleichgewichtsenergie eines Oszillators}, wobei
alle Felder entsprechend der spektralen Energiedichte $u$ gewichtet
werden.
Das Resultat ist (empirische Formel, 07.10.1900):
\begin{gather*}
  u(\nu, T) = \frac{8\pi\nu^2}{c^3} U(\nu, T)
\end{gather*}
Es folgt mit \eqref{planck}
\begin{gather}
  U(\nu, T) = \frac{h\nu}{\exp(\frac{h\nu}{uT}) - 1}
  \label{GGWEnergie}
\end{gather}
Jetzt die Bedeutung von Formel~\eqref{GGWEnergie}:

\subparagraph{Thermodynamischer Schritt}
Ausgehend von Gleichung~\eqref{GGWEnergie} berechnet Planck die
Entropie $S$\nomenclature{$S$}{Entropie}, denn es gilt der Zusammenhang
\begin{gather*}
  \dd U = T\dd S - p\underbrace{\dd V}_{=0} \dotsb
  \qquad\text{also} \qquad
  \frac{\dd S}{\dd U} = \frac{1}{T}
  \qquad\text{bzw.} \qquad
  \dd S = \frac{\dd U}{T} 
\end{gather*}
Da mit Gleichung~\eqref{GGWEnergie}
$\frac{1}{T(U)} = \frac{K_B}{h\cdot\nu}\ln\left(\frac{h\nu+U}{U}\right)$,
gilt insgesamt
\begin{gather*}
  S = K_B \left[ 
    \left(1+\frac{U}{h\nu}\right) \ln\left(1+\frac{U}{h\nu}\right)
    + \frac{U}{h\nu}\ln\left(\frac{U}{h\nu}\right)
  \right]
\end{gather*}
Die Entropie \emph{muss} diese Form haben!

\subparagraph{Statistischer Schritt}
Wir behandeln nun die Herleitung der Entropie mit statistischen
Argumenten.
Die Entropie von $N$ unabhängigen Oszillatoren der Frequenz $\nu$
ist
\begin{gather*}
  S_N = K_B\cdot \ln(W_N)
\end{gather*}
wobei $W_N$\nomenclature{$W_N$}{thermodynamische Wahrscheinlichkeit;
  Anzahl der möglichen Verteilungen der Gesamtenergie $U_N$ auf $N$
  Oszillatoren}
die thermodynamische Wahrscheinlichkeit ist, also die Anzahl der
möglichen Verteilungen der Gesamtenergie $U_N$ auf $N$ Oszillatoren.
Die Gesamtenergie ist $U_N=N\cdot
U$\nomenclature{$U_N$}{Gesamtenergie eines Systems mit $N$
  Oszillatoren; $U_N=N\cdot U$}.

\red{Achtung:} Es gilt $W\to\infty$ für eine beliebige,
kontinuierliche Verteilung von $U_N$.

\begin{postulat}[Quantenpostulat]
  Die Oszillatoren dürfen Energie nur in ganzzahligen Vielfachen einer
  Grundeinheit $\eps$ aufnehmen, d.\,h.
  \begin{gather*}
    U_N = p\cdot \eps 
    \qquad \text{mit } p\in\Z, p\gg 1
  \end{gather*}
\end{postulat}

\begin{postulat}[statistisches Postulat]
  Die Anzahl der Möglichkeiten, $p$ ununterscheidbare
  Energiequanten auf $N$ unterscheidbare Oszillatoren zu verteilen,
  ist (Bose-Einstein-Statistik)
  \begin{gather*}
    W_N = \frac{(N+p-1)!}{(N-1)!\cdot p!}
    \overset{\text{Stirling}}{\approx} 
    \frac{(N+p)^{N+p}}{N^N \cdot p^p}
  \end{gather*}
\end{postulat}  

\subparagraph{Beispiel}
Es sei $N=2$, $p=3$ und $W_2=4$.
Dann sind die vier Möglichkeiten, die gesamte Energie $U=3\eps$ auf
die beiden Oszillatoren 1~und 2 zu verteilen, leicht grafisch zu
ermitteln
%% IMAGE MISSING (Verteilung der Energie auf Oszillatoren)
und für die Entropie gilt
\begin{align*}
  S_N &= K_B \left[
        (N+p)\ln(N+p) - N\ln(N) - p\ln(p)
        \right]\\
  &= N \underbrace{
    K_B \left[
    \left(1+\frac{U}{\eps}\right) \ln\left(1+\frac{U}{\eps}\right) 
    - \frac{U}{\eps}\ln\left(\frac{U}{\eps}\right)
    \right]
    }_{\text{Entropie für einen Oszillator}}\\
  &= N\cdot S
\end{align*}




%%% Local Variables:
%%% mode: latex
%%% TeX-master: "../OptikSkriptWS1516"
%%% End:
