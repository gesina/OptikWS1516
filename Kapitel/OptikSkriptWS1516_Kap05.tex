\chapter{Polarisation}

Im allgemeinen Fall ist $E_{ox}\neq E_{oy}$ und $\varphi$
beliebig. Deshalb beschreibt das $\vecE$-Feld eine Ellipse in der
$x$-$y$-Ebene. Stärke und Richtung des $\vecE$-Felds ändern sich als
Funktion der Zeit. Der Endpunkt des $\vecE$-Felds beschreibt eine
Ellipse in der $x$-$y$-Ebene. Für den Winkel $\alpha$ zwischen der
großen Halbachse der Ellipse und der $x$-Achse erhalten wir:
\begin{gather*}
  \tan(2\alpha) 
  = \frac{2E_{ox}E_{oy}\cos\varphi}{E_{ox}^2 + E_{oy}^2}
\end{gather*}
Für spezielle Werte von $\alpha$ wird $\varphi = 0$ und es gilt
\begin{gather*}
  \frac{E_x^2}{E_{ox}^2} + \frac{E_y^2}{E_{oy}^2} = 1
\end{gather*}

\section{Mathematische Formulierung}
Definiere Polarisationszustand eines Feldes als 2D-Vektor mit
komplexen Amplituden, den sogenannten
\emph{Bones-Vektor}\index{Bones-Vektor}:
\begin{gather*}
  \vecE = \begin{pmatrix}E_{ox}\\E_{oy}\end{pmatrix}
\end{gather*}
Normiere nun den Vektor auf Eins, denn oft ist nur der relative Wert
von Interesse:
\begin{gather*}
  \frac{vecE}{E_x} 
  = \begin{pmatrix}
    1\\\frac{E_{oy}}{E_{ox}}
  \end{pmatrix}
\end{gather*}

\minisec{Lineare Polarisation}
Lineare Polarisation lässt sich wie folgt charakterisieren
\begin{itemize}
\item[\ang{0}:] $\frac{E_y}{E_x} = 0$
\item[\ang{90}:] $\frac{E_y}{E_x} = \infty$, d.\,h.
$\scriptsize\begin{pmatrix}E_x\\E_y\end{pmatrix}
= \begin{pmatrix}0\\1\end{pmatrix}$
\item[\ang{45}:] $\frac{E_y}{E_x} = 1$
\item[bel.:] 
  $\frac{E_y}{E_x} = \frac{\sin\alpha}{\cos\alpha} = \tan\alpha$
\end{itemize}

\minisec{Zirkulare und elliptische Polarisation}
Bei \emph{zirkularer Polarisation}\index{Polarisation!zirkular}
hat der Bones-Vektor eine rein imaginäre
$y$-Komponente
\begin{gather*}
  \vecE = \begin{pmatrix}E_x\\E_y\end{pmatrix} 
  = \frac{1}{\sqrt{2}}\begin{pmatrix}1\\\pm i\end{pmatrix}
\end{gather*}
Die Richtungen lassen sich wie folgt unterscheiden:
\begin{description}
\item[rechtszirkular] $\frac{E_y}{E_x}=-i$
\item[linkszirkular] $\frac{E_y}{E_x}=+i$
\end{description}

Bei \emph{elliptischer Polarisation}\index{Polarisation!elliptisch}
hat der Bones-Vektor eine komplexe (nicht rein imaginäre)
$y$-Komponente
\begin{gather*}
  \frac{E_y}{E_x} = a+bi
\end{gather*}

%----------

%5.2
\section{Polarisatoren}
Ein Polarisator\index{Polarisator} ist ein optisches Element,
das aus \enquote{unpolarisiertem} Licht einen definierten
Polarisationszustand selektiert.
Beispiele sind Linearpolarisatoren, Zirkularpolarisatoren oder
elliptische Polarisatoren.
Für die Funktionsweise ist eine \emph{optisch asymmetrische
  Komponente}\index{optisch asymmetrische Komponente} nötig,
die Licht mit ungewünschter Polarisation absorbiert/unterbindet.

Man unterscheidet vier Mechanismen von Polarisatoren:
\begin{enumerate}
\item \emph{Reflexion} (Brewsterfenster, Fresnel-Rhombus)
\item \emph{Streuung} (Hertz'scher Dipol)
\item \emph{richtungsselektive Absorption} (Dichroismus)
\item \emph{Doppelbrechung}
\end{enumerate}

Mit Hilfe eines Polarisators kann man die Polarisationseigenschaften
von Licht bestimmen, ihn also als \emph{Analysator}\index{Analysator}
verwenden.

\paragraph{Beispiel} Selektiere den Polarisationszustand mit einem
Linearpolarisator und analysiere mit dem Linearpolarisator
%% IMAGE MISSING
Es gilt für den Winkel $\alpha$ zwischen Polarisator und Analysator
\begin{gather*}
  \vecE_0 = \vecE_{0\parallel} + \vecE_{0\bot} 
  = |\vecE_0|\cos\alpha \vece_\parallel
  + |\vecE_0|\sin\alpha \vece_\bot
\end{gather*}
und $|\vecE_\parallel| = |\vecE_0|\cos\alpha$.

\paragraph{Beispiel Polarisation durch Dichroismus}
Dichromatische Materialien sind z.\,B. Turmalin oder (preiswerter)
gestreckte, gefärbte Polymerfolien.
Hier gilt das Malus'sche Gesetz\index{Malus'sches Gesetz}
\begin{gather*}
  I=I_0\cos^2\alpha
\end{gather*}
wobei die Stromstärke proportional zur Intensität ist.


\minisec{Polarisation durch Reflexion}
Beispiele für Polarisation durch Reflexion sind:
\begin{itemize}
\item Für den \emph{Brewster Winkel} gilt, dass das reflektierte Licht senkrecht
zur Einfallsebene polarisiert ist.
\item \emph{Fresnel-Rhombus}
\item \emph{Polarisationsdrehung} durch Reflexion, z.\,B. Doppelspiegel
%% IMAGE MISSING
\end{itemize}

%---------------------
% 16.11.2015
%---------------------
\marginpar{16.11.2015}

\minisec{Polarisation durch Dichroismus}
Bisher haben wir eine kugelsymmetrische Elektronenverteilung um den
Atomkern betrachtet. Jetzt werden wir eine \emph{anisotrope
Ladungsverteilung} betrachten. Der einfachste Fall ist eine
zigarrenförmige Ladungsverteilung.
%% IMAGE MISSING
Wir machen folgende vereinfachende Annahmen:
\begin{itemize}
\item Die Bewegungen senkrecht und parallel zur
  Atom-/Molekülvorzugsachse sind unabhängig voneinander 
\item zwei unabhängige harmonische Oszillatoren mit Eigenfrequenzen
$\omega\bot$%
\nomenclature{$\omega_\bot$}{Eigenfrequenz der harmonischen
  Oszillation der Elektronen senkrecht zur Atom-/Molekülvorzugsachse}
und $\omega\parallel$%
\nomenclature{$\omega_\parallel$}{Eigenfrequenz der harmonischen
  Oszillation der Elektronen parallel zur Atom-/Molekülvorzugsachse}
\end{itemize}
In unserem \enquote{Zigarrenmodell} sind folgende zwei Fälle der
Ausrichtung der asymmetrischen Moleküle/Atome im Material zu
unterscheiden:
\begin{description}
\item[isotrope Verteilung] Die Ausrichtung der asymmetrischen
  Moleküle/Atome ist isotrop in alle Richtungen verteilt.
\item[anisotrope Verteilung] Die asymmetrischen Moleküle
  sind geordnet, d.\,h. gleich ausgerichtet, wie z.\,B. im
  kristallinen Festkörper. Hier können die optischen Eigenschaften
  stark anisotrop (richtungsabhängig) werden.
\end{description}
%% IMAGE MISSING : s. Folien; Absorption, Dispersion 
Eine Folge von anisotroper Verteilung der Ausrichtung ist der
\emph{Dichroismus}\index{Dichroismus} (richtungsabhängige Absorption):
Je nach Richtung der Polarisation ist die Absorption stark
unterschiedlich.
Beispiele für Materialien, die ausgeprägten Dichroismus aufweisen,
sind:
\begin{itemize}
\item Turmalinkristalle
\item Polaroidfolien (Folienpolarisatoren)
\item Drahtgitterpolarisatoren
\end{itemize}

%---------

%5.3
\section{Doppelbrechung} 
Doppelbrechung gibt es in folgenden Formen
\begin{description}
\item[natürlich] anisotrope Kristalle (z.\,B. Calcit)
\item[mechanisch induziert] z.\,B. durch mechanische Spannung
\item[durch Felder induziert] z.\,B. durch elektrische Felder
\end{description}
Der physikalische Hintergrund ist eine anisotrope Ladungsverteilung,
d.\,h. zwei Resonanzfrequenzen $\omega_\bot$ und $\omega_\parallel$.
Dies lässt sich schön anhand des folgenden Modells darstellen:

\paragraph{Federmodell}
Als Modell betrachtet man eine Kugel, die mit zwei unterschiedlichen
Federn vefestigt ist.
Zieht man mit der Kraft $\vecF(x,y)$, so beugt sich die Masse nicht
parallel zur Kraft.
%% IMAGE MISSING

\paragraph{Lichtausbreitung in einem doppelbrechenden Medium}
Die Experimentelle Beobachtung ergibt
\begin{itemize}
\item Es treten Doppelbilder auf.
\item Es tritt eine Aufspaltung in zwei getrennte Strahlen auf; Es
  gibt eine Abhängigkeit von der Polarisation.
\item Die relativen Intensitäten sind abhängig von der Polarisation
  der einlaufenden Welle.
\item Nur der ordentliche Strahl folgt dem Snellius'schen Gesetz.
\item Der ordentliche und der außerordentliche Strahl bilden einen
  Winkel zueinander; Dieser wird
  Strahlenversatz\index{Strahlenversatz} genannt.
\end{itemize}
Für die dielektrischen Eigenschaften bedeutet das, dass
\begin{enumerate}
\item zwischen $\vecD$ und $\vecE$ ein tensorieller Zusammenhang
  besteht:
  \begin{gather*}
    \vecD=\epso\teps\vecE
    \qquad\text{oder}\qquad
    D_i = \epso\sum_{k=1}^i \eps_{ik}E_k
  \end{gather*}%
  \nomenclature{$\teps$}{Dielektrizitätstensor; $\vecD=\epso\teps\vecE$}%
  \index{Dielektrizitätstensor}%
\item $\vecE$ und $\vecP$ sind i.\,A. nicht mehr parallel
\end{enumerate}
Wir erhalten durch orthogonale Transformation der Koordinatenachsen die
sog. \emph{Hauptachsenform des Dielektrizitätstensors}%
\index{Dielektrizitätstensor!Hauptachsenform},
d.\,h. die Diagonalform der Matrixdarstellung des Tensors 
($\eps_i$%
\nomenclature{$\eps_i$}{$i$-tes Tensorelement des
  Dielektrizitätstensors in der Hauptachsenform}
ist der Eintrag auf der Diagonale in der $i$-ten Zeile). Hier gilt
\begin{gather*}
  D_i = \epso\eps_iE_i
  \qquad\text{bzw.}\qquad
  E_i = \frac{1}{\epso\eps_i}
\end{gather*}%
Die spezielle Form von $\teps$ hängt von der Symmetrie des Mediums ab.
Man unterscheidet drei Fälle:
\begin{description}
\item[optisch isotrope Medien:  
  $\eps_i = \eps$] 
  Alle drei Tensorelemente sind
  gleich. Hier sind die optischen Eigenschaften nicht
  richtungsabhängig.
\item[optische einachsige Kristalle: 
  $\eps_x=\eps_y=\eps_\bot$ und $\eps_z=\eps_\parallel$]
  Zwei der drei Tensorelemente sind gleich, d.\,h. es gibt eine ausgezeichnete
  optische Achse (hier als Beispiel die $z$-Achse). Für
  Richtungsausbreitung entlang der optischen Achse tritt keine
  Polarisationsabhängigkeit der Ausbreitung auf (die $\vecE$-Felder
  schwingen entlang $\eps_\bot$). In der anderen Richtung treten
  Abweichungen vom Snellius'schen Gesetz auf.
  Dieser Effekt wird \emph{Doppelbrechung}\index{Doppelbrechung}
  genannt.
\item[optisch zweiachsige Kristalle:
  $\eps_x\neq\eps_y\neq\eps_z\neq\eps_x$]
  Es gibt zwei optische Achsen, längs derer polarisationsunabhängige
  Lichtausbreitung stattfinden kann.
\end{description}
Für die Herleitung verwende die MWGl in Tensorform mit dem Ansatz
ebene Welle und Grenzbedingungen.
Dann erhält man die Beziehungen für die Richtungen von $\vecE$,
$\vecB$, $\vecD$, $\vecS$ und $\veck$.
Das Resultat ist:
Die Dielektrische Verschiebung ist i.\,A. nicht parallel zu $\vecE$.

\minisec{Resultat für den Brechungsindex (einachsige Kristalle)}
Es sei wie oben die Koordinatendarstellung so gewählt, dass der
Dielektrizitätstensor Hauptachsenform hat, und die optische Achse sei
wieder die $z$-Achse. Wir erhalten die Beziehung
\begin{gather*}
  \frac{1}{n_{ao}^2} 
  = \frac{\cos^2\theta}{\eps_\bot} + \frac{\sin^2\theta}{\eps_\parallel}
  \qquad\text{und} \qquad
  n_o = \sqrt{\eps_\bot}
\end{gather*}
Der Brechungsindex ist also richtungsabhängig mit
$\eps_x=\eps_y=\eps_\bot$ und $\eps_z=\eps_\parallel$.
$\theta$ ist der Winkel, den der Wellenvektor $\veck$ mit der optischen
Achse ($z$-Achse) einschließt.
Wir betrachten folgende Fälle:
\begin{description}
\item[$\theta=\ang{0}$] Es gibt keine Doppelbrechung für Licht, das
  sich entlang der optischen Achse ausbreitet ($n_{ao} = n_o$).
\item[$\theta=\ang{90}$] Wenn der Wellenvektor in $x$-Richtung zeigt,
  ergeben sich zwei Werte für den Brechungsindex: 
  $n_o=\sqrt{\eps_\bot}$ und $n_{ao}=\sqrt{\eps_\parallel}$.
  Dabei liegen die $\vecE$-Vektoren entlang der $z$-Richtung (optische
  Achse) bzw. längs der $y$-Richtung. Licht kann sich in $x$-Richtung
  nur mit diesen beiden Polarisationsrichtungen im Kristall ausbreiten!
\end{description}
Im optisch einachsigen Kristall gibt es zwei Polarisationsrichtungen
\begin{description}
\item[ordentlischer Strahl] Hier haben wir die Eigenschaften
  \begin{itemize}
  \item $\vecE$ und $\vecD$ sind senkrecht zur optischen Achse
    gerichtet.
  \item Der Brechungsindex ist unabhängig von der Ausbreitungsrichtung
    $n_o=\sqrt{\eps_\bot}$.
  \item Das Snellius'sche Gesetz gilt.
  \end{itemize}
\item[außerordentlicher Strahl] Hier gilt:
  \begin{itemize}
  \item Der Strahl ist in der Ebene, die durch die optische Achse und
    $\veck$ gebildet wird, polarisiert
    (Hauptschnitt\index{Hauptschnitt}).
    \item Der Brechungsindex ist richtungsabhängig.
    \item Das Snellius'sche Gesetz gilt nicht.
  \end{itemize}
\end{description}

\minisec{Phasengeschwindigkeiten}
Für den ordentlichen Strahl erhalten wir eine Kugelgleichung für $n$
bzw. $v_{ph}$ und
für den außerordentlichen Strahl wird es zu einem Ellipsoid:
\begin{align*}
  v_{ao} &= \frac{c}{\sqrt{\eps_\parallel}} = \frac{c}{n_{ao}}\\
  v_o &= \frac{c}{\sqrt{\eps_\bot}} = \frac{c}{n_{o}}\\
\end{align*}
Es gilt:
\begin{description}
\item[$v_{ao}>v_o$ ($n_{ao}<n_o$):] negativ einachsig
\item[$v_{ao}<v_o$ ($n_{ao}>n_o$):] negativ einachsig
\end{description}


%%% Local Variables:
%%% mode: latex
%%% TeX-master: "../OptikSkriptWS1516"
%%% End:
