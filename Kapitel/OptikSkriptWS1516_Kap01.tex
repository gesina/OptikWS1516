\chapter{Einführung}
\section{Historischer Überblick}
siehe Folien\\
Versuche: Messung der Lichtgeschwindigkeit

\section{Hierachie der Berschreibung optischer Phänomene}
\begin{itemize}
	\item geometrische Optik
	\item Wellenoptik
	\item Elektromagnetismus
	\item Quantenoptik
\end{itemize}

\section{Licht als elektromagnetische Welle}
Eine wichtige Frage vorab ist:
Was ist Licht? Teilchen oder Welle?

\paragraph{pro elektromagnetische Welle}
\begin{itemize}
	\item Licht transportiert Energie, auch im Vakuum
	\item Licht wechselwirkt mit Atomen/Materie (z.B. 		Absorption)
	\item Licht zeigt Brechungserscheinungen
\end{itemize}
Daraus folgt: \red{Licht ist elektromagnetische Welle}

\paragraph{pro Korpuskel}
\begin{itemize}
	\item Licht zeigt \enquote{Körnigkeit}, 
      es besteht aus Energiequanten (Photonen) mit $E=h\nu$
	\item Licht stößt wie ein Teilchen (Compton-Effekt)
\end{itemize}

\section{Das elektromagnetische Spektrum}
Die Vorlesung \enquote{Wellen und Quanten} beschäftigt sich mit den
Eigenschaften elektromagnetischer (Hertzscher) Wellen über einem breiten
Wellenlängenbereich von
$\SI{e-15}{\m} \leq \lambda\leq \SI{e3}{\m}$.
Zum Vergleich: Sichtbares Licht hat Wellenlängen im Bereich
$\SI{350}{\nano\m} \leq \lambda \leq \SI{800}{\nano\m}$.

%%% Local Variables:
%%% mode: latex
%%% TeX-master: "../OptikSkriptWS1516"
%%% End:
