%%-----------------------------------------

%% Vorlesungsmitschrift (Kapitel 4)

%% an der Uni Regensburg, gelesen von Christian Back

%%-----------------------------------------


\chapter{Elektromagnetische Wellen an Grenzflächen}
\section{Randbedingungen der elektromagnetischen Welle}
Wir wollen jetzt Wellenausbreitung in inhomogenen Medien beschreiben,
z.\,B. den Übergang von Medium~1 nach Medium~2, also einer
Grenzfläche.

\paragraph{Randbedingungen der MWGl.:} Die Tangentialkomponenten von
$\vecE$ und $\vecH=\fracone{\muo\mu_r}\vecB$ sind stetig. Die
Normalkomponenten von $\vecD=\epso\eps_r\vecE$ und $\vecB$ sind
ebenfalls stetig (Erinnerung: Für isotrope, isolierende, nicht
magnetische Medien gilt $\mu_r\eqqcolon\mu=1$).
\paragraph{einfachster Fall:} Wir betrachten zwei homogene Medien mit
Brechungsindex $n_e$ (einfallender Strahl) und $n_t$ (transmittierte
Welle).
%
%% Skizze zu Reflexion
%
Der Winkel $\alpha$ liegt zwischen $\veck_e$ und $\vece_y$ bzw. zwischen $\veck_t$ und $\vece_y$. Wir nehmen an, dass es eine fest vorgegebene einlaufende Welle ist.
\begin{align*}
  \vecE_e &= \vecE_{0,e}\cos(\omega_et-\veck_e\vecr)
            =\vecE_{0,e}(\phi_e(\vecr,t))\\
  \vecE_r &= \vecE_{0,r}\cos(\omega_rt - \veck_r\vecr+\varphi_r)
            =\vecE_{0,r}(\phi_r(\vecr,t))\\
  \vecE_t &= \vecE_{0,t}\cos(\omega_tt - \veck_t\vecr + \varphi_t)
            =\vecE_{0,t}(\phi_t(\vecr,t))
\end{align*}
Die Wellenvektoren $\veck_e$, $\veck_r$ und $\veck_t$ müssen die%
\nomenclature{$\veck_e$}{Wellenvektor der einlaufenden Welle}%
\nomenclature{$\veck_t$}{Wellenvektor der transmittierten Welle}%
\nomenclature{$\veck_r$}{Wellenvektor der reflektierten Welle}
Dispersionsrelationen im jeweiligen Medium erfüllen. Die
\emph{Phasenfaktoren}\index{Phasenfaktoren}
$\varphi_r$, $\varphi_t$%
\nomenclature{$\varphi_r$}{Phasenfaktor der reflektierten Welle;
  bestimmt Phasenverschiebung zur einlaufenden Welle}%
\nomenclature{$\varphi_t$}{Phasenfaktor der transmittierten Welle;
  bestimmt Phasenverschiebung zur einlaufenden Welle} 
bestimmen die Phasenlage relativ zur einlaufenden Welle.
