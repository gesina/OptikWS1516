%%-----------------------------------------
%%
%% Vorlesungsmitschrift (Kapitel 3)
%% Optik im WS2015/16
%% an der Uni Regensburg, gelesen von Christian Back
%%
%%-----------------------------------------

\chapter{Dispersion von Licht}
Die Ausbreitung von Licht hängt vom Brechungsindex $n=n(\omega)$
ab. $n(\omega)$ bestimmt die Geschwindigkeit von Licht, das
Auseinanderfließen von Lichtimpulsen, Ablenkungen und Reflexion an
Grenzflächen. 

\section{Die Frequenzabhängigkeit der Dielekrizitätskonstante $\eps$}
I.A. ist $\eps$ ein Tensor mit $\omega$-Abhängigkeit. Vergleicht man
den statischen Dielektrizitätskonstante bzw. den Brechungsindex
$n_0=\sqrt{\eps(\omega=0)}$ mit $
n=\sqrt{\eps(\omega=\SI{589}{\nano\meter})}$, ist ein Unterschied zu erkennen!

\section{Der Brechungsindex}
Am einfachsten ist die Betrachtung eines verdünnten Mediums,
d.\,h. man ist weit weg von Resonanzerscheinungen (s. Folien).
Dazu machen wir folgende Annahmen:
\begin{align*}
  \eps(\omega) &\approx 1\\
  |\Delta\eps| &\ll 1\\
\Longrightarrow n(\omega) &= \sqrt{\eps(\omega)} \approx 1
\end{align*}
Mit dieser Schätzung erhalten wir
\begin{align*}
  \eps - 1 &= n^2-1 = \underbrace{(n+1)}_{\approx 2}(n-1) \approx 2(n-1)\\
  n-1 &\approx \frac{1}{2} (\eps -1) 
  = \frac{e^2N}{\epso m\cdot 2}\cdot
  \frac{1}{\omegao^2-\omega^2 + i\gamma\omega}
\end{align*}
und es folgt
\begin{align*}
  n_R &= 1 + \frac{e^2N}{2m\epso} 
        \frac{\omegao^2-\omega^2}{(\omegao^2-\omega^2)^2 + \gamma^2\omega^2}
  && \text{Realteil}\\
  n_I &= \frac{e^2N}{2m\epso}
        \frac{-\gamma\omega}{(\omegao^2-\omega^2)^2 + \gamma^2\omega^2}
  && \text{Imaginärteil}\\
\end{align*}
Oft schreibt man $n'$ für den Realteil, genannt 
\emph{reelle Brechzahl}\index{Brechzahl}, und 
$k$ für den Imaginärteil, genannt
\emph{Absorptionsindex}\index{Absoprtionsindex}:
\begin{gather*}
  n(\omega) = n'(\omega) - ik(\omega)
\end{gather*}%
\nomenclature{$n'$}{reelle Brechzahl eines Mediums;
  Realteil $n_R$ des Brechungsindex $n$}%
\nomenclature{$k$}{Absorptionsindex eines Mediums;
  Imaginärteil $n_I$ des Brechungsindex $n$}%


\section{Absorption von Licht}
Wir nehmen eine \emph{ebene Welle $E(z,t)$} an, welche in $z$-Richtung
mit Wellenvektor $k=\frac{n\omega}{c}$ propagiert, in einem Medium mit
$n(\omega)=n_R(\omega)+in_I(\omega)$.
Dann ergibt sich
\begin{align*}
  E(z,t) &= E_0\exp(i\omega t-ikz) \\
         &= E_0\exp\left[
           i\omega t - i\frac{\omega n_R}{c}\cdot z 
           + \frac{\omega n_I}{c}\cdot z
           \right]\\
         &= \left( 
           E_0\exp\left(\frac{\omega n_I}{c}\cdot z\right)
           \right)
%           \underbrace{
           \exp\left( i\omega t-\frac{i\omega n_R}{c}\cdot z\right)
%           }_\substack{\text{\enquote{normale} ebene Welle \\
%  Wellenlänge wird durch $n_R$ bestimmt}}
\end{align*}
Die Amplitude der Welle wird exponentiell gedämpft, falls $n_I<0$, was
fast immer der Fall ist (aber z.\,B. nicht bei Lasermedien). Daraus
folgt für die Intensität der ebenen Welle
\begin{gather*}
  I(z) = I(0)\exp\left(\frac{2\omega n_I z}{c}\right)
  = I(0) \exp(-az)
\end{gather*}
wobei $a=\frac{2\omega n_I}{c}$
\emph{Extinktionskoeffizient}%
\index{Extinktionskoeffizient}%
\nomenclature{$a$}{Extinktionskoeffizient eines Mediums}
genannt wird. Es ist
\begin{gather*}
  a = \frac{e^2N}{\epso mc}        
  \frac{\gamma\omegao^2}{(\omegao^2-\omega^2)^2 + \gamma^2\omega^2}
\end{gather*}
Das Verhältnis von Intensität zu Nullintensität wird 
Transmission $T$%
\index{Transmission}%
\index{Lambert-Beersches Gesetz}%
\nomenclature{$T$}{Transmission eines Mediums bei ebener Welle;
$T \coloneqq \frac{I(z)}{I(0)} = \exp(-az)$} 
genannt
\begin{align*}
  T &\coloneqq \frac{I(z)}{I(0)} = \exp(-az)
  &&\text{Lambert-Beersches Gesetz}
\end{align*}
\red{Achtung:} Imaginärteil und Realteil des Brechungsindex sind nicht
unabhängig voneinander! Die Relation zwischen ihnen wird
\emph{Kramer-Kronis-Relation}\index{Kramer-Kronis-Relation}
genannt:
\begin{gather*}
  \left(\Real(n(\omega))\right)^2 - 1
  = \frac{2c}{\pi} \int_0^\infty
  \frac{a(\omega')}{\omega'^2-\omega^2}\dd \omega'
\end{gather*}
Diese Relation ist wichtig für die optische Spektroskopie.


\section[Detailbetrachtung]{Wie sieht das im Detail aus?}
Die zu erzwungenen Schwingungen angeregten Atome (Elektronenwolken) im
Medium strahlen mit der Anregungsfrequenz, aber mit \emph{verzögerter
  Phase} ab (s. Mechanik)

\section{Das elektromagnetische Feld eines oszillierenden Dipols}
%% Image Missing!!
Die Licht-Materie-Wechselwirkung wird in erster Linie durch die
Wechselwirkungen elektromagnetischer Wellen mit atomaren Dipolen
beschrieben. Es wird hierzu das elektrische und magnetische Feld eines
strahlenden Dipols berechnet.
Wir werden im Folgenden das $\vecB$-Feld betrachten (das $\vecE$-Feld
wird von den Maxwell-Gleichungen beschrieben).

Es gilt für das Vektorpotential $\vecB=\vecna\times\vecA$ wobei
\begin{gather*}
  \vecA (\vech) = \frac{\muo}{4\pi} \int 
  \frac{\vecj(\vecr_2)\dd V_2}{|\vecr_{12}|}
\end{gather*}
\red{Achtung:} für $\vecj=\vecj(\vecr_2,t)$ ergibt sich eine
Retardierung, da sich das Feld nur mit Lichtgeschwindigkeit
ausbreitet.
Damit hängt $\vecA(\vecr_1, t)$ von $\vecj(\vecr_2)$ zur Zeit 
$t-\frac{|\vecr_{12}|}{c}$ ab:
\begin{gather*}
  \vecA(\vecr_1, t) = \frac{\muo}{4\pi} \int
  \frac
  {\vecj\left( \vecr_2, t-\frac{|\vecr_{12}|}{c} \right)}
  {|\vecr_{12}|}
  \dd V_2
\end{gather*}
Im Fernfeld (d.\,h. $|\vecr_{12}|\gg l$) gilt mit $\vecj = \vecv\rho$
\begin{gather*}
  \vecA(\vecr_1, t) = \frac{\muo}{4\pi r_{12}} \int
  \vecv \rho\left( \vecr_2, t-\frac{|\vecr_{12}|}{c} \right)
  \dd V_2
\end{gather*}
wobei $\rho \dd V_2 = \dd q$ und $\int\rho\dd V = q$ für die Ladung $q$
gilt.

Das elektrische Dipolmoment $\vecp$ wird beschrieben durch
\begin{gather*}
  \vecp(t) = q\underbrace{d_0\sin(\omega t)}_{\vecd(t)}\hat{e}_z
  = q\vecd(t)
\end{gather*}
mit $d_0$ als maximale Auslenkung der Ladung nach einer Viertel
Periode.

Jetzt verwenden wir $\vecp(t) = \vecp_0 \sin(\omega t)$ und
$\dif[\vecp]{t} = \dot{\vecp} = q\vecv$.
Damit erhält man
\begin{gather*}
  \vecA(t)(\vecr_1, t) = \frac{\muo}{4\pi\vecr_{12}}
  \dif[\vecp(t-\frac{r_{12}}{c})]{t}
\end{gather*}
mit $\omega(t-\frac{r_{12}}{c}) = \omega t- r_{12}\frac{2\pi}{\lambda}
= \omega t - kr_{12}$ ($k$ als Wellenvektor) sowie
$\dif[\vecp]{t} = q\cdot d_0\omega\cos(\omega t) \hat e_z$.
Insgesamt ergibt sich ein zeitlich veränderliches Vektorpotential, das
sich mit Lichtgeschwindigkeit ausbreitet:
\begin{gather*}
  \vecA(\vecr_1,t) = \frac{\muo}{4\pi}
  \underbrace{
    \frac{1}{\vecr_{12}} q\cdot d_0\omega\cos(\omega t-kr_{12})
    \vece_z
  }_{\text{Kugelwelle mit }c=\frac{\omega}{k}}
\end{gather*}
Zusammengefasst erzeugt eine oszillierende Ladung ein Vektorfeld
$\vecA$, welches wiederum $\vecB$ und $\vecE$ erzeugt.

Es bleibt die Frage, wie sieht das $\vecB$-Feld am Ort $P$ aus?
Wir haben
\begin{align*}
  \vecB &= \vecna\times\vecA 
  &\text{mit } \vecA = \left( 0,0,\vecA_z \right)\\
  B_x &= \dif[A_z]{y} \\
  B_y &= -\dif[A_z]{x}\\
  B_z &= 0
\end{align*}
Das heißt, das $\vecB$-Feld liegt in der $x$-$y$-Ebene.
Die genaue Formel für $B_x$ ist
\begin{gather*}
  B_x = \frac{\muo}{4\pi}\left[
    \dot{p}\left(t-\frac{r_{12}}{c}\right)\dif{y}\cdot\frac{1}{r}
      + \frac{1}{r}\left(\dot{p}\left(t-\frac{r_{12}}{c}\right)\right)
    \right]
\end{gather*}
Definiere zur Vereinfachung $u=t-\frac{r_{12}}{c}$, dann ist
$\dot{p} = \dif[p]{u}\cdot\dif[u]{t} = \dif[p]{u}$
und $\dif[u]{r} = -\frac{1}{c}$.
Außerdem gilt $\dif[r]{y}=\frac{y}{r}$, d.h.
\begin{gather*}
  \dif[\dot{p}]{y} = \dif[\dot{p}]{u}\cdot\dif[u]{r}\cdot\dif[r]{y}
  = -\ddot{p}\frac{1}{c}\frac{y}{r}
\end{gather*}
Damit gilt dann
\begin{gather*}
  B_x = -\frac{1}{4\pi\epso c^2}
  \left[ \dot{p}\frac{y}{r^3} 
    + \ddot{p} \frac{y}{cr^2} \right]
\end{gather*}
$B_y$ analog. Insgesamt
\begin{gather*}
  \vecB(\vecr) = \frac{1}{4\pi\epso c^2}\cdot
  \frac{1}{r^3}
  \left[ 
    \dot{\vecp}\times\vecr 
    + \frac{r}{c}\ddot{\vecp}\times\vecr
  \right]
\end{gather*}
Wegen $\vecp\parallel\dot{\vecp}\parallel\ddot{\vecp}$
gilt $\vecB \bot \vecp_{\text{Dipol}}$ und
$\vecB \bot$Ausbreitungsrichtung.
Damit folgen zwei Terme für $\vecB$:
\begin{enumerate}
\item Nahfeld $\propto \dot\vecp \propto \frac{1}{r^2}$
\item Fernfeld $\propto \ddot\vecp \propto \frac{1}{r}$
\end{enumerate}
Daher kommen die Terme
\begin{gather*}
  \vecna \times \vecB 
  = \underbrace{\mu_0\vecj}_{\text{Nahfeld}}
  + \underbrace{\muo\epso \dif[\vecE]{t}}_{\text{Fernfeld}}
\end{gather*}
wobei das $\vecE$-Feld über die Maxwell-Gleichungen beschrieben wird.

%-------------------------
% bis 23.10.
% Hier ein Kommentar

%%% Local Variables:
%%% mode: latex
%%% TeX-master: "../OptikSkriptWS1516"
%%% End:
