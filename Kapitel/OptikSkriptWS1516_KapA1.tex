%%----------------------------------------------
%%
%%            Mitschrift der Vorlesung
%%   Experimentalphysik III: Wellen und Quanten 
%%       im WS2015/16 an der Uni Regensburg
%%
%% Anhang 1: Vektorableitungen
%%
%%
%% gesetzt von Gesina und Hedwig
%%
%%----------------------------------------------

\chapter{Vektorableitungen}\label{vektorableitungen}
Im Folgenden wird ein Überblick über die häufig gebrauchten Operatoren 
Gradient ($\Grad$), Rotation ($\Rot$), Divergenz ($\Div$), den Nabla-Operator
($\vecna$) und den Laplace-Operator ($\Delta$) gegeben.

\paragraph*{Nabla-Operator}
Für 3-dimensionale Vektoren verwendet man den 
\emph{Nabla}-Operator%
\index{Operatoren!Nabla-Operator}%
\index{Nabla-Operator}
\begin{gather*}
  \vecna = 
  \begin{pmatrix}
    \dif[]{x}, \dif[]{y}, \dif[]{z}
  \end{pmatrix}
\end{gather*}

\paragraph*{Gradient}
Der \emph{Gradient}%
\index{Operatoren!Gradient}%
\index{Gradient}
einer skalaren Funktion $f=f(x,y,z)$ zeigt in die Richtung des größten Anstiegs.
\begin{gather*}
  \Grad f = \vecna f = 
  \begin{pmatrix}
    \dif[f]{x},\dif[f]{y},\dif[f]{z}
  \end{pmatrix}
\end{gather*}

\paragraph*{Divergenz}
Die \emph{Divergenz}%
\index{Operatoren!Divergenz}%
\index{Divergenz}
einer Vektorfunktion ist
\begin{gather*}
  \Div\vec{f} = \vecna\vec{f}
  = \dif[f_{x}]{x}+\dif[f_{y}]{y}+\dif[f_{z}]{z}
\end{gather*}
Die Divergenz ist ungleich null, wenn es Quellen oder Senken gibt
(vgl. elektr. Ladung).

\paragraph*{Laplace-Operator}
Der \emph{Laplace-Operator}%
\index{Operatoren!Laplace-Operator}%
\index{Laplace-Operator}
einer skalaren Funktion ist die \emph{Divergenz des Gradienten}.
\begin{gather*}
  \Delta f = \vecna^2 f= \vecna\vecna f = \vecna 
  \begin{pmatrix}
	\dif[f]{x},\dif[f]{y},\dif[f]{z}
  \end{pmatrix}
  = \diff[f]{x} + \diff[f]{y} + \diff[f]{z}
\end{gather*}
Der Laplace einer Vektorfunktion wir komponentenweise gebildet.
\begin{gather*}
  \Delta \vec{f} = \vecna^2 \vec{f}= 
  \begin{pmatrix}
    \diff[f_x]{x}+\diff[f_x]{y}+\diff[f_x]{z}+\dots
  \end{pmatrix}
\end{gather*}
Die \emph{Rotation} einer Vektorfunktion $\vec{f}$ ist
\begin{gather*}
  \Rot \vec{f}= \vecna\times \vec{f}=
  \begin{pmatrix}
    \dif[f_z]{y}-\dif[f_y]{z},\dif[f_x]{z}-\dif[f_z]{x},\dif[f_y]{x}-\dif[f_x]{y}
  \end{pmatrix}
\end{gather*}
Funktionen, die sich stark \enquote{winden}, haben eine starke Rotation.




%%% Local Variables:
%%% mode: latex
%%% TeX-master: "../OptikSkriptWS1516"
%%% End:
