%% ----------------------------------------------
%% 
%% Mitschrift der Vorlesung
%% Experimentalphysik III: Wellen und Quanten 
%% im WS2015/16 an der Uni Regensburg
%% 
%% Kapitel 2: Elektromagnetische Wellen
%% 
%% 
%% gesetzt von Gesina und Hedwig
%% 
%% ----------------------------------------------

\chapter{Elektromagnetische Wellen}
\section{Licht als elektromagnetische Welle}
Im Folgenden werden Kenntnisse in \emph{Vektorableitung} benötigt,
siehe dazu \nameref{vektorableitungen} im Anhang,
S.\,\pageref{vektorableitungen}.

In dieser Vorlesung behandeln wir Lichtausbreitung in
nicht-magnetischen Medien, d.\,h. man kann die 
magnetische Permeabilität $\mu =1$%
\nomenclature{$\mu$}{magnetische Permeabilität eines Mediums; hier immer $\mu=1$}
setzen. Für nicht leitende Materialien ist zudem die
Ladungsdichte $\rho_\text{frei}$%
\nomenclature{$\rho_\text{frei}$}{Ladungsdichte; in isolierenden Materialien $\rho_\text{frei}=0$}
und die Stromdichte $j_\text{frei}$%
\nomenclature{$j_\text{frei}$}{Stromdichte; in isolierenden Materialien $j_\text{frei}=0$}
gleich null. 
In Formeln also $\mu=1$, $\rho_\text{frei}=0$, $j_\text{frei}=0$. 

\minisec{Lichtausbreitung im Vakuum oder im Dielektrikum}
Im Dielektrikum muss man die \emph{Maxwellgleichungen} (kurz
MWGl.) \emph{für ein Medium mit dielektrischer Verschiebung}%
\index{Maxwellgleichungen!mit dielektrischer Versch.}
verwenden.
Weitere Annahmen sind
\begin{itemize}
\item lineare Optik
\item isotropes Medium (Gase, Flüssigkeit, kubische Kristalle)
\end{itemize}
Mit diesen Annahmen gilt
\begin{gather*}
  \vec{D}=\eps_0\eps\vec{E}
\end{gather*}%
\nomenclature{$\vec{D}$}{dielektrische Verschiebung}%
\nomenclature{$\vec{E}$}{elektrische Feldstärke}%
\nomenclature{$\eps_0$}{elektrische Feldkonstante;
  $\eps_0=\SI{8.854e-12}{\coulomb^2\m^{-2}\newton^{-1}}$}%
\nomenclature{$\eps$}{relative Dielektrizitätskonstante eines Mediums}%
wobei $\eps_0=\SI{8.854e-12}{\coulomb^2\m^{-2}\newton^{-1}}$
die elektrische Feldkonstante\index{elektrische Feldkonstante} ist und
\emph{$\eps$} die relative \emph{Dielektrizitätskonstante}%
\index{Dielektrizitätskonstante} des Mediums.

\red{Achtung:} in optisch anisotropen Medien wird
$\eps$ durch einen Tensor ersetzt.

% ---

\subsection{Maxwell-Gleichungen}
Elektrische und magnetische Felder sind wie folgt über die MWGl. verknüpft:
\minisec{MWGl. für isolierendes nicht magnetisches Medium:}
\begin{align*}	 
  \vecna\vec{D}&=0\\
  \vecna\vec{B}&=0\\
  \vecna\times\vec{E}&=-\dif[\vec{B}]{t}\\
  \vecna\times\vecB&=\mu_0 \dif[\vec{D}]{t}
\end{align*}%
\nomenclature{$\vecB$}{magnetische Flussdichte}%
mit magnetischer Feldkonstante 
$\mu_0=\SI{1.2566e-6}{\newton\ampere^{-2}}$%
\nomenclature{$\mu_0$}{magnetische Feldkonstante; $\mu_0=\SI{1.2566e-6}{\newton\ampere^{-2}}$}

\minisec{MWGl. im Vakuum}
\begin{align*}
  \vecna \vecE &=\frac{\rho}{\eps_0}\\
  \vecna\vecB&=0\\
  \vecna\times\vecE&=-\dif[\vecB]{t}\\
  \vecna\times\vecB&=\mu_o(\vec{j}+\eps_0\frac{\partial\vecE}{\partial t})
\end{align*}
mit $\eps_0,\mu_0$ \enquote{Materialparameter} für Vakuum.

\minisec{Effekte in Materie} 
In echter Materie können (mikroskopisch)
Polarisationsladungen und Ampere'sche Kreisströme induziert werden, was
\begin{itemize}
\item mikroskopische elektrische Dipole
\item mikroskopische Kreisströme
\end{itemize}
verursacht. Zur Vereinfachung betrachten wir makroskopische, örtlich
gemittelte Größen.

\minisec{Unterschiede zwischen freien und gebundenen Ladungen}
Es gilt allgemein für die Ladungsdichte $\rho$%
\nomenclature{$\rho$}{Ladungsdichte (Ladung pro Volumen)} 
und die
Stromdichte $\vecj$%
\nomenclature{$\vecj$}{Stromdichte (Ladung pro Zeit pro Fläche)} 
\begin{align*}
  \rho&=\rho_\text{frei}+\rho_\text{gebunden}\\
  \text{analog}\quad
  \vecj&=\vecj_\text{frei} + \vecj_\text{mag} + \vecj_\text{Polarisation}
\end{align*}
wobei   
$\vecj_\text{mag} = \vecna\times\vec{M}$ und 
$\vecj_\text{Polarisation}=\dif[\vec{P}]{t}$.
Die gebundenen Ladungsträger führen zu einer makroskopischen
Polarisation $\vec{P}$%
\nomenclature{$\vecP$}{Polarisation (Dipolmoment pro Volumen)}
(bzw. zur makroskopischen Magnetisierung $\vec{M}$%
\nomenclature{$\vecM$}{Magnetisierung (magnetisches Dipolmoment pro Volumen)}
im Fall der Stromdichte), welche sich auf die elektrische
Verschiebung $\vecD$ und die magnetische Feldstärke $\vecB$ auswirkt:
\begin{align*}
  \vecD&=\epso\vecE+\vecP\\
  \vecH&=\frac{1}{\muo}\vecB-\vecM
\end{align*}%
\nomenclature{$\vecH$}{Magnetische Feldintensität; unabhängig davon,
  ob Materie im Magnetfeld ist}
Für isotrope, lineare Materialien mit Dielektrizitätskonstante
$\eps_r$ gilt
\begin{align*}
  \vecP&= \chi \epso\vecE\\
  \vecD&= (1+\chi)\epso\vecE=\eps_r\epso\vecE\\
  \rho_\text{frei}
       &= \rho-\rho_\text{Pol}
         = \rho+\Div\vecP\\
  \Longrightarrow \quad
  \Div\vecP&= -\rho + \rho_\text{frei} = -\rho_\text{Pol}
\end{align*}%
\nomenclature{$\chi$}{elektrische Suszeptibilität}%
Damit ergeben sich die

\minisec{MWGl. in Materie mit Spezialfall (isotropes, ungeladenes, unmagnetisches Medium)}
\begin{align*}
  \vecna\vecD&=\epso\vecna\vecE+\vecna\vecP=\rho_\text{frei}=0\\
  \vecna\vecB&=0\\
  \vecna\times\vecE&=-\dif[\vecB]{t}\\
  \vecna\times\vecH&=\vecj_\text{frei}+\dif[\vecD]{t}
                     =\dif[\vecD]{t}=\frac{1}{\muo}(\vecna\times\vecB)
\end{align*}

% ---

\subsection[Wellengleichung]{Von den MWGl. zur Wellengleichung für das $\vecE$-Feld}
Wir erhalten folgende Zusammenhänge
\begin{align*}
  \vecna \times \vecE 
  &=-\dif[\vecB]{t}
  \\
  \vecna \times (\vecna\times\vecE)
  &=-\vecna\times\dif[\vecB]{t}
    =-\eps\muo\epso\diff[E]{t}
  \\
  -\vecna \times \dif[\vecB]{t}
  &= -\dif[]{t}(\vecna\times\vecB)
    = -\dif[]{t}\muo\epso\eps\dif[E]{t}
  \\
  \vecna\times(\vecna\times\vecE)
  &=\vecna(\underbrace{\vecna\vecE}_{\mathclap{=0 \text{ da $\rho=0$}}})
    -(\vecna\vecna)\vecE
    =-\Delta\vecE
\end{align*}
Setzen wir diese zusammen, folgen die
\emph{Wellengleichungen für elektromagnetische Wellen}%
\index{Wellengleichung!Elektromagnetische Wellen}:
\begin{align*}
  \Delta\vecE-\eps\epso\muo\diff[\vecE]{t} &= 0\\
  \Delta\vecB-\eps\epso\muo\diff[\vecB]{t} &= 0
\end{align*}

\minisec{Allgemeine Form} 
Die allgemeine Form der Wellengleichungen (u.\,a. für elektromagnetische
Wellen) sind Differentialgleichungen, die eine 2.~Ableitung einer Größe nach der Zeit mit der
2.~Ableitung der Größe nach dem Ort verknüpft: 
\begin{align*}
  \diff[y]{t}&=\frac{\tau}{\rho}\diff[y]{x}
\end{align*}
wobei $v_{ph}=\sqrt{\frac{\tau}{\rho}}$ die Ausbreitungsgeschwindigkeit
(\emph{Phasengeschwindigkeit})%
\nomenclature{$v_{ph}$}{Phasengeschwindigkeit;
  Ausbreitungsgeschw. einer Welle; 	
  $\diff[y]{t}=v_{ph}^2\diff[y]{x}$}%
\index{Phasengeschwindigkeit}%
ist. 
Die Berechnungen oben liefern für ein Elektrische Feld,
das sich in einem isolierenden, nicht magnetischen Material
ausbreitet, die Wellengleichung
\begin{gather}
  \diff[\vecE]{t} = \frac{1}{\varepsilon\varepsilon_0\mu_0} 
  \Delta \vecE
  \label{elektrWG}
\end{gather}
Die Ausbreitungsgeschwindigkeit des elektrischen Feldes in einem
solchen Medium ist also
\begin{gather*}
  v_{ph} = \frac{1}{\sqrt{\eps\epso\muo}}
  =\frac{1}{\sqrt{\eps}}\cdot c
\end{gather*}
wobei $c$ die Lichtgeschwindigkeit im Vakuum%
\nomenclature{$c$}{Lichtgeschwindigkeit im Vakuum;
  $	c=\frac{1}{\sqrt{\epso\muo}}=\SI{2.9979e8}{\m\per\s}$} ist mit
$c=\frac{1}{\sqrt{\epso\muo}}=\SI{2.9979e8}{\m\per\s}$.
\red{Achtung:} Nur im Vakuum (hier ist $\eps = 1$) gilt
$c=\frac{1}{\sqrt{\epso\muo}}=v_{ph}$!

In anderen Medien ist der Einfluss des Mediums durch
$\frac{1}{\sqrt{\eps}}=\frac{1}{n}$ gegeben. 
Der \emph{Brechungsindex} 
\begin{gather*}
  n=\sqrt{\eps}
\end{gather*}%
\nomenclature{$n$}{materialspezifischer Brechungsindex;
  $n=\sqrt{\eps}$}%
ist direkt mit der Wellenausbreitung verknüpft.

\section{Bestimmung der Lichtgeschwindigkeit}
\sFolien
\begin{itemize}
\item Planetenmethode
\item Zahnradmethode
\item Drehspiegel
\end{itemize}

\section[Lösung der Wellengleichung]{Lösung der Wellengleichung des elektrischen Feldes im Spezialfall}
Einfachste Lösung der Wellengleichung \eqref{elektrWG} von oben
(Ausbreitung eines Elektrischen Feldes in einem isolierenden, nicht
magnetischen Material) ist die 
\emph{ebene Welle}%
\index{ebene Welle}%
\index{Wellenfunktion!ebene Welle}
\begin{align*}
  \vecE(\vecr,t)&= \vecE_0 \cos(\omega t- \veck\vecr+\phi)\\
  \text{bzw.}\quad\vecE(\vecr,t)&= \Real[\vecE_0e^{i(\omega t-\veck\vecr)+\phi}]
\end{align*}
Ebenfalls ist die \emph{Kugelwelle}%
\index{Kugelwelle}\index{Wellenfunktion!Kugelwelle}
eine Lösung. Der Phasenterm $\phi$ legt den Nulldurchgang des
Kosinus/Sinus fest. Die Lösung eingesetzt in die Wellengleichung führt
zur \emph{linearen Dispersionsrelation}\index{Dispersionsrelation!linear}:
\begin{align*}
  \veck^2=k_x^2+k_y^2+k_z^2=n^2\frac{\omega^2}{c^2}
\end{align*}
Allgemein nennt man eine Beziehung, die den Betrag des Wellenvektors $\veck$%
\nomenclature{$\veck$}{Wellenvektor; 
  senkrecht zur Wellenfront;
  $|\veck|=2\pi\lambda$}%
mit der Kreisfrequenz
verknüpft, \emph{Dispersionsrelation}\index{Dispersionsrelation} 
(z.\,B. bei Photonen $\omega\propto k$, bei freien $e^-$ ist
$\omega\propto k^2$).
Es gelten die Beziehungen
\begin{align*}
  k &= \frac{2\pi n}{\lambda}
  &&\text{(Allgemein für beliebige Welle)}\\
  \lambda &= \frac{2\pi n}{k} = \frac{2\pi c}{\omega}=\frac{c}{\nu}
            \quad \text{mit } \nu=\frac{\omega}{2\pi}
  &&\text{(Wellenlänge im Vakuum)}\\
  \lambda_m &= \frac{\lambda}{n} 
  &&\text{(Wellenlänge im Medium)}\\
  \omega(k) &= c\cdot k\cdot \frac{1}{n}
\end{align*}%
\nomenclature{$\lambda$}{Wellenlänge; $\lambda=\frac{c}{\nu}$}%
\nomenclature{$\omega$}{Kreisfrequenz; 
  $\omega=2\pi\nu=\frac{2\pi}{T}$}%
\nomenclature{$\nu$}{Frequenz; auch $f$; $\nu=\frac{v_\text{ph}}{\lambda}$}
Weitere wichtige Beziehungen sind
\begin{align*}
  \omega&=2\pi \nu =\frac{2\pi}{T}\\
  c&=\lambda\nu=\frac{\lambda\omega}{2n}
\end{align*}%
\nomenclature{$T$}{Periodendauer}%
Des weiteren gilt für $\vecE$, $\vecD$, $\vecB$ und $\veck$
\begin{align*}
  \veck&\bot\vecD \quad(\text{bzw. }\vecE)&
                                            \veck&\bot \vecB&
                                                              \vecE&\bot \vecB&
                                                                                \vecD&\bot\vecB
\end{align*}
In optisch isotropen Medien gilt $\vecE\bot\veck$ und
$\vert\vecE\vert= \frac{c}{n} \vert\vecB\vert$.
$\veck,\vecD,\vecB$ bilden ein rechtshändiges System.
Elektromagnetische Wellen in isolierenden Medien sind transversale
Wellen (Beweis siehe Folien) mit Ausbreitungsrichtung $\veck$.

Wechselwirkungen zwischen Licht und Materie werden fast immer durch
die elektrische Feldstärke dominiert. Meist werden also nur
$\vecE$-Felder diskutiert. Begründung: Betrachte die Kraft auf
geladenes Teilchen, die durch Wechselwirkung entsteht
\begin{align*}
  \vecF
  &=\vecF_\text{el}+\vecF_\text{mag}=q\vecE +q\vecv\times\vecB\\
  \frac{F_\text{mag}}{F_\text{el}}
  &=\frac{qvB}{qE}
    \underset{(B=\frac{1}{c}E)}{=}\frac{v}{c}
\end{align*}%
\nomenclature{$\vecF$}{Kraft}%
\nomenclature{$F_\text{el}$}{elektrische Kraft}%
\nomenclature{$F_\text{mag}$}{magnetische Kraft, Lorentzkraft}
Daraus folgt: Für $v\ll c$ ist  $F_\text{mag}\ll F_\text{el}$.

\section{Energie von Licht, Poynting-Vektor}
Licht kann Energie transportieren, z.\,B. von der Sonne zur Erde.

In der Elektrodynamik wird die Energiestromdichte einer
elektromagnetischen Welle durch den 
\emph{Poynting-Vektor} $\vecS$%
\nomenclature{$\vecS$}{Poynting-Vektor, Energiestromdichte einer
  elektrom. Welle}%
\index{Poynting-Vektor}
beschrieben.
\begin{align*}
  \vecS(\vecr,t)=\frac{1}{\muo}(\vecE\times\vecB)=\epso c^2 \vecE\times\vecB
\end{align*}
Die zeitliche Mittelung von $\vecS$ über eine Schwingungsperiode $T$ des
Feldes gibt einem die
\emph{Strahlungsflussdichte}\index{Strahlungsflussdichte} 
(mittlere Lichtenergie pro Zeit und Fläche) 
und die \emph{Lichtintensität~$I$}%
\nomenclature{$I$}{Lichtintensität; 
  $I=\langle\vert\vecS\vert\rangle$}%
\index{Lichtintensität}.
Mit $\vert\vecE\vert = \frac{c}{n}$ folgt
\begin{gather*}
  I \coloneqq \langle|\vecS|\rangle
  =\epso nc \langle|\vecE|^2\rangle
\end{gather*}
Im Speziellen gilt für eine ebene Welle
$\vecE(\vecr,t)= \vecE_0 \cos(\omega t -\veck\vecr+\phi)$
mit Bedingungen wie in \eqref{elektrWG} und Brechungsindex $n$
\begin{align*}
  \langle\vert\vecE\vert^2\rangle
  &=\fracone{T}\int_{o}^{T} \vert E_0\vert^2 
    \cos^2( \omega t-\veck\vecr+\phi )dt
    =\fracone{2}\vert E_0\vert^2\\
  \Longrightarrow\quad I&=\fracone{2}\epso n c \vert E_0\vert^2
\end{align*}

\section{Impuls von Licht}
Licht besitzt eine Impulsdichte (wichtig bei Absorption und
Reflexion), eine Art \enquote{Strahlungsdruck}. Beschreibungen in den
beiden Modellen:
\minisec{Teilchenbild}
\begin{align*}
  &\text{Energie des Photons:\index{Photonenenergie}} 
  & E_{Ph}&=\hbar\omega=h\nu\\
  % 
  &\text{Impuls des Photons:\index{Photonenimpuls}} 
  & p &= \frac{E_{Ph}}{c}=\hbar k\\
  % 
  &\text{Gesamtimpuls:} 
  & p_\text{ges} &= \frac{N E_{Ph}}{c}\\
  % 
  &\text{Intensität:} 
  & I &= \frac{N E_{Ph}}{\Delta t A} = \frac{\Phi h \nu}{A}\\
  % 
  &\text{mittlere Photonenflussdichte:\index{Photonenflussdichte}} 
  &\frac{\Phi}{A} &= \frac{I}{n\nu}\\
\end{align*}%
\nomenclature{$t$}{Zeit}%
\nomenclature{$A$}{Fläche}%
\nomenclature{$E_\text{ph}$}{Energie eines Photons}%
\nomenclature{$N$}{Anzahl (einheitenlos)}
\nomenclature{$\hbar$}{normiertes Planksches Wirkungsquantum;
  Naturkonstante;
  $\hbar = \frac{h}{2\pi}$}%
\nomenclature{$\Phi$}{Photonenfluss}%
\minisec{Wellenbild}
Hier wird als Ursache die Wechselwirkung eines elektromagnetischen
Feldes mit einer zunächst ruhenden Ladung $q$ gedeutet.
\begin{itemize}
\item Beschleunigung der Ladung im $\vecE$-Feld 
\item Aus dem Lichtfeld wird Leistung entnommen:\\
  \begin{tabular}{ll}
    Kraft 
    & $\vecF = q\vecE$%
      \nomenclature{$q$}{elektrische Ladung}\\
    Geschwindigkeit 
    & $\vecv_q$%
      \nomenclature{$v_q$}{Geschwindigkeit eines Teilchens mit Ladung $q$}\\
    entnommene Leistung & $L = qEv_q$
  \end{tabular}
\item das sich jetzt bewegende Elektron erfährt eine Lorentzkraft $\vecF_\text{Lorentz}$
  im $\vecB$-Feld ($\vecB\bot\vecE$ und $\vecB\bot\veck$)
\item $\vecF_\text{Lorentz}$ zeigt in Richtung von $\veck$
\end{itemize}

\section{Wellenpakete}
Die Wellengleichung erfüllt das Superpositionsprinzip, d.\,h. 
sind $\vecE_1$ und $\vecE_2$ Lösungen der Wellengleichung, dann ist
auch $\vecE_s=\vecE_1+\vecE_2$ eine Lösung (verwende
Fouriertransformation).
Durch Addition von Wellen verschiedener Frequenzen ($\omega_j$ wird zu
$j\omega_0$) und Amplituden ($E_{0_j}$ wird zu beliebigem
$\vecE(\vecr,t)$ mit Perioden $T=\frac{2\pi}{\omega_0}$) lassen sich
beliebige Lösungen der Wellengleichung konstruieren. Z.\,B. bei
$\vecr=0$:
\begin{gather*}
  \vecE(t)= \sum_{j=-\infty}^{\infty}
  \vecE_{0_j}\exp(i\omega_j t)
\end{gather*}
oder bei kontinuierlicher Verteilung der Frequenzkomponenten
\begin{gather}
  \vecE(t)= c_1\int_{-\infty}^{\infty}\vecE_{0}(\omega)\exp(i\omega t)\dd \omega
  \label{trafo}
\end{gather}
was gerade die Fourierreihe (erster Fall) bzw. die
Fouriertransformation von $\vecE(\omega)$ ist. 
Da $\vecE(t)$ eine reelle Größe ist, kann man auch $E_0(\omega)=
E_0^*(-\omega)$ schreiben. Die Rücktransformation ist:
\begin{gather}
  \vecE(\omega) = c_2\int_{-\infty}^{\infty}
  \vecE_{0}(t)\exp(-i\omega t)\dd t
  \label{ruecktrafo}
\end{gather}
Je nachdem, ob man die normierte Fouriertransformation durchführt oder
nicht, ist der Vorfaktor $c_1=\frac{1}{2\pi}$ in \eqref{trafo} und 
$c_2=1$ in \eqref{ruecktrafo}, oder beide Male $c_1,c_2=\frac{1}{\sqrt{2\pi}}$.
Der Frequenz-~und der Zeitraum sind durch die eindeutigen
Transformationen äquivalent. Die eindimensionale Darstellung ist:
\begin{align}
  E(\omega) = A\exp\left(-\left(
  \frac{\omega-\omega_0}{\delta \omega}
  \right)^2\right)
  + A \exp\left(-\left(
  \frac{\omega+\omega_0}{\delta \omega}
  \right)^2\right)
\end{align}
Einsetzen in Gleichung \eqref{trafo}
\begin{align*}
  E(t)&= \int_{-\infty}^{\infty}
        \frac{\dd\omega}{2\pi} E(\omega)\exp(i\omega t) \\
      &= \frac{A}{\sqrt{\pi}}\frac{\delta\omega}{2}\exp\left[
        -\left( \frac{\delta\omega}{2} \right)^2 t^2
        \right]
        \left( \exp(i\omega_0 t) + \exp(-i\omega_0 t) \right)\\
      &=\frac{A\delta\omega}{\sqrt{\pi}}
        \exp\left[
        -\left(\frac{\delta\omega}{2}\right)^2 t^2
        \right]
        \cos(\omega_0t)
\end{align*}
Das Resultat ist ein Wellenpaket mit Schwingungsfrequenz $\omegao$ und
zeitlich modulierter Amplitude mit $\delta\omega\delta t=2$ 
\begin{align*}
  \Delta\omega_F\Delta t_F=8\ln(2)\approx5,55
\end{align*}
Es sind auch andere Einhüllende möglich, z.\,B. so dass $\Delta\omega_F\Delta
t_F\approx2\pi$, $\Delta\nu_F\Delta t\approx 1$.

\section{Phasen-~und Gruppengeschwindigkeiten}
Ein Lichtimpuls breitet sich \emph{nicht} mit der
Phasengeschwindigkeit $v_\text{Ph}=\frac{\omegao}{k_0}=\frac{c}{n}$,
sondern mit der \emph{Gruppengeschwindigkeit}%
\index{Gruppengeschwindigkeit}%
\nomenclature{$v_\text{gr}$}{Gruppengeschwindigkeit einer Welle;
  $v_\text{gr} = \left( \dif[\omega]{k} \right)_{\omega=\omega_0}$}
aus. 
\begin{gather}
  v_\text{gr} = \left(\ddif[\omega]{k}\right)_{\omega_0}
  =\frac{c}{n}-\frac{kc}{n^2}\ddif[n]{k}
\end{gather}
Wichtig! Zur Berechnung der Gruppengeschwindigkeit benötigen wir die
\emph{Dispersionrelation} $\omega(k)$. Für Licht gilt
$\omega=\frac{c}{n}k$, also für $n=n(k)$
\begin{gather*}
  \frac{\dd \omega}{\dd k}=\frac{c}{n}-\frac{kc}{n^2}\frac{\dd n}{\dd k}
\end{gather*}






%%% Local Variables:
%%% mode: latex
%%% TeX-master: "../OptikSkriptWS1516"
%%% End:
