\chapter[Quantenphänomene]{Quantenphänomene: Welle-Teilchen-Dualismus}

\section[Der Photoeffekt]{Licht als Teilchen: Der Photoeffekt}
Entscheidende Experimente bzw. theoretische Arbeiten zum Verständnis
der Teilchennatur von Licht sind der
\emph{thermische Strahler (Planck 1900)}\index{Thermischer Strahler} 
und der \emph{Photoeffekt (Einstein 1905)}\index{Photoeffekt}.

Der Photoeffekt ist ein qualitatives Experiment.
%% IMAGE MISSING

Erste Versuche zum Photoeffekt (Herz und Hallwachs 1887/88) hatten als
Aufbau eine Metallkathode, die von UV-Licht (Filterung über
Quarzkristall) bestrahlt wird. Bei Bestrahlung kann man einen Strom
zwischen Kathode und einer Anode messen, es werden also Ladungen
freigesetzt.
Das Experiment liefert, dass sichtbares Licht beliebiger Intensität
nicht ausreicht, um Ladungen freizusetzen, nur die Wellenlänge ist
entscheidend.
Daraus kann man schließen, dass das Licht in Quanten auftreten muss.

Die genaue Messung ergibt:
\begin{description}
\item[Variation von $\Phi$:] Mit wachsender Spannung $U$
  steigt der Strom $I(U)$ bis zu einem Sättigungsstrom $I_\text{max}$%
  \nomenclature{$I_\text{max}$}{Sättigungsstrom beim Photoeffekt}, 
  welcher mit $\Phi$ wächst;
  die minimale Spannung $U_0$, bei der Strom gemessen
  wird, bleibt gleich.
  %% IMAGE MISSING

\item[Variation von $\nu$:] Mit sinkender Frequenz $\nu$ steigt die
  minimale Spannung $U_0$; der Sättigungsstrom $I_\text{max}$ bleibt konstant. 
  %% IMAGE MISSING

\item[Variation des Kathodenmaterials:] Bei unterschiedlichem
  Kathodenmaterial ändert sich die minimale Spannung $U_0$.
  %% IMAGE MISSING
\end{description}

Daraus kann man schließen
\begin{itemize}
\item $U_0$ ist unabhängig von der Lichtintensität.
\item $I_\text{max}$ steigt mit der Lichtintensität.
\item $|U_0|$ steigt linear mit der Frequenz $\nu$ an.
\end{itemize}
Eine Erklärung ist, dass durch das einfallende Licht Elektronen aus
der Kathode ausgelöst werden. Diese besitzen eine gewisse kinetische
Energie $E_\text{kin}$, die zum Überwinden der Gegenspannung
eingesetzt wird.
Erst ab der Gegenspannung $U_0(\nu)$ \enquote{schaffen} es die
Elektronen die Anode zu erreichen
\begin{gather*}
  E_{\text{kin,max}}(\nu) = -e U_0(\nu)
\end{gather*}
Die \emph{Einsteingleichung}\index{Einsteingleichung} lautet
\begin{gather*}
    E_{\text{kin,max}}(\nu) = h\nu - A
\end{gather*}
wobei 
$A$\nomenclature{$A$}{Austrittsarbeit beim Photoeffekt; 
Materialkonstante} 
die materialabhängige Konstante für die \emph{Austrittsarbeit} und
$h=\SI{6.626e-34}{\J\s}$%
\nomenclature{$h$}{Plank'sches Wirkungsquantum; 
  $h=\SI{6.626e-34}{\Joule\second}$}
das Plank'sche Wirkungsquantum ist.


%--------------
% VL 04.02.2016
%--------------
\marginpar{04.02.2016}

\begin{enumerate}
\item  \emph{e.\,m. Schritt}\\
  -> Energie des s.g.h.O. mit Eigenfrequenz $\nu$ im harmonischen
  e.m. Feld\\
  -> in kohärente Überlagerung von harmonischen Feldern variabler
  Frequenz

Berechne Gleichgewichtsenergie $U$ des Oszillators, wobei alle Felder
entsprechnd der spektralen Energiedichte $u$ gewichtet werden.
Resultat (empirische Formel, 07.10.1900):
\begin{gather*}
  u(\nu, T) = \frac{8\pi\nu^2}{c^3} U(\nu, T)
\end{gather*}
Es folgt
\begin{gather}
  U(\nu, T) = \frac{h\nu}{\exp(\frac{h\nu}{uT}) - 1}
  \label{GGWEnergie}
\end{gather}
Bedeutung von Formel~\eqref{GGWEnergie}:

\item \emph{Thermodynamischer Schritt}
Ausgehend von Gleichung~\eqref{GGWEnergie} berechnet Planck die
Entropie $S$\nomenclature{$S$}{Entropie}
\begin{align*}
  \dd U &= T\dd S - p\underbrace{\dd V}_{=0} \dotsb \\
  \Rightarrow \frac{\dd S}{\dd U} &= \frac{1}{T}
\end{align*}
Man findet 
$\dd S = \frac{\dd U}{T}$ und 
$\frac{1}{T(U)} = \frac{k_B}{h\nu} \ln\left(\frac{h\nu+U}{U}\right)$
aus Gleichung~\eqref{GGWEnergie} dann
\begin{gather*}
  S = k_B \left[ 
    \left(1+\frac{U}{h\nu}\right) \ln\left(1+\frac{U}{h\nu}\right)
    + \frac{U}{h\nu}\ln\left(\frac{U}{h\nu}\right)
  \right]
\end{gather*}
Die Entropie \emph{muss} diese Form haben!

\item \emph{Statistischer Schritt}
  Wir behandeln nun die Herleitung der Entropie mit statistischen
  Argumenten.
  Die Entropie von $N$ unabhängigen Oszillatoren der Frequenz $\nu$
  ist
  \begin{gather*}
    S_N = k_B\cdot \ln(W_N)
  \end{gather*}
  wobei $W_N$\nomenclature{$W_N$}{thermodynamische Wahrscheinlichkeit;
    Anzahl der möglichen Verteilungen der Gesamtenergie $U_N$ auf $N$
    Oszillatoren}
  die thermodynamische Wahrscheinlichkeit ist, also die Anzahl der
  möglichen Verteilungen der Gesamtenergie $U_N$ auf $N$ Oszillatoren.
  Die Gesamtenergie ist $U_N=N\cdot
  U$\nomenclature{$U_N$}{Gesamtenergie eines Systems mit $N$
    Oszillatoren; $U_N=N\cdot U$}.

  \red{Achtung:} Es gilt $W\to\infty$ für eine beliebige,
  kontinuierliche Verteilung von $U_N$.

  \minisec{1. Postulat (Quantenpostulat)}
  Die Oszillatoren dürfen Energie nur in ganzzahligen Vielfachen einer
  Grundeinheit $\eps$ aufnehmen
  \begin{gather*}
    U_N = p\cdot \eps 
    \qquad \text{mit } p\in\Z, p\gg 1
  \end{gather*}

  \minisec{2. Postulat (statistisches Postulat)}
  Gibt die Anzahl der Möglichkeiten an $p$ ununterscheidbare
  Energiequanten auf $N$ unterscheidbare Oszillatoren zu verteilen
  (Bose-Einstein-Statistik)
  \begin{gather*}
    W_N = \frac{(N+p-1)!}{(N-1)!\cdot p!}
    \overset{\mathllap{\text{Stirling}}}{\approx} 
    \frac{(N+p)^{N+p}}{N^N \cdot p^p}
  \end{gather*}
  
  \minisec{Beispiel}
  Es sei $N=2$, $p=3$ und $W_2=4$.
  Dann sind die vier Möglichkeiten, die gesamte Energie $U=3\eps$ auf
  die beiden Oszillatoren 1~und 2 zu verteilen
  %% IMAGE MISSING (Verteilung der Energie auf Oszillatoren)

  Also
  \begin{align*}
    S_N &= k_B \left[
      (N+p)\ln(N+p) - N\ln(N) - p\ln(p)
      \right]
    &= N \underbrace{
      k_B \left[
      \left(1+\frac{U}{\eps}\right) \ln\left(1+\frac{U}{\eps}\right) 
      - \frac{U}{\eps}\ln\left(\frac{U}{\eps}\right)
      \right]
      }_{\text{Entropie für einen Oszillator}}\\
    S_N &= N\cdot S
  \end{align*}
\end{enumerate}




%%% Local Variables:
%%% mode: latex
%%% TeX-master: "../OptikSkriptWS1516"
%%% End:
