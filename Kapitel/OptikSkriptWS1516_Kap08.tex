%% ----------------------------------------------
%% 
%% Mitschrift der Vorlesung
%% Experimentalphysik III: Wellen und Quanten 
%% im WS2015/16 an der Uni Regensburg
%% 
%% Kapitel 8: Wellenoptik
%% 
%% 
%% gesetzt von Gesina und Hedwig
%% 
%% ----------------------------------------------

\chapter{Wellenoptik}

In der geometrischen Optik wird angenommen, dass wir immer einen
Fokus im Punkt erreichen können, d.\,h. unendlich gute Ortsauflösung.
In der Realität liegt der mögliche Fokus im Bereich der Wellenlänge.

Weiterhin betrachte das Gedankenexperiment, bei dem ein durch eine Blende
erzeugter, rechteckiger Lichtstrahl auf einem Schirm aufgefangen wird.
%% IMAGE MISSING
Das Bild ist kein exaktes Rechteck, denn durch Interferenz zeigt sich
eine Modulation der Intensitätsverteilung. Im Randbereich sorgt die
abnehmende Intensität dafür, dass es keinen scharfen Rand gibt, das
Rechteck wird zu einem unscharfen Fleck.
Allgemein beobachtet man bei Schattenwurf mit monochromatischem Licht
an einer scharfen Kante Licht an Orten, wo nach klassischer
geometrischer Optik keines sein sollte.
Daraus sieht man, dass die Wellennatur des Lichts zur
Abweichung von der geradlinigen Ausbreitung (\emph{Beugung} an
der Blende) und zu Intensitätsmodulationen am Schirm
(\emph{Interferenz}) führt.

% --------------
% VL 10.12.2015
% --------------
\VL{10.12.2015}

\section{Fourier-Transformation}
Mit der Fouriertransformation ist die Fourierreihe für periodische
Funktionen verwandt.

\minisec{Fourierreihe}
Sei $f(t)$ eine komplexwertige periodische Funktion mit Periode
$T=\frac{2\pi}{\omegao}$, dann lässt sie sich als Summe wie folgt darstellen
\begin{align*}
  f(t) &= \sum_{k=-\infty}^{\infty} F_k\cdot e^{ik\omegao t}\\
  \Rightarrow\quad 
  f(t) &= \sum_0^{\infty} A_k\cos(\omega_k t)+B_k\sin(\omega_k t)
\end{align*}
wobei $B_0=0$ und $\omega_k=k\omegao=\frac{2\pi k}{T}$ die
Kreisfrequenz der periodischen Funktion ist. Die Koeffizienten lassen
sich berechnen, indem man beide Seiten mit $\exp(-ikt\omegao)$
multipliziert und über eine Periode integriert
\begin{align*}
  A_k &= \frac{2}{T} \int_{-\frac{T}{2}}^{\frac{T}{2}}
        f(t)\cos(\omega_k t)\dd t
  & k&\neq 0\\
  A_0 &= \fracone{T}\int_{-\frac{T}{2}}^{\frac{T}{2}} f(t)\dd t\\
  B_k &= \frac{2}{T}\int_{-\frac{T}{2}}^{\frac{T}{2}}
        f(t)\sin(\omega_k t)\dd t\\
\end{align*}
Das ergibt für die Summen-Koeffizienten $F_k$ der
Fouriertransformierten
\begin{align*}
  F_k &= \fracone{T}\int_{-\frac{T}{2}}^{\frac{T}{2}}
        f(t) e^{-k\omegao t}\dd t
  &-\infty &\leq k\leq\infty\\
  \text{also}\quad 
  f(t) &= A_0 + \sum_{1}^{\infty}(A_k\cos(\omega_k t) + B_k\sin(\omega_k t)\\
      &= F_0 + \sum_{1}^{\infty}e^{i\omega_kt} 
        \left( \frac{A_k-iB_k}{2} \right)
        + e^{-i\omega_kt}\left( \frac{A_k+iB_k}{2} \right)\\
      &= \sum_{-\infty}^{\infty} F_k e^{i\omega_k t}
\end{align*}
\paragraph{Beispiel 1} periodische Rechtecksfunktion
\begin{align*}
  f(t)&=\left\{
        \begin{array}{ll}
          0&\text{für $-\frac{T}{2}\leq t\leq -\frac{b}{2}$}\\
          a&\text{für $-\frac{b}{2}\leq t\leq \frac{b}{2}$}\\
          0&\text{für $\phantom{-}\frac{b}{2}\leq t\leq \frac{T}{2}$}
        \end{array}
             \right.
\end{align*}
Der nullte Koeffizient lautet dann
\begin{align*}
  F_0 &= \frac{1}{T} \int_{-\frac{T}{2}}^{\frac{T}{2}}
        f(t)e^{-i\omegao t}\dd t
        = \frac{ab}{T}
\end{align*}
Die weiteren Koeffizienten lauten:
\begin{align*}
  F_k &= \fracone{T} \int_{-\frac{b}{2}}^{\frac{b}{2}} 
        ae^{i\omegao t}\dd t
        = \frac{a}{k\pi}\sin(k\omegao\frac{b}{2})
\end{align*}
Im Spezialfall $b=\frac{b}{2}$ (50\% Duty-Cycle) gilt:
\begin{align*}
  F_0 &= \frac{a}{2}
  &F_{\pm 2n} &= 0
  &F_{\pm (2n+1)} &= \frac{a}{(2n+1)\pi}
\end{align*}

\paragraph{Beispiel 2} Übergang zu nicht-periodischen
Funktionen. Betrachte nun wieder Rechtecksfunktionen mit der Breite $b$,
verändere nun aber die Periode bzw. den Abstand zwischen
Rechtecksimpulsen.
Wenn die Periodendauer gegen unendlich geht
($T\longrightarrow\infty$), geht der Abstand der Fourierkoeffizienten
gegen null. Dabei beschreiben die Fourierkoeffizienten immer noch die
gleiche Funktion $f(\omega)$. Führt man nun eine Grenze $T_0$ ein kann
man oberhalb davon, also bei $\vert t\vert >T_0$, $f(t)$
vernachlässigen. Die Koeffizienten werden also durch $\omega=k\omegao$
und $\fracone{T}$ bestimmt. 
\begin{align*}
  F_k &= \fracone{T} \int_{-\frac{T}{2}}^{\frac{T}{2}} 
        f(t) e^{-ik\omegao t}\dd t
\end{align*}
Führe nun eine Funktion $F(\omega)$ ein, deren Funktionswerte bei
$\omega=\omegao k$ die Fourierkoeffizienten $F_k$ bestimmen.
\begin{align*}
  F(\omega) &= \int_{-\frac{T}{2}}^{\frac{T}{2}}
              f(t)e^{-i\omega t}\dd t\\
  F_k &= \fracone{T} F(\omega_k)        
        =\fracone{T}F(k\omega)
\end{align*}
Wenn man dies nun in $f(t)$ einsetzt erhält man
\begin{align*}
  f(t) &= \sum_{k=-\infty}^{\infty}
         \fracone{T} F(k\omegao)e^{ik\omegao t}
\end{align*}
Da $\fracone{T}$ direkt proportional zum Frequenzabstand 
$\dd\omega=\omegao$ zwischen zwei Fourierkomponenten ist, folgt daraus
\begin{align*}
  f(t) &= \sum_{k=-\infty}^{\infty}
         \frac{\dd \omega}{2\pi} F(k\omegao) e^{ik\omegao t}
\end{align*}
Wenn nun die Periodendauer gegen unendlich geht, 
wird $\dd \omega$ infinitesimal klein und man erhält die
Fouriertransformation $f(t)$ mit ihrer Fouriertransformierten
$F(\omega)$.
\begin{align*}
  f(t) &= \int_{-\infty}^{\infty} 
         F(\omega)e^{i\omega t}\frac{\dd\omega}{2\pi}\\
  F(\omega) &= \int_{-\infty}^{\infty} 
              f(t)e^{-i\omega t}\dd t
\end{align*}
Die Abbildung $f(t)\longrightarrow F(\omega)$ ergibt das
Frequenzspektrum der Funktion $f(t)$.

\paragraph{Messungen} Bei Messungen wird meist nur die Intensität in
Abhängigkeit von Zeit und Frequenz gemessen. Die Intensitäten sind
aber nicht über eine Fouriertransformation miteinander verknüpft,
deshalb muss man $I\propto E^2$ ausnutzen, denn die $E$-Felder in
Abhängigkeit zur Zeit und Frequenz sind wieder über eine
Fouriertransformation miteinander verknüpft.

\minisec{Rechenregeln und Spezialfälle}
Betrachte nun die Delta-Funktion $\delta(x)$ bzw. $\delta(\omega)$,
$\delta(t)$. $\delta(t)$ ist überall identisch Null außer bei $t=0$. 
\begin{align*}
  \delta(t) &= \begin{Bmatrix}
    0 &\text{für} &t\neq 0\\
    \infty &\text{für} &t=0
  \end{Bmatrix}\\
  \delta(\omega) &= \fracone{2\pi}\int_{-\infty}^{\infty}
                   e^{-i\omega t}\dd t
\end{align*}
Eine konstante Funktion besitzt also als Fouriertransformierte die
$\delta(\omega)$-Funktion.
\begin{align*}
  \int_{x-a}^{x+a} f(t)\delta(t-\omega)\dd t
  &= f(\omega)\\
  \delta(ax) &= \fracone{a}\delta(x)\\
  \delta(g(x)) &= \sum_{i=1}^{n}\fracone{g\prime (x_i)}\delta(x-x_i)
             &\text{mit } g(x_i)&0&g\prime(x_i)&\neq 0
\end{align*}

\minisec{Der Faltungssatz}
Eine Faltung ist mathematisch definiert als 
\begin{align*}
  f(t)\otimes g(t)
  &\coloneqq \int_{-\infty}^{\infty} f(\zeta)g(t-\zeta)\dd\zeta
\end{align*}
Die Fouriertransformation einer Faltung sieht wie folgt aus ($F$ FT
von $f$, $G$ FT von $g$):
\begin{align*}
  x(t) &\coloneqq g(t)\otimes f(t)
  &\text{FT: } X(\omega) &= G(\omega)F(\omega)\\
  \Rightarrow\quad 
  x(t) &= FT^{-1}(FT(g(t))\cdot FT(f(t))
\end{align*}
Eine Faltung lässt sich berechnen, indem die zu faltenden Funktionen
$g(t)$ und $f(t)$ fouriertransformiert, dann miteinander
multipliziert werden und schließlich das Produkt wieder in den
ursprünglichen Raum rücktransformiert wird.
Hierbei ist die Fouriertransformation (FT) als Operation zwischen
Funktionsräumen aufgefasst (Funktionen $\to$ Funktionen).

\minisec{Spezielle Funktionen}
\paragraph{Gauß-Funktion} Eine (breite) Gauß-Funktion transformiert
immer in eine (schmale) Gauß-Funktion.
\begin{align*}
  F(\omega) &= \fracone{\sigma\sqrt{2\pi}}\int_{-\infty}^{\infty}
              e^{-\frac{t^2}{2\sigma^2}} e^{-i\omega t} \dd t
              = e^{-\frac{\sigma^2\omega^2}{2}}
\end{align*}
Die Gauß-Funktion ist eine Darstellung der $\delta$-Funktion für
$\sigma\longrightarrow 0$.

\paragraph{Exponentialfunktion} Eine (breite) Exponentialfunktion
transformiert in eine (schmale) Lorentzfunktion.
\begin{align*}
  f(t) &= e^{-\frac{\vert t\vert}{\tau}}\\
  F(\omega) &= \frac{2\tau}{1+\omega^2\tau}
\end{align*}

\paragraph{Rechteckfunktion} Eine Rechteckfunktion transformiert in
eine sinc-Funktion.
\begin{align*}
  F(\omega) &= 2\int_{0}^{\frac{T}{2}}\cos(\omega t)\dd t
  &\propto -\frac{\sin(x)}{x} x
\end{align*}

% Bei der Funktion stimmt iwas noch nicht und sie kommt vollständig noch in der Übung dran

% --------------
% VL 14.12.2015
% --------------
\VL{14.12.2015}

Es gilt die \emph{Parseval'sche Formel}\index{Parseval'sche Formel}
\begin{gather*}
  \int_{\infty}^{\infty} |f(t)|^2 \dd t 
  = \int_{\infty}^{\infty} |F(\omega)|^2 \dd\omega
\end{gather*}
und der \emph{Verschiebungssatz}\index{Verschiebungssatz}
\begin{align*}
  FT(f(t-t_0)) = FT(f(x)\otimes\delta(t-t_0))
  = \underbrace{FT(f(t))}_{F(\omega)}e^{-i\omegao t_0}
\end{align*}

% -----

\section{Beugung}
\subsection{Das Huygen'sche Prinzip}
Das \emph{Huygen'sche Prinzip}\index{Huygen'sche Prinzip} lautet
\begin{quote}
  Jeder Punkt einer primären Wellenfront ist Quelle von sekundären
  Elementarwellen
\end{quote}
Die genannten Elementarwellen breiten sich mit der Dispersion des
Mediums aus (also isotrop oder anisotrop).
Die Einhüllende dieser Elementarwellen bildet eine neue Wellenfront.
Die Einhüllende beschreibt nicht die Randbereiche (hier findet Beugung statt).

\paragraph{Fresnel-Huygen'sches Prinzip}
Eine Verbesserung dieser Theorie liefert das 
\emph{Fresnel-Huygen'sches Prinzip}\index{Fresnel-Huygen'sches
  Prinzip}
\begin{quote}
  Das Lichtfeld am Punkt $P$ wird durch \emph{Interferenz} (Summation
  der Amplituden/Phasen) aller Elementarwellen beschrieben.
\end{quote}

% ---

\subsection{Die Fresnel'sche Beugung}
Es sind die \emph{Fresnel'sche Beugung}\index{Fresnel'sche Beugung} 
($d$ groß aber endlich) und die \emph{Fraunhofer'sche Beugung}%
\index{Fraunhofer'sche Beugung} ($d=\infty$) zu unterscheiden.

%% IMAGE MISSING
Die Fresnel'sche Beugung lässt sich anhand des Huygen'schen Prinzips
erklären:
Elementarwellen aus der Ebene $z=0$ erreichen aufgrund der
unterschiedlichen Wegstrecken zu unterschiedlichen Zeiten einen Punkt
$P$ im Abstand $R$\nomenclature{$R$}{Abstand von
  Schirm/Beobachtungspunkt zu Blende} von der Ebene.
Die Strecke von einem Geradenpunkt zu $P$ ist
\begin{gather*}
  r = \sqrt{x^2+R^2}
\end{gather*}
wobei $x$ die Verschiebung bzgl. $P$ parallel zur Ebene ist.

Die Vorgehensweise zur Berechnung der Intensität in $P$ ist, die
Elementarwellen am Ort $P$ mit Phasenfaktor $\exp(ikr)$ aufzusummieren.
Für $x=0$ ist $r=R$. $r$ wächst mit $x$ an und, solange $r<R+\frac{\lambda}{2}$, tragen die
Bereiche der Blendenöffnung konstruktiv zum Feld bei.
Es gilt allgemein für $r$:
\begin{description}
\item[$r<R+\frac{\lambda}{2}$] konstruktive Interferenz
\item[$R+\frac{\lambda}{2}<r<R+\lambda$] destruktive Interferenz
\item[$R+\lambda<r<R+\frac{3\lambda}{2}$] konstruktive Interferenz
\end{description}
% IMAGE MISSING (Fresnelzonen einer Punktquelle)

Beim Blendenversuch tragen die Zonen zur Lichtintensität im Punkt $P$
bei, die nicht von der Kante der Blende abgedeckt werden.

\subsection{Die Fresnel-Kirchhoff'sche Beugungstheorie}
%% IMAGE MISSING
Betrachte die Situation, dass eine Lichtquelle im Punkt $Q$ Licht der
Wellenlänge $\lambda$ ($k=\frac{2\pi}{\lambda}$) emittiert.
Ziel ist nun, die Wellenamplitude bzw. Intensität am Ort $P$ zu
berechnen, der hinter einem Objekt $B$ mit der Transmissionsfunktion
$\Omega$ liegt. $\Omega$ hänge von den Orthonormalkoordinaten $\zeta$
und $\eta$ auf der Ebene senkrecht zur Verbindung von $Q$ und $P$ ab
und sei nur innerhalb der Blendenöffnung ungleich 0.
Die Lösung hierfür über die MWG als Randwertproblem ist
kompliziert. Daher trifft man folgende Näherungen
\begin{itemize}
\item Das skalare Lichtfeld $U$ breitet sich von $Q$ nach $B$
  ungestört aus.
\item Das Blendenobjekt $B$ sei \enquote{vernünftig} (eben und
  beeinflusst das einfallende Lichtfeld nicht)
\item Für die Beeinflussung des Lichtfelds durch die
  Transmissionsfunktions $\Omega(\zeta,\eta)$ auf der rechten Seite der
  Blende gilt ($U_e$ einfallendes Lichtfeld)
  \begin{gather*}
    U_0(\zeta,\eta) = \Omega(\zeta,\eta)\cdot U_e(\zeta,\eta)
  \end{gather*}
\end{itemize}
Aus der Theorie erhält man eine Vorschrift zur Berechnung der
Feldstärke im Punkt $P$
\begin{gather*}
  U_P(\vecR) \propto \iint_{\Omega} 
  U_0(\zeta,\eta)\frac{\exp(ikr)}{r}\dd\zeta\dd\eta
\end{gather*}
Jetzt betrachte eine Beleuchtung der Blende durch eine
Punktlichtquelle in $Q$ mit $\vecR_0=(x_0,y_0,z_0)$ und Feldstärke
$e_0$.
Von dieser Kugelwelle wird eine Kugelwelle mit der Feldstärke $U_0$ in
der Öffnung $\Omega$ erzeugt
\begin{gather*}
  U_0(\zeta,\eta) = \frac{e_0}{r_0} \exp(ikr_0)
\end{gather*}
Wir erhalten
\begin{gather}
  U_P(\vecR) \propto e_0 \iint_{\Omega} 
  \frac{\exp(ik(r+r_0))}{rr_0}\dd\zeta\dd\eta
\end{gather}
Wir treffen die vereinfachende Annahme, dass $r, r_0\gg\lambda$.
Bei Änderung des Beobachtungspunkts $P$ ändert sich das Produkt $rr_0$
viel schwächer als $\exp(ik(r+r_0))$. Ersetze also zur Vereinfachung
$rr_0\approx RR_0$ und ziehe diesen Ausdruck aus dem Integral:
\begin{gather*}
  U_P(\vecR) \propto \frac{e_0}{RR_0} \iint_{\Omega} 
  \exp(ik(r+r_0)) \dd\zeta\dd\eta  
  \label{UP}
\end{gather*}
$r$ und $r_0$ werden in der Exponentialfunktion durch die Koordinaten
$\zeta$ und $\eta$ ersetzt.
Für $Q$ und $P$ weit voneinander entfernt (im Vergleich zum
Blendendurchmesser) gilt
\begin{align*}
  r &= \sqrt{(x-\zeta)^2+(y-\eta)^2+z^2} \\
    &= R-\alpha\zeta - \beta\eta 
      + O\left(\frac{\zeta^2}{R},\frac{\eta^2}{R}\right)
      + \dotsb \\
  r_0 &= R_0-\alpha_0\zeta - \beta_0\eta 
        + O\left(\frac{\zeta^2}{R},\frac{\eta^2}{R}\right)
        + \dotsb \\
\end{align*}
$\alpha$ und $\beta$ bzw. $\alpha_0$ und $\beta_0$ nennt man
Richtungskosinus. Es gilt
\begin{align*}
  \alpha &= \frac{x}{R}
  &\beta &= \frac{y}{R}
  &\alpha_0 &= -\frac{x_0}{R}
  &\beta_0 &= -\frac{y_0}{R}
\end{align*}
(\red{Achtung:} Der Richtungskosinus ist gegen $\zeta(x)$ und $\eta(y)$ zu
berechnen, achte auf das Vorzeichen!)
Insgesamt erhalten wir
\begin{align*}
  r+r_0 = R+R_0 - (\alpha-\alpha_0)\zeta - (\beta-\beta_0)\eta 
  + \underbrace{\psi}_{\mathclap{\text{höhere Ordn.}}}
\end{align*}
Die $k$-Vektoren der gebeugten Strahlen werden beschrieben durch
\begin{align*}
  k_x = k\alpha \qquad \text{und}\qquad k_y = k\beta
  \qquad k=\frac{2\pi}{\lambda}=\sqrt{k_x^2+k_y^2+k_z^2}
\end{align*}

Die Näherungen, die in der Fraunhofer'schen Beugung getroffen werden,
sind:
\begin{itemize}
\item $R\rightarrow\infty$
\item $R_0\rightarrow\infty$
\item $\psi\equiv 0$
\end{itemize}
Dann wird die Formel \eqref{UP} zu
\begin{align*}
  U_P(\alpha,\beta) 
  & \underbrace{\propto \frac{e_0}{RR_0}}_{\nu_{P,0}} 
    \iint_{\Omega} \dd\zeta\dd\eta
    \exp\left( ik(\alpha-\alpha_0)\zeta - ik(\beta-\beta_0)\eta \right)\\
  U_P(\alpha,\beta) 
  &= \nu_{P,0} \int_{-\infty}^{\infty}\int_{-\infty}^{\infty}
    \dd\zeta\dd\eta \Omega(\zeta,\eta)
    \exp\left( ik(\alpha-\alpha_0)\zeta - ik(\beta-\beta_0)\eta \right)
\end{align*}


\subsubsection[Senkrechter Einfall]%
{Spezialfall: Senkrechter Einfall}
Hier gilt $\alpha_0=\beta_0=0$, also
\begin{gather*}
  U_P(\alpha,\beta) 
  = \nu_{P,0} \int_{-\infty}^{\infty}\int_{-\infty}^{\infty}
  \dd\zeta\dd\eta \Omega(\zeta,\eta)
  \exp\left( ik\alpha\zeta - ik\beta\eta \right)
\end{gather*}
Das ist die 2D-Fouriertransformation der Blendenfunktion.

\subsubsection{Beispiele für Fraunhofer'sche Beugung}
\paragraph{Beugung am langen Spalt}\index{Beugung!am langen Spalt}
%% IMAGE MISSING
Wir betrachten senkrecht einfallendes Licht auf einen Spalt mit Breite
$b$ und Höhe $h$ ($h\gg b$). Die Transmissionsfunktion sei 
$\Omega_\text{Spalt}(\zeta,\eta) = \Omega_\text{Spalt}(\eta)$.
Die Integration über $\zeta$ liefert
\begin{align*}
  U_P(\alpha,\beta) 
  &= \nu_{P,0} 
    \int_{-\infty}^{\infty} \exp(k\alpha\zeta)\dd\zeta
    \int_{-\infty}^{\infty} \Omega_\text{Spalt}(\eta)\exp(k\beta\eta)\dd\eta\\
  &= \nu_{P,0} 2\pi\delta(k\alpha)
    \int_{-\infty}^{\infty} \Omega_\text{Spalt}(\eta)\exp(k\beta\eta)\dd\eta
\end{align*}
wobei
\begin{gather*}
  \Omega_\text{Spalt}(\eta) = 
  \begin{cases}
    1 & -\frac{b}{2} < \eta < \frac{b}{2}\\
    0 & \text{sonst}
  \end{cases}
\end{gather*}
Damit wird die Fouriertransformation die einer Rechtecksfunktion:
\begin{align*}
  U_P(\beta) 
  &\propto \frac{\sin(\frac{k\beta b}{2})}{\frac{k\beta}{2}}
  &\text{Intensität}
  &\propto (U_P(\beta))^2
\end{align*}
Insbesondere erhält man für $\beta=0$ 
\begin{gather*}
  I_\text{Spalt}(\beta=0) 
  \propto \lim_{\beta\to0} \left(
    \frac{\sin(\frac{k\beta b}{2})}{\frac{k\beta}{2}}
  \right)^2
  = \dotsb = b^2
\end{gather*}
Damit ist die Formel für Beugungsmuster am langen Spalt
\begin{gather*}
  \frac{I_\text{Spalt}(\beta)}{I_\text{Spalt}(0)}
  = \left(
    \frac{\sin(\frac{k\beta b}{2})}{\frac{k\beta}{2}}
  \right)^2
  = \left(
    \frac{\sin(\frac{\pi b\sin\theta}{\lambda})}
    {\frac{\pi b\sin\theta}{\lambda}}
  \right)^2
  = \left(\frac{\sin B}{B}\right)^2
\end{gather*}
Das \emph{Hauptmaximum} liegt bei $\theta=0$.
Die \emph{Minima} mit $I(\theta_\text{min})=0$ liegen bei den Nullstellen von
$\sin(B)$ (für $B\neq 0$), also 
\begin{align*}
  \sin(B) &= \sin\left(
            \pi\frac{b}{\lambda}\sin\theta_\text{min}
            \right) = 0\\
  \Leftrightarrow\;
  \sin\theta_\text{min} &= \pm\frac{n\lambda}{b}
                        & \text{für } n\in\N_{>0}
\end{align*}
Für die Lage der \emph{Nebenmaxima} muss gelten
\begin{align*}
  \ddif{B} \left( \frac{\sin^2(B)}{B^2} \right) &= 0 
  &\Rightarrow\; \tan(B) &= B
\end{align*}
Diese Gleichung lässt sich graphisch lösen und liefert für die Lage
und Intensität der Nebenmaxima
\begin{align*}
  \sin\theta_\text{max}
  &= \pm \num{1,43}\frac{\lambda}{b},\,
    \pm \num{2,46}\frac{\lambda}{b},
    \dotsc\\
  \frac{I(\theta_\text{max})}{I_0} 
  &= \num{0,047} ,\,
    \num{0,017},
    \dotsc
\end{align*}
Für kleine Winkel $\theta$ kann man direkt die Lage der Minima auf
einem Schirm im Abstand $R$ bestimmen, denn es gilt
$\sin\theta\approx\theta\approx\tan\theta=\frac{y}{z_0}$, also
\begin{gather*}
  y(\pm n) = \pm n\lambda\frac{z_0}{b}
\end{gather*}

% --------------
% VL 17.12.2015
% --------------
\VL{17.12.2015}

\paragraph{Beugung an einer Rechtecksblende}%
\index{Beugung!an Rechtecksblende}
Betrachte eine Rechtecksblende mit 
Höhe $h$\nomenclature{$h$}{Höhe der Blende} und
Breite $b$\nomenclature{$b$}{Breite der Blende}.
Für die Intensitätsverteilung erhält man analog zum langen Spalt nur
unter Berücksichtigung der Beugung in $x$-Richtung
\begin{gather*}
  \frac{I(\alpha,\beta)}{I_0} = \frac{\sin^2(A)}{A^2} \cdot \frac{\sin^2(B)}{B^2}
\end{gather*}
mit $A=\frac{h}{2}\pi\alpha$ und $B=A=\frac{b}{2}\pi\beta$.

\paragraph{Beugung an einer kreisförmigen Öffnung}%
\index{Beugung!an kreisförmiger Öffnung}
Die Beugung an Kreisblenden ist wichtig für das
Auflösungsvermögen optischer Instrumente.
Wir betrachten eine Blende mit Durchmesser
$D$\nomenclature{$D$}{Blendendurchmesser} und erhalten für die
Intensität
\begin{gather*}
  I(r) \propto \left(
    \frac
    { \Bessel \left(\frac{\pi D\sin\theta}{2}\right) }
    { \left(\frac{\pi D\sin\theta}{2}\right) }
    \right)^2
\end{gather*}
wobei $\Bessel$ die Besselfunktion 1.~Ordnung ist.
Es fallen also etwa \SI{85}{\percent} der Intensität auf den Kreis ab,
der vom 1.~Beugungsminimum begrenzt ist. Das erste Beugungsminimum
liegt bei $\sin\theta = \num{1,22}\frac{\lambda}{b}$ 
(\emph{Rayleigh-Kriterium}\index{Rayleigh-Kriterium})

\paragraph{Beugung am Doppelspalt}%
\index{Beugung!am Doppelspalt}\index{Doppelspalt}
Die betrachtete Anordnung ist eine Blende aus zwei Einfachspalten, je mit
Breite $b$, nebeneinander mit Spaltabstand
$a$\nomenclature{$a$}{Spaltabstand beim Mehrfachspalt/Gitter}.
%% IMAGE MISSING (Aufbau Doppelspalt)

Beide Spalte werden mit einer in Näherung ebenen Welle
beleuchtet. Deckt man jeweils einen Spalt ab, so erhält man entsprechend das
Beugungsmuster des Einfachspalts. Die Summe der Beugungsmuster ergibt
aber \emph{nicht} das Beugungsmuster des Doppelspalts!
Beispielsweise sind für $a=3b$ ausgeprägte Modulationen zu erkennen.

% --------------
% VL 21.12.2015
% --------------
\VL{21.12.2015}

\subparagraph{Beschreibung über Fouriertransformation}
\begin{align*}
  \Omega_1(\eta) 
  &= \begin{cases}
    1 & \text{für } 
    -\left(\frac{a+b}{2}\right) < \eta < \frac{a+b}{2}\\
    0 & \text{sonst}
  \end{cases}\\
  \Omega_2(\eta) 
  &= \begin{cases}
    1 & \text{für } 
    -\left(\frac{a-b}{2}\right) < \eta < \frac{a-b}{2}\\
    0 & \text{sonst}
  \end{cases}    
\end{align*}
Der Doppelspalt wird beschrieben durch 
$\Omega_\text{DS}(\eta)=\Omega_1(\eta)-\Omega_2(\eta)$
Berechne zuerst die Fouriertransformation der Blendenfunktion
$\Omega_\text{DS}$ um den Feldverlauf zu erhalten
($k_y=k\beta$, $\beta=\frac{y}{R}=\sin\theta$):
\begin{align*}
  \FT(\Omega_\text{DS}) 
  =\, &\int_{-\infty}^{\infty} 
        (\Omega_1-\Omega_2) e^{-ik_y\eta} \dd\eta \\
  =\, &\int_{-\left(\frac{a+b}{2}\right)}^{+\left(\frac{a+b}{2}\right)}
        e^{-ik_y\eta} \dd\eta 
        -  \int_{-\left(\frac{a-b}{2}\right)}^{+\left(\frac{a-b}{2}\right)}
        e^{-ik_y\eta} \dd\eta \\
  =\, &-\frac{1}{ik_y} \left[
        \exp\left( -ik_y\left(\frac{a+b}{2}\right) \right)
        -\exp\left( ik_y\left(\frac{a+b}{2}\right) \right)
        \right]\\
      &-\frac{1}{ik_y} \left[
        \exp\left( -ik_y\left(\frac{a-b}{2}\right) \right)
        -\exp\left( ik_y\left(\frac{a-b}{2}\right) \right)
        \right]\\
  =\, &\frac{2}{k_y}\left[
        \sin\left(  k_y\left(\frac{a+b}{2}\right) \right)
        -\sin\left( k_y\left(\frac{a-b}{2}\right) \right)
        \right] \\
  =\, &\underbrace{ 2\cos\left( k_y\frac{a}{2} \right) }_{\FT(\Omega_\delta)}
        \cdot \underbrace{ \left(
        \frac{ \sin\left( k_y\frac{b}{2} \right) }{ \frac{k_y}{2} }
        \right) }_{\FT(\Omega_\text{Spalt})}
\end{align*}
Normierung des Intensitätsverlaufs
\begin{gather*}
  I_\text{DS}(\beta=0) 
  \propto \lim_{\beta\to 0} 
  4\cos^2\left( k\beta\frac{a}{2} \right)
  \cdot \frac
  { \sin^2\left( k\beta\frac{b}{2} \right) }
  { \left(k\beta\frac{b}{2}\right)^2 }
  = 4b^2
\end{gather*}
Damit ergibt sich für die Beugungsmuster am Doppelspalt
\begin{gather*}
  \frac{ I_\text{DS}(\beta) }{ I_\text{DS}(\beta=0) }
  = \cos^2\left( k\beta\frac{a}{2} \right)
  \cdot \frac
  { \sin^2\left( k\beta\frac{b}{2} \right) }
  {       \left( k\beta\frac{b}{2} \right)^2 }
\end{gather*}
wobei $\beta=\sin\theta$.

Bei Beleuchtung mit ebenen Wellen wird das Beugungsbild des
Doppelspalts gebildet durch Multiplikation des Beugungsbildes des
Einzelspalts mit $\cos^2\left(\frac{k\beta a}{2}\right)$.
Ist der Spaltabstand wesentlich größer als die Spaltbreite, erhält man
eine hochfrequente Modulation.
Die Bedeutung der Fouriertransformation ist hier:
\begin{align*}
  \Omega_\delta 
  &= \delta\left( \eta-\frac{a}{2} \right) 
    +  \delta\left( \eta+\frac{a}{2} \right)\\
  \FT(\Omega_{DS}) 
  &= \FT(\Omega_\delta) \cdot \FT(\Omega_\text{Spalt})
\end{align*}
Mit dem Faltungstheorem 
$\FT(A\otimes B) = \FT(A)\cdot\FT(B)$
wird das zu
\begin{gather*}
  \Omega_\text{DS} = \Omega_\delta \otimes \Omega_\text{Spalt}
\end{gather*}

%% IMAGE MISSING


\paragraph{Beugung am Gitter}
Wir betrachten ein Beugungsgitter, das aus $N$ langen Spalten mit
konstantem Abstand $a$ besteht.
%% IMAGE MISSING
Die Beugung setzt sich zusammen aus:
\begin{itemize}
\item[$\Omega_\text{Spalt}$] Beugung am Einzelspalt
\item[$\Omega_\text{Gitter}$] Position der Spalte
\end{itemize}
wobei
\begin{align*}
  \Omega_\text{Gitter} = \sum_{m=0}^{N-1}\delta(\eta-ma)
\end{align*}
Die Fouriertransformation von $\Omega_\text{Gitter}$
($\FT(\Omega_\text{Gitter})$) bestimmt das Lichtfeld des Gitters:
\begin{align*}
  U_\text{Gitter}(\beta) 
  &= U_0 \sum_{m=0}^{N-1} \delta(\eta-ma)\\
  &= U_0 \frac
    {e^{-iNk\beta a} - 1}
    {e^{-i k\beta a} - 1} \\
  &= U_0 \cdot \frac
    { \sin\left( Nk\beta\frac{a}{2} \right) }
    { \sin\left(  k\beta\frac{a}{2} \right) }
    \cdot e^{-i(N-1)k\beta\frac{a}{2}}
\end{align*}
Die Intensität für den Grenzfall $\beta=\sin\theta=0$ (senkrechter
Einfall) ist
$I_\text{Gitter}(\beta=0) = N^2$, also
\begin{gather*}
  \frac{ I_\text{Gitter}(\sin\theta) }{ I_\text{Gitter}(0)}
  = \frac
  {    \sin^2\left( Nk\frac{a}{2} \sin\theta \right) }
  { N^2\sin^2\left(  k\frac{a}{2} \sin\theta \right) }
\end{gather*}
%% IMAGE MISSING

Für die Lage der \emph{Hauptmaxima}\index{Hauptmaxima} gilt
\begin{gather*}
  a(\sin\theta-\sin\theta_0) = \pm n\lambda
\end{gather*}
Damit ist die Gittergleichung abhängig von $\lambda$ und es lässt sich
dadurch ein Gitterspektrometer\index{Gitterspektrometer} realisieren.
Für die Lage der \emph{Nullstellen} gilt ($m\in\{1,\dotsc,N-1\}$)
\begin{gather*}
  a(\sin\theta-\sin\theta_0) = \pm \frac{m}{N}\lambda \pm n\lambda
\end{gather*}
Für die Lage der \emph{Nebenmaxima}\index{Nebenmaxima} gilt ($m\in\N$)
\begin{gather*}
  a(\sin\theta-\sin\theta_0) = \pm \frac{2m+1}{2N}\lambda \pm n\lambda
\end{gather*}

Insgesamt erhalten wir, dass Gitter als Spektrometer verwendet werden
können, da es einen eindeutigen Zusammenhang zwischen Wellenlänge und
dem Beugungswinktel $\Sigma = \sin\theta-\sin\theta_0$ gibt.
(\red{Achtung:} Das gilt nur für höhere Ordnungen, nicht für $m=0$.)
Der Strahl mit $m=0$ (\emph{Hauptmaximum}\index{Hauptmaximum}) wird
auch \emph{Weißlicht}\index{Weißlicht} genannt.

Die Breite des Hauptmaximums ist
\begin{gather*}
  \Delta\Sigma 
  \approx \num{0,88}s\cdot \frac{\lambda}{aN} 
  \propto \frac{\lambda}{aN} 
\end{gather*}
und daher umgekehrt proportional zur Anzahl der ausgeleuchteten
Gitterspalte $N$.
Zwei spektrale Komponenten können gerade noch aufgelöst werden, wenn
das Maximum des einen auf das Maximum des anderen gebeugt wird.
D.\,h. das Auflösungsvermögen eines Gitterspektrometers wird
beschrieben durch
\begin{gather*}
  \frac{\Delta\lambda}{\lambda} = \frac{1}{nN}
  \qquad\text{bzw.}\qquad
  \frac{\lambda}{\Delta\lambda} = nN
\end{gather*}
wobei $n$ die Beugungsordnung\index{Beugungsordnung} ist 
und $N$ die Zahl der Gitterstriche.

% --------------
% VL 07.01.2015
% --------------
\VL{07.01.2016}

\subsubsection{Spektrometer}
Für ein Spektrometer fordert man eine hohe spektrale Auflösung bei
großem Lichtdurchsatz.
Im Folgenden sind einige der verschiedenen Arten aufgeführt.

\paragraph{Prismenspektrograph}\index{Prismenspektrograph}
Der Vorteil an diesem ist die eindeutige Zuordnung der Wellenlänge. 
Er besitzt aber nur einen geringen Lichtdurchsatz und eine geringe
Dispersion.

\paragraph{Gitterspektrograph}\index{Gitterspektrograph}
mit Reflexionsgitter und Reflexionsoptik. 
Es besteht aufgrund der Gittergleichung ein fester Zusammenhang
zwischen Ablenkwinkel $\theta_a=\alpha+\beta$ und der Wellenlänge.
%% IMAGE MISSING (Skizze zum Gitterspektrometer)
Durch Drehen des Gitters bei festem $\theta_a$
werden Ein-~und Ausfallswinkel geändert und so die
Transmissionswellenlänge bestimmt.
Für $\theta_a\approx 0$ (bspw. in der
Litrow-Anordnung\index{Litrow-Anordnung}) gilt $\alpha=-\beta$.
Die \emph{Gittergleichung für die Litrow-Anordnung}%
\index{Gittergleichung!Litrow-Anordnung} ist
\begin{align*}
  2a\sin\beta &=n\lambda\\
  \text{bzw.}\quad \lambda &= \frac{2a}{n}\sin\beta\,,
\end{align*}
wobei $a$ der Gitterabstand und $n$ die Beugungsordnung ist. 
Die \emph{apparative Wellenlängenauflösung} wird durch die Spaltbreite
und die Brennweite des Spektrometers wie folgt bestimmt
\begin{align*}
  \Delta\lambda &=\diff[\lambda]{\beta}\Delta\Phi\\
  \Delta\Phi &\approx \frac{\beta_1}{f}
\end{align*}
Die zweite Gleichung ist die Divergenz des Lichtbündels vom
Hohlspiegel auf das Gitter, $b_1$ die Breite des Eintrittspalts und
$f$ die Brennweite des Hohlspiegels. Man erhält eine bessere Auflösung
bei einem langen Spektrometer und kleinem Spalt. Der Austrittsspalt
bestimmt die Genauigkeit in der Bestimmung des Ablenkwinkels
$\theta_a$. Weiter nehmen wir an, dass $B_1=B_2=B$.
\begin{align*}
  \Delta\lambda 
  \approx \diff[\lambda]{\beta}\Delta\Phi
  = \frac{2a}{n}\cos\beta \cdot \frac{B}{f}
\end{align*}
Man erhält somit eine ideale Auflösung für $B\approx\lambda$.

\paragraph{Beugung am 2D-Gitter}
\sFolien

\paragraph{Beugung am 3D-Gitter}
Die Beugung am 3D-Gitter ist Wichtig für die Strahlenanalyse, z.\,B. bei
\begin{itemize}
\item Röntgenbeugung (im Bereich von \SI{10}{\kilo\eV})
\item Elektronenbeugung (50–100\si{\eV})
\end{itemize}

% -----

\section{Interferenz und Kohärenz}
Betrachte zwei Punktquellen $Q_1$ und $Q_2$, die am Ort $\vecr_0$ zur
Überlagerung der elektrischen Felder führen. 
\begin{align*}
  E_\text{ges}(x=0) 
  = E_1+E_2
  =\frac{A_1}{x_0}e^{i\omega t-ikx_0} + \frac{A_2}{x_0}e^{i\omega t- ikx_0}
\end{align*}
mit $A_1=A_2=A$ folgt für die Intensitäten:
\begin{align*}
  I(0)
  = \fracone{2}\epso nc\left\langle\left|E_1 + E_2\right|^2\right\rangle 
  = \epso nc\frac{2A^2}{x_0^2}
  = 4I_1
\end{align*}
Man erhält also \emph{konstruktive Interferenz}.

Bewege jetzt den Beobachtungspunkt $P$ um $\Delta x$. 
Dadurch haben die Wellen einen relativen Phasenunterschied von
$\Delta\phi = 2\Delta x\frac{2\pi}{\lambda}$.
Wenn $\Delta x=\frac{\lambda}{4}$, ist $\Delta\phi=\pi$ und man erhält
\emph{destruktive Interferenz}, wobei die mittlere Intensität gleich 0
ist.

\red{Achtung:} Dies ist nur für \emph{kohärentes} Licht möglich, also
für Licht mit fester Phasenbeziehung. Schwankt die Phase statistisch,
so kann man (je nach Beobachtungsdauer) die Interferenz nicht mehr
beobachten.

\minisec{Kohärenz von Lichtquellen}
Wir betrachten eine Abschätzung der Kohärenzzeit aus der
\enquote{Unschärfebeziehung} $\D t\D\nu\approx1$.
Das Weißlicht einer Glühbirne hat eine Bandbreite von
$\D\nu=\SI{4e14}{\hertz}$. Daraus erhält man die \emph{Kohärenzzeit}
$t_c=\frac{l_c}{c}=\SI{2,5e-15}{\second}$ und die
\emph{Kohärenzlänge}\index{Kohärenzlänge}
\begin{align*}
  l_c=t_c c_{Licht}
\end{align*}
Kohärentes Licht für Interferenzexperimente wird typischerweise durch
\enquote{Spaltanordnungen} realisiert, wie etwa
\begin{itemize}
\item Young'scher Doppelspalt
\item Fresnel'sches Biprisma
\item Fresnel'scher Doppelspiegel
\end{itemize}
Betraachte nun die \emph{räumliche Kohärenz} bei einem Doppelspalt.
%% IMAGE MISSING (Skizze mit Doppelspalt)

Die Lichtquelle hat eine Ausdehnung von $2d$, den Abstand $z_0$ vom Schirm,
den Spaltabstand $a$ und den Gangunterschied $\D g=g_1-g_2\approx a\alpha$.
Wenn $\D g\ll\frac{\lambda}{2}$, liegt räumliche Kohärenz vor. Die
Interferenz verschwindet, wenn das Beugungsmaximum im Beugungsminimum
\enquote{sitzt}.

\red{Achtung:} gilt für alle Punkte der Lichtquelle.

Für große Abstände $z_0$ gilt $\D g=\frac{ad}{z_0}\ll\frac{\lambda}{2}$
mit Öffnungswinkel $\phi=\frac{a}{z_0}$ oder
\begin{align*}
  \D g = \frac{ad}{z_0} &\ll \frac{\lambda}{2}\\
  \phi &\ll \frac{\lambda}{2d}
\end{align*}
als Bedingung für räumliche Kohärenz. Kohärenz liegt dann vor, wenn
der Durchmesser der Quelle und genannter Öffnungswinkel klein sind.

% --------------
% VL 11.01.2016
% --------------
\VL{11.01.2016}

\section{Zweistrahlinterferenzen}
\subsection{Interferenzen dünner Schichten}
Bei dünnen, dielektrischen Schichten mit $n\approx\num{1.5}$ zeigt
sich ein Reflexionsvermögen von ca. \SI{4}{\percent}.
Daher wird nur der 1.~und 2.~Reflex an der zweiten Grenzfläche
betrachtet. Es zeigt sich \emph{Zweifachinterferenz} (der Strahl, der an der
1.~Grenzfläche reflektiert wird, interferiert mit dem Strahl, der an der
2.~Grenzfläche reflektiert und durch die 1. wieder transmittiert wird).
%% IMAGE MISSING

Die Lage der Interferenzmaxima~bzw. -minima ist durch die
Phasenverschiebung bei Reflexion
(\emph{Fresnel-Koeffizient}\index{Fresnel-Koeffizient}) und durch den
Gangunterschied gegeben.
\begin{enumerate}[a)]
\item \emph{Interferenzen gleicher Neigung}
  Betrachte einen ausgedehnten, parallelen Film der Dicke
  $d$\nomenclature{$d$}{Filmdicke} mit Brechungsindex
  $n_f$ (umgebendes Medium habe $n_1=1$ und $n_2=1$), auf den ein Strahl mit
  Einfallswinkel $\theta_1$ trifft. 
  Es ergeben sich Phasensprünge $\Delta\Phi_1$ und $\Delta\Phi_2$ an der
  1.~ bzw. 2.~Grenzfläche.

  Der Winkel des transmittierten Strahls zur 2.~Grenzfläche sei
  $\theta_f$ (gegeben mit Snellius).
  Wir bezeichnen mit $a=b$ die Strecke, die der transmittierte Strahl
  von der 1.~zur 2.~Grenzfläche (bzw. umgekehrt) zurücklegt und mit $x$
  den Streckenunterschied der betrachteten zurückgeworfenen Strahlen.
  %% IMAGE MISSING (Stahlteilung am ausgedehnten dünnen Film mit Phasenverschiebungen)

  Es gilt
  \begin{align}\label{*}
    a &= 2d\tan\theta_f \\\nonumber
    b &= \frac{d}{\cos\theta_f}\\\nonumber
    x &= a\sin\theta_1
  \end{align}
  Für die Phasensprünge gilt
  \begin{alignat}{2}\label{Phasendifferenzen}
    \Delta\Phi_1 &= \pi &\quad\text{falls } &n_f>n_1\\\nonumber
    \Delta\Phi_1 &= 0   &\text{falls } &n_f<n_1\\\nonumber
    \Delta\Phi_2 &= \pi &\text{falls } &n_2>n_f\\\nonumber
    \Delta\Phi_2 &= 0   &\text{falls } &n_2<n_f
  \end{alignat}
  Wichtig für die Interferenz ist die Phasendifferenz
  \begin{gather*}
    \Delta\Phi \coloneqq \Delta\Phi_2 - \Delta\Phi_1
  \end{gather*}
  Wegen \eqref{Phasendifferenzen} gibt die drei Möglichkeiten
  \begin{gather*}
    \Delta\Phi \pm \pi  
    \qquad\text{oder}\qquad 
    \Delta\Phi = 0\,,
  \end{gather*}
  je nachdem, ob der Film einen größeren oder
  kleineren Brechungsindex als \emph{beide} umgebenden Materialien hat,
  oder bei einem größer, beim anderen kleiner. Genauer
  \begin{align*}
    \Delta\Phi 
    &=
      \begin{cases}
        \phantom{-}\pi & n_f<n_1\text{ und }n_f<n_2\\
        -\pi           & n_f>n_1\text{ und }n_f>n_2\\
        \phantom{-}0   & n_1<n_f<n_2\\ 
        \phantom{-}0   & n_2<n_f<n_1\\
      \end{cases}
  \end{align*}
  Der Gangunterschied $\GU$\nomenclature{$GU$}{Gangunterschied} ist
  \begin{gather*}
    \GU = 2n_f b - n_1x 
  \end{gather*}
  Verwende \eqref{*} und Snellius ($n_1\sin\theta_1 = n_f\sin\theta_f$),
  um $\GU$ zu berechnen:
  \begin{align*}
    \GU &= 2 n_f\frac{d}{\cos\theta_f} 
          - n_f\frac{\sin\theta_f}{\sin\theta_1} \cdot a\sin\theta_1\\
        &= 2 n_f\frac{d}{\cos\theta_f} - n_f\sin\theta_f \cdot a\\
        &\overset{\mathllap{\eqref{*}}}{=} 
          2 n_f\frac{d}{\cos\theta_f} - n_f\sin\theta_f \cdot 2d\tan\theta_f\\
        &= \frac{2d}{\cos\theta_f}\left( n_f-n_f\sin^2\theta_f\right)\\
        &= \frac{2dn_f}{\cos\theta_f}\cos^2\theta_f\\
        &= 2dn_f\cos\theta_f
  \end{align*}
  Wir erhalten konstruktive Interferenz bei
  \begin{gather*}
    \frac{\GU}{\lambda} = \frac{\Delta\Phi}{2\pi} = m 
    \qquad m\in\N
  \end{gather*}
  Konstruktive Interferenz findet statt, wenn folgendes erfüllt ist
  \begin{gather*}
    2n_f d\cos\theta_f = \left( m+\frac{\Delta\Phi}{2\pi} \right)\lambda
  \end{gather*}
  Hier bilden sich \emph{Haidingerschen Ringe}\index{Haidingerschen Ringe}.

\item \emph{Interferenzen gleicher Dicke}
  Betrachte eine Variation der optischen Dicke $n_f\cdot d$.
  (z.\,B. Ölfilme oder Seifenblasen).
  %% IMAGE MISSING (Zweistrahlinterferenz am Ölfilm auf Wasser)

  Wir nehmen senkrechten Lichteinfall an, dann ergibt sich eine
  konstruktive Interferenz bei
  \begin{gather*}
    \frac{2 n_f d}{\lambda} = (m+\frac{1}{2})
  \end{gather*}

  Betrachtet man zwei Platten, die in einem Winkel $\alpha$ übereinander
  liegen (äquidistant), ergeben sich nach Keil Interferenzen bei
  \begin{gather*}
    \Delta x = \frac{\lambda}{2\alpha}
  \end{gather*}
  %% IMAGE MISSING

  So kann auch die Präzision einer Linse geprüft werden: Lege die Linse
  auf eine planare Platte (Krümmungsradius $R$ der Linse wesentlich
  größer als maximaler Abstand der Linse zur Platte).
  Dann sollten sich konzentrische Kreise (Newton'sche Ringe) vom Radius
  \begin{gather*}
    r_m = \sqrt{\frac{(m+1)R\lambda}{n_f}}
  \end{gather*}
  für den $m$ten-Ring bilden (s. Übung).
  In gleicher Weise kann ein kleiner Abstand zweier Objekte oder die
  Dicke einer Schicht/Stufen auf einer dünnen Schicht gemessen werden.

\item \emph{Dielektrische Schichten und Antireflexvergütungen}
  Hat das an Forder-~und Rückseite der dünnen Schicht reflektierte Licht
  die gleiche Feldstärke, kann es bei destruktiver Interferenz zu
  völliger Auslöschung kommen. 
  So können \emph{Anti-Reflex-Schichten}\index{Anti-Reflex-Schicht}
  zustande kommen.

  Betrachte eine Glasplatte mit Brechungsindex $n_2$ und eine Schicht
  mit Dicke $d$ und Brechungsindex $n_f$, die aus Luft ($n_1=1$) heraus
  beleuchtet wird. Es sei $n_f<n_2$, also $n_1<n_f<n_2$, so dass die
  Phasenverschiebung $\Delta\Phi$ in Reflexion Null wird.
  Wir nehmen senkrechten Einfall an. Dann findet destruktive
  Interferenz (1.~Minimum) bei
  \begin{gather*}
    2n_fd = \frac{\lambda}{2}
  \end{gather*}
  statt. Die optische Schichtdicke $dn_f$ muss also
  $\frac{\lambda}{4}$ sein, auch \enquote{$\frac{\lambda}{4}$-Schicht}
  genannt.
  Für völlige Auslöschung müssen die Reflexionskoeffizienten beim
  Eintritt und Austritt aus der Schicht (mit $n_f$) gleich groß
  werden, die Bedingung für eine 
  \emph{Anti-Reflex-Vergütung}\index{Anti-Reflex-Vergütung} in
  Abhängigkeit von $\lambda$ ist also
  \begin{gather*}
    n_f = \sqrt{n_1n_2} 
    \qquad\text{und}\qquad
    n_fd = \frac{\lambda}{4}
  \end{gather*}

\end{enumerate}

% ---

\subsection{Interferometer}
\paragraph{Michelson-Interferometer}%
\index{Interferometer!Michelson-Interferometer}
%% IMAGE MISSING

Das Michelson Interferometer kann zur Längenmessung verwendet werden.
Es ergibt sich Interferenz durch eine räumliche Aufspaltung der
Wellenamplitude.
Außerdem kann man die Kohärenzlänge $\Delta s$ einer Lichtquelle
messen, da diese sich proportional zu $\frac{1}{\Delta\lambda}$
(d.\,h. invers zur spektralen Bandbreite) verhält ($N$ ist die Anzahl
der durchlaufenen Intensitätsmaxima):
\begin{gather*}
\Delta s = \frac{N\lambda}{2}
\end{gather*}
Besonders gut geeignet ist z.\,B. die Wellenlänge der roten Cadmium-Linie.

% --------------
% VL 14.01.2016
% --------------

\section{Vielstrahlinterferenz: Fabry-Perot-Interferometer}
\index{Vielstrahlinterferenz}
\index{Interferomenter!Fabry-Perot-Interferometer}
Vielfachinterferenzen können zu scharfen Interferenzfiguren führen.
Der grundsätzliche Aufbau sieht vor, dass der einfallende Strahl
(z.\,B. durch Reflexion und Transmission an Schichtübergängen)
geteilt wird und so wie beim Gitter mehrere Strahlen derselben Quelle
miteinander interferieren.
%% IMAGE MISSING Zinth, Abb. 4.46
Man erhält konzentrische Kreise als Interferenzen gleicher-Neigung,
die sogenannten \emph{Fabry-Perot-Ringe}\index{Fabry-Perot-Ringe}

Mögliche Bauformen sind:
\begin{description}
\item[Reflexionsschichten] mit definiertem Abstand $d$ im Vakuum
\item[planparallele Platte] mit hochreflektierenden Schichten
\end{description}

Für die Funktionsweise betrachte eine Schar ebener Lichtbündel, die
schräg im Winkel
$\theta_F$\nomenclature{$\theta_F$}{Fabry-Perot-Interferometer:
  Winkel, in dem die Lichtbündel auf die Platte treffen}
auf eine planparallele Platte mit Dicke $d$ und Brechungsindex $n_F$
trifft.
Zur Berechnung der gesamten Reflexion und Transmission mache die
Annahme eines idealen Interferometers, d.\,h. vernachlässige
Absorption und Streuung.
Es seien 
$t$%
\nomenclature{$t$}{Fabry-Perot-Interferometer:
  Transmissionskoeffizient bei Eintritt in die planparallele Platte}%
/$r$%
\nomenclature{$t$}{Fabry-Perot-Interferometer:
  Reflexionskoeffizient bei Eintritt in die planparallele Platte}
die Transmissions-/ Reflexionskoeffizienten der
Amplitude für Eintritt in die Platte
und $t'$%
\nomenclature{$t$}{Fabry-Perot-Interferometer:
  Transmissionskoeffizient bei Austritt aus der planparallelen Platte}%
/$r'$%
\nomenclature{$t$}{Fabry-Perot-Interferometer:
  Reflexionskoeffizient bei Austritt aus der planparallelen Platte}
die Transmissions-/ Reflexionskoeffizienten der
Amplitude für Austritt aus der Platte,
sowie $\delta$%
\nomenclature{$\delta$}{Fabry-Perot-Interferometer: Phasenverschiebung
  des elektrischen Felds bei einfacher Reflexion;
$\delta = \frac{4\pi n_F d}{\lambda}\cos(\theta_F)$}
die Phasenverschiebung des elektrischen Felds bei einfacher Reflexion.
Für die Feldstärke gilt dann
\begin{align*}
  E_r &= E_{1,r} + E_{2,r} + E_{3,r} + \dotsb \\
      &= E_0r + E_0tr't'e^{i\delta} + E_0tr'r'r't'e^{i\delta} +
        \dotsb \\
      &= E_0r + E_0tt'r'e^{i\delta}\cdot\left(
        1 + {(r')}^{2}e^{i\delta} 
        + \left( {(r')}^{2}e^{i\delta} \right)^2 + \dotsb
        \right)\\
      &= E_0r + E_0tt'r'e^{i\delta}\cdot 
        \sum_{j=0}^{\infty}\left( {(r')}^{2}e^{i\delta} \right)^{j}
\end{align*}
Da $\left|(r')^2e^{i\delta}\right|<1$ konvergiert die geometrische
Reihe und die Gleichung lässt sich umschreiben zu
\begin{gather*}
  E_r = E_0\left( 
    r + \frac{r'tt'e^{i\delta}}{1-(r')^2e^{i\delta}} 
  \right)
\end{gather*}
Mit $t\cdot t' = 1-r^2$ und $-r=r'$ (s.\,\nameref{Fresnel}) folgt
\begin{align*}
  E_r &= E_0\left( 
    r + \frac{r(1-e^{i\delta})}{1-r^2e^{i\delta}} 
  \right)\\
  I_R &= I_0\;\frac{2r^2(1-\cos\delta)}{(1+r^4)-2r^2\cos\delta} 
        = I_0\;\frac
        {F\sin^2\left(\frac{\delta}{2}\right)}
        {1+F\sin^2\left(\frac{\delta}{2}\right)}
\end{align*}
wobei $F\coloneqq \left(\frac{2r}{1-r^2}\right)^2 
= \frac{4R}{(1-R)}^2$\nomenclature{$F$}{Konstante für das
  Fabry-Perot-Interferometer;
  $F\coloneqq \frac{4R}{(1-R)}^2$},
$R=r^2$ der Intensitätskoeffizient,
$I_0$ die transmittierte Intensität
ist und
$\delta = \frac{4\pi n_F d}{\lambda}\cos(\theta_F)$
die Phasenverschiebung beschreibt.

Mit $I_0=I_R+I_T$ erhalten wir die \emph{Transmission des
Fabry-Perot-Interferometers} als
\begin{gather*} 
I_T = I_0\;\frac{1}{1+F\sin^2\left( \frac{\delta}{2} \right)}
\end{gather*}
Diese Funktionsform wird \emph{Airy-Funktion}\index{Airy-Funktion}
genannt.
Für $T\approx 1$ erhält man die Lage der \emph{Transmissionsmaxima} mithilfe
der Bedingung
\begin{gather*}
  2n_Fd\cos\theta_F = m\lambda \qquad m\in\N
\end{gather*}
\emph{Minimale Transmission} ist bei
\begin{gather*}
  T_\text{min} = \frac{1}{1+F} = \frac{(1-R)^2}{(1+R)^2}
\end{gather*}
Das Reflexionsvermögen beeinflusst die Spiegelschichten wesentlich,
s.\,\autoref{tab:finesse}.
\begin{table}
\centering
\captionabove[Reflexionskonstanten von Fabry-Perot-Interferometern]
{\label{tab:finesse}
  Einfluss des Reflexionsvermögens der Spiegelschichten eines
  Fabry-Perot-Interferometers}
\begin{equation*}
\begin{array}{ r l l r }
  \toprule
  \multicolumn{1}{l}{R} & F & T_\text{min} & \multicolumn{1}{l}{\text{Modulationstiefe}}\\
  \midrule[\heavyrulewidth]
  \SI{99}{\percent} & \num{1520} & \num{6.6e-4} & \SI{100}{\percent}\\
  \midrule
  \SI{10}{\percent} & \num{0.5} & \num{0.66} & \\
  \midrule
  \SI{4}{\percent} & \num{0.17} & \num{0.85} & \SI{15}{\percent}\\
  \bottomrule
\end{array}
\end{equation*}
\end{table}
Für große $R$ wird der Verlauf von $I_T$ in der Nähe eines Maximums
ähnlich einer Lorentzfunktion mit Halbwertszeit 
$\Delta\delta=\frac{4}{\sqrt{F}}$.
Daraus ist die Definition der sog. \emph{Finesse}\index{Finesse}
entstanden als
\begin{gather*}
  \widetilde F 
  \coloneqq \frac{\text{Abstand Maxima}}{\text{Breite Maxima}} 
  = \frac{2\pi\sqrt{F}}{4}
  = \frac{\pi\sqrt{R}}{(1-R)}
\end{gather*}%
\nomenclature{$\widetilde F$}{Finesse; 
  $\widetilde F = \frac{\pi\sqrt{R}}{(1-R)}$}

\section[Holographie]{Holographie (D. Gabor)}\index{Holographie}
Um die volle Information eines Objekts aufzuzeichnen, muss die
Amplituden-~und Phaseninformation des elektrischen Feldes aufgenommen
werden, was zu einem \emph{Hologramm} führt.
Die Idee ist die Überlagerung zweier ebener, kohärenter Teilwellen,
welche ein streifenförmiges Interferenzmuster erzeugt.
Für die Umsetzung kann man den von einem Spiegel reflektierten Anteil
mit dem am Objekt gestreuten Anteil des kohärenten Lichts
interferieren lassen.

\subsection*{1. Schritt: Aufzeichnung des Hologramms}%
\index{Hologramm!Aufzeichnung}
  \nomenclature{$E_O$}{Objektwelle in der Holographie}
  \nomenclature{$E_R$}{Referenzwelle in der Holographie}
  Man beleuchtet das Objekt mit \emph{monochromatischem} Licht und
  nimmt die Überlagerung des \enquote{direkten} Lichts (am Spiegel
  reflektiert) mit dem vom Objekt gestreuten Licht auf.
  Dadurch erhält man die Phaseninformation mithilfe der Interferenz
  (bei ausreichender Kohärenzlänge ergibt sich ein
  Interferenzstreifenmuster):
  \begin{description}
    \item[Objektwelle] 
      $\displaystyle E_0(x,y) = E_{0,0}(x_y)\exp(i\omega t+i\phi_0(x,y))$\\
      Die Objektwelle enthält die Bildinformation mit einer stark
      ortsabhängigen Amplitude und Phase.
    \item[Referenzwelle]
      $\displaystyle E_R(x,y) = E_{R,0}(x_y)\exp(i\omega t+ i\phi_R(x,y))$\\
      Die Referenzwelle hat eine konstante Amplitude und einen
      \enquote{einfachen} Phasenverlauf.
  \end{description}
  Zusammen ergibt sich die Intensität auf dem Film als
  \begin{align*}
    I(x,y) & \propto E_{R,0}^2 + E_{0,0}^2 
             + E_{R,0}E_{0,0}(x,y)\left(
             e^{i(\phi_0(x,y)-\phi_R(x,y))}
             + e^{-i(\phi_0(x,y)-i\phi_R(x,y))}
             \right)\\
    I(x,y) &\propto E_{R,0}^2 + E_{0,0}^2 
             + \underbrace{
             2E_{R,0}E_{0,0}(x,y)
             \cos\left(\phi_0(x,y)-\phi_R(x,y)\right)}_{
             \mathclap{\substack{
             \text{hochfrequente Modulation}\\
    \text{des Intensitätsverlaufs}}}
    }
  \end{align*}
  Diese enthält nun die Gesamtinformation über das Objekt, da
  $\phi_R(x,y)$ bekannt ist.

  % --------------
  % VL 18.01.2016
  % --------------
  \VL{18.01.2016}

\subsection*{2. Schritt: Auslesen des Hologramms}
  Nach Entwicklung der Photoplatte erhält man ein
  \emph{Transmissionsbild} $T(x,y)$%
  \nomenclature{$T(x,y)$}{Transmissionsbild eines Hologramms}
  (nur Streifen).
  Das \enquote{Beugungsgitter} wird jetzt wieder mit einer Referenzwelle
  beleuchtet.
  Das Auslesefeld $E_L(x,y)$%
  \nomenclature{$E_L(x,y)$}{Auslesefeld beim Auslesen eines Hologramms} 
  soll den gleichen Amplituden und Phasenverlauf haben wie $E_R(x,y)$
  \begin{gather*}
    E_L(x,y) = E_{R_0}\exp(i\omega t + i\Phi_R(x,y))
  \end{gather*}
  Das Transmissionsbild $T(x,y)$ sei proportional zur Intensität des
  gesamten Beleuchtungslichts auf der Platte
  \begin{gather*}
    T(x,y) = \kappa \cdot I(x,y)
  \end{gather*}
  Durch Beleuchtung wird das Bildfeld $E_B(x,y)$\nomenclature{$E_B(x,y)$}{Bildfeld} 
  erzeugt
  \begin{align*}
    E_B(x,y) &= E_L(x,y) \cdot T(x,y)\\
    E_B(x,y) &= E_{R_0}\exp(i\omega t + i\Phi_R(x,y))\cdot \kappa 
               \cdot\left\{E_{R_0}^2 + E_{O_0}^2 +
               2E_{R_0}E_{O_0}(x,y)\cos(\Phi_O(x,y))\right\}\\
             &= E_1 + E_2 + E_3\\
    E_B &= \kappa E_R(x,y) \left( E_{R_0}^2 + E_{O_0}^2 \right)\\
             &\phantom{=}\; + 
               \kappa E_{R_0}^2 E_{O_0}(x,y)\exp(i\omega t + i\Phi_O(x,y))\\
             &\phantom{=}\; + 
               \kappa E_{R}^2 E_{O_0}(x,y)\exp(i\omega t-\Phi_O(x,y)+2i\Phi_R(x,y))
  \end{align*}
  Man sieht:
  \begin{itemize}
  \item[$E_1$:] proportional zum Auslesefeld, aber abgeschwächt;
  \item[$E_2$:] direkt proportional zum Objektfeld an der Photoplatte;
    verhält sich so, als ob die Information direkt vom Ort des Objekts
    käme, enthält dieselbe Information wie das Original (virtuelles
    Bild); Hologramm
  \item[$E_3$:] hat gegenüber $E_2$ negative Phase (reelles Bild)
  \end{itemize}


%%% Local Variables:
%%% mode: latex
%%% TeX-master: "../OptikSkriptWS1516"
%%% End:
