%%-----------------------------------------

%% Vorlesungsmitschrift (Kapitel 3)

%% an der Uni Regensburg, gelesen von Christian Back

%%-----------------------------------------

\chapter{Dispersion von Licht}
Die Ausbreitung von Licht hängt vom Brechungsindex $n=n(\omega)$
ab. $n(\omega)$ bestimmt die Geschwindigkeit von Licht, das
Auseinanderfließen von Lichtimpulsen, Ablenkungen und Reflexion an
Grenzflächen. 

\section{Die Frequenzabhängigkeit der Dielekrizitätskonstante $\eps$}
I.A. ist $\eps$ ein Tensor mit $\omega$-Abhängigkeit. Vergleicht man
die statische Dielektrizitätskonstante bzw. den Brechungsindex
$n_0=\sqrt{\eps(\omega=0)}$ mit $
n=\sqrt{\eps(\omega=\SI{589}{\nano\meter})}$, ist ein Unterschied zu erkennen! Typischerweise zeigt $\eps$ die Resonanzen an. Die Ursache ist mikroskopisch, da die Elektronenwolke gegenüber dem Atomkern durch das $\vecE$-Feld ausgelenkt wird.
\paragraph{einfaches Modell:}
\begin{itemize}
	\item$\infty$-schwerer Kern
	\item negativ geladenes Elektron
\end{itemize}
\textit{Model:}harmonischer Oszilator\\
\textit{Idee:} Berechne $x(t)$ der Elektronenwolke und daraus das Dipolmomentbzw. die Polarisation $\vecP$. es folgt $\epsilon(\omega)$ über $\vecP=(\epsilon-1)\epso \vecE$ mit der treibenden Kraft $\vecF(t)=-e\vecE(t)=-e\vecE_0 \exp(i\omega t)$, wobei $\vecE\|\vecx$ und $\vecE_0$ reell ist.
Daraus erhält man die Bewegungsgleichung für einen 1-dim. harmonischen Oszillator mit schwacher Dämpfung.
\begin{align*}
	\ddot{x}+\gamma\dot{x}+\omegao x&=\fracone{m}F(t)=\frac{-e}{m}E_0\exp(i\omega t)\\
	\text{mit der Lösung:}\quad x(t)&=\frac{-e}{m}\fracone{(\omegao^2-\omega^2)+i\gamma\omega}E(t)
\end{align*}
Wenn man nun noch die Teilchendichte N (Teilchen pro Volumen) dazunimmt, erhält man:
\begin{align*}
	P(t)&=-ex(t)N=\frac{eN^2}{m}\fracone{(\omegao^2-\omega^2)+i\gamma\omega}E(t)=(\eps(\omega)-1)\epso E(t)\\
\end{align*}
Daraus erhält man die frequenzabhängige Dielektrizitätskonstante:
\begin{align*}
	 \eps(\omega)&=1+\frac{e^2N}{\epso m}\fracone{(\omegao^2-\omega^2)+i\gamma\omega}
\end{align*}
Bei einem Medium mit verschiedenen Atomen/Molekülen summiert man über die Dielektrizitätskonstanten und man erhält folgendes:
\begin{align*}
	 \eps(\omega)&=1+\frac{e^2}{\epso m}\sum_{j}\frac{N_j}{(\omegao^2-\omega^2)+i\gamma\omega}
\end{align*}


\section{Der Brechungsindex}
Am einfachsten ist die Betrachtung eines verdünnten Mediums,
d.\,h. man ist weit weg von Resonanzerscheinungen (s. Folien).
Dazu machen wir folgende Annahmen:
\begin{align*}
  \eps(\omega) &\approx 1\\
  |\Delta\eps| &\ll 1\\
\Longrightarrow n(\omega) &= \sqrt{\eps(\omega)} \approx 1
\end{align*}
Mit dieser Schätzung erhalten wir
\begin{align*}
  \eps - 1 &= n^2-1 = \underbrace{(n+1)}_{\approx 2}(n-1) \approx 2(n-1)\\
  n-1 &\approx \frac{1}{2} (\eps -1) 
  = \frac{e^2N}{\epso m\cdot 2}\cdot
  \frac{1}{\omegao^2-\omega^2 + i\gamma\omega}
\end{align*}
und es folgt
\begin{align*}
  n_R &= 1 + \frac{e^2N}{2m\epso} 
        \frac{\omegao^2-\omega^2}{(\omegao^2-\omega^2)^2 + \gamma^2\omega^2}
  && \text{Realteil}\\
  n_I &= \frac{e^2N}{2m\epso}
        \frac{-\gamma\omega}{(\omegao^2-\omega^2)^2 + \gamma^2\omega^2}
  && \text{Imaginärteil}\\
\end{align*}
Oft schreibt man $n'$ für den Realteil, genannt 
\emph{reelle Brechzahl}\index{Brechzahl}, und 
$k$ für den Imaginärteil, genannt
\emph{Absorptionsindex}\index{Absoprtionsindex}:
\begin{gather*}
  n(\omega) = n'(\omega) - ik(\omega)
\end{gather*}%
\nomenclature{$n'$}{reelle Brechzahl eines Mediums;
  Realteil $n_R$ des Brechungsindex $n$}%
\nomenclature{$k$}{Absorptionsindex eines Mediums;
  Imaginärteil $n_I$ des Brechungsindex $n$}%


\section{Absorption von Licht}
Wir nehmen eine \emph{ebene Welle $E(z,t)$} an, welche in $z$-Richtung
mit Wellenvektor $k=\frac{n\omega}{c}$ propagiert, in einem Medium mit
$n(\omega)=n_R(\omega)+in_I(\omega)$.
Dann ergibt sich
\begin{align*}
  E(z,t) &= E_0\exp(i\omega t-ikz) \\
         &= E_0\exp\left[
           i\omega t - i\frac{\omega n_R}{c}\cdot z 
           + \frac{\omega n_I}{c}\cdot z
           \right]\\
         &= \left( 
           E_0\exp\left(\frac{\omega n_I}{c}\cdot z\right)
           \right)
%           \underbrace{
           \exp\left( i\omega t-\frac{i\omega n_R}{c}\cdot z\right)
%           }_\substack{\text{\enquote{normale} ebene Welle \\
%  Wellenlänge wird durch $n_R$ bestimmt}}
\end{align*}
Die Amplitude der Welle wird exponentiell gedämpft, falls $n_I<0$, was
fast immer der Fall ist (aber z.\,B. nicht bei Lasermedien). Daraus
folgt für die Intensität der ebenen Welle
\begin{gather*}
  I(z) = I(0)\exp\left(\frac{2\omega n_I z}{c}\right)
  = I(0) \exp(-az)
\end{gather*}
wobei $a=\frac{2\omega n_I}{c}$
\emph{Extinktionskoeffizient}%
\index{Extinktionskoeffizient}%
\nomenclature{$a$}{Extinktionskoeffizient eines Mediums}
genannt wird. Es ist
\begin{gather*}
  a = \frac{e^2N}{\epso mc}        
  \frac{\gamma\omegao^2}{(\omegao^2-\omega^2)^2 + \gamma^2\omega^2}
\end{gather*}
Das Verhältnis von Intensität zu Nullintensität wird 
Transmission $T$%
\index{Transmission}%
\index{Lambert-Beersches Gesetz}%
\nomenclature{$T$}{Transmission eines Mediums bei ebener Welle;
$T \coloneqq \frac{I(z)}{I(0)} = \exp(-az)$} 
genannt
\begin{align*}
  T &\coloneqq \frac{I(z)}{I(0)} = \exp(-az)
  &&\text{Lambert-Beersches Gesetz}
\end{align*}
\red{Achtung:} Imaginärteil und Realteil des Brechungsindex sind nicht
unabhängig voneinander! Die Relation zwischen ihnen wird
\emph{Kramer-Kronis-Relation}\index{Kramer-Kronis-Relation}
genannt:
\begin{gather*}
  \left(\Real(n(\omega))\right)^2 - 1
  = \frac{2c}{\pi} \int_0^\infty
  \frac{a(\omega')}{\omega'^2-\omega^2}\dd \omega'
\end{gather*}
Diese Relation ist wichtig für die optische Spektroskopie.

%----------------

\section[Detailbetrachtung]{Wie sieht das im Detail aus?}
Die zu erzwungenen Schwingungen angeregten Atome (Elektronenwolken) im
Medium strahlen mit der Anregungsfrequenz, aber mit \emph{verzögerter
  Phase} ab (s. Mechanik)

%----------------

\section{Das elektromagnetische Feld eines oszillierenden Dipols}
%% Image Missing!!
Die Licht-Materie-Wechselwirkung wird in erster Linie durch die
Wechselwirkungen elektromagnetischer Wellen mit atomaren Dipolen
beschrieben. Es wird hierzu das elektrische und magnetische Feld eines
strahlenden Dipols berechnet.
Wir werden im Folgenden das $\vecB$-Feld betrachten (das $\vecE$-Feld
wird von den Maxwell-Gleichungen beschrieben).

Es gilt für das Vektorpotential $\vecB=\vecna\times\vecA$, wobei
\begin{gather*}
  \vecA (\vech) = \frac{\muo}{4\pi} \int 
  \frac{\vecj(\vecr_2)\dd V_2}{|\vecr_{12}|}
\end{gather*}
\red{Achtung:} für $\vecj=\vecj(\vecr_2,t)$ ergibt sich eine
Retardierung, da sich das Feld nur mit Lichtgeschwindigkeit
ausbreitet.
Damit hängt $\vecA(\vecr_1, t)$ von $\vecj(\vecr_2)$ zur Zeit 
$t-\frac{|\vecr_{12}|}{c}$ ab:
\begin{gather*}
  \vecA(\vecr_1, t) = \frac{\muo}{4\pi} \int
  \frac
  {\vecj\left( \vecr_2, t-\frac{|\vecr_{12}|}{c} \right)}
  {|\vecr_{12}|}
  \dd V_2
\end{gather*}
Im Fernfeld (d.\,h. $|\vecr_{12}|\gg l$) gilt mit $\vecj = \vecv\rho$
\begin{gather*}
  \vecA(\vecr_1, t) = \frac{\muo}{4\pi r_{12}} \int
  \vecv \rho\left( \vecr_2, t-\frac{|\vecr_{12}|}{c} \right)
  \dd V_2
\end{gather*}
wobei $\rho \dd V_2 = \dd q$ und $\int\rho\dd V = q$ für die Ladung $q$
gilt.

Das elektrische Dipolmoment $\vecp$ wird beschrieben durch
\begin{gather*}
  \vecp(t) = q\underbrace{d_0\sin(\omega t)}_{\vecd(t)}\hat{e}_z
  = q\vecd(t)
\end{gather*}
mit $d_0$ als maximale Auslenkung der Ladung nach einer Viertel
Periode.

Jetzt verwenden wir $\vecp(t) = \vecp_0 \sin(\omega t)$ und
$\dif[\vecp]{t} = \dot{\vecp} = q\vecv$.
Damit erhält man
\begin{gather*}
  \vecA(t)(\vecr_1, t) = \frac{\muo}{4\pi\vecr_{12}}
  \dif[\vecp(t-\frac{r_{12}}{c})]{t}
\end{gather*}
mit $\omega(t-\frac{r_{12}}{c}) = \omega t- r_{12}\frac{2\pi}{\lambda}
= \omega t - kr_{12}$ ($k$ als Wellenvektor) sowie
$\dif[\vecp]{t} = q\cdot d_0\omega\cos(\omega t) \hat e_z$.
Insgesamt ergibt sich ein zeitlich veränderliches Vektorpotential, das
sich mit Lichtgeschwindigkeit ausbreitet:
\begin{gather*}
  \vecA(\vecr_1,t) = \frac{\muo}{4\pi}
  \underbrace{
    \frac{1}{\vecr_{12}} q\cdot d_0\omega\cos(\omega t-kr_{12})
    \vece_z
  }_{\text{Kugelwelle mit }c=\frac{\omega}{k}}
\end{gather*}
Zusammengefasst erzeugt eine oszillierende Ladung ein Vektorfeld
$\vecA$, welches wiederum $\vecB$ und $\vecE$ erzeugt.

Es bleibt die Frage, wie sieht das $\vecB$-Feld am Ort $P$ aus?
Wir haben
\begin{align*}
  \vecB &= \vecna\times\vecA 
  &\text{mit } \vecA = \left( 0,0,\vecA_z \right)\\
  B_x &= \dif[A_z]{y} \\
  B_y &= -\dif[A_z]{x}\\
  B_z &= 0
\end{align*}
Das heißt, das $\vecB$-Feld liegt in der $x$-$y$-Ebene.
Die genaue Formel für $B_x$ ist
\begin{gather*}
  B_x = \frac{\muo}{4\pi}\left[
    \dot{p}\left( t-\frac{r_{12}}{c} \right)\dif{y} \cdot \frac{1}{r}
    + \frac{1}{r}
    \left( 
      \dot{p}\left(t-\frac{r_{12}}{c}\right)
    \right)
  \right]
\end{gather*}
Definiere zur Vereinfachung $u=t-\frac{r_{12}}{c}$, dann ist
$\dot{p} = \dif[p]{u}\cdot\dif[u]{t} = \dif[p]{u}$
und $\dif[u]{r} = -\frac{1}{c}$.
Außerdem gilt $\dif[r]{y}=\frac{y}{r}$, d.h.
\begin{gather*}
  \dif[\dot{p}]{y} 
  = \dif[\dot{p}]{u}\cdot\dif[u]{r}\cdot\dif[r]{y}
  = -\ddot{p}\frac{1}{c}\frac{y}{r}
\end{gather*}
Damit gilt dann
\begin{gather*}
  B_x = -\frac{1}{4\pi\epso c^2}
  \left[ \dot{p}\frac{y}{r^3} 
    + \ddot{p} \frac{y}{cr^2} \right]
\end{gather*}
$B_y$ analog. Insgesamt
\begin{gather*}
  \vecB(\vecr) = \frac{1}{4\pi\epso c^2}\cdot
  \frac{1}{r^3}
  \left[ 
    \dot{\vecp}\times\vecr 
    + \frac{r}{c}\ddot{\vecp}\times\vecr
  \right]
\end{gather*}
Wegen $\vecp\,\parallel\,\dot{\vecp}\,\parallel\,\ddot{\vecp}$
gilt $\vecB\,\bot\,\vecp_{\text{Dipol}}$ und
$\vecB\,\bot$ Ausbreitungsrichtung.
Damit folgen zwei Terme für $\vecB$:
\begin{enumerate}
\item Nahfeld $\propto \dot\vecp \propto \frac{1}{r^2}$
\item Fernfeld $\propto \ddot\vecp \propto \frac{1}{r}$
\end{enumerate}
Daher kommen die Terme
\begin{gather*}
  \vecna \times \vecB 
  = \underbrace{\mu_0\vecj}_{\text{Nahfeld}}
  + \underbrace{\muo\epso \dif[\vecE]{t}}_{\text{Fernfeld}}
\end{gather*}
wobei das $\vecE$-Feld über die Maxwell-Gleichungen beschrieben wird.

\paragraph{Ergebnisse im Fernfeld}
\begin{itemize}
	\item $\vecE$ und $\vecB$ sind Pulse.
	\item $E=cB$ (für elektromagnetische Wellen)
	\item Das $\vecB$-Feld beschreibt konzentrische Kreise um die
      Dipolachse, $\vecE\bot\vecB$.
	\item Im großen Abstand sind $\vecB$ und $\vecE$ linear
      polarisierte ebene Wellen.
	\item Das abgestrahlte Feld ist proportional zur
      Dipolbeschleunigung und die Amplitude nimmt proportional zu
      $\fracone{r}$ ab.
\end{itemize}
Zusammengefasst haben wir folgende Formeln:
\begin{align*}
  \vert \vecE\vert
  &\propto\vert\vecB\vert\propto\frac{\dot{\vecp}(\vecr,t)}{r}\\
  \vert\vecE\vert
  &=\fracone{4\pi\epso c^2} \frac{p_0\omega^2}{r}
    \sin\theta \sin(\omega t-\veck\vecr)
\end{align*}
Die Intensität ist dabei $\propto\sin^2\theta$. Die gesamte vom Dipol
abgestrahlte zeitlich gemittelte Leistung dabei beträgt: 
\begin{align*}
	\langle P_{em}\rangle
  =\frac{q^2\omega^4 d_0^2}{12\pi\epso c^3}
  \propto\omega^4 
\end{align*}

%----------------

\section{Die Dispersion von dichten Medien}
Bisher lagen die Werte für  $n$ bzw. $\eps$ nahe bei 1. Jetzt
betrachten wir $\eps>1$ und mehrere Resonanzen (beispielsweise bei
verschiedenen Molekülsorten). Bei mehreren Atomen im System erhält man
folgende Gleichung
\begin{gather}
  n^2(\omega)
  = \eps(\omega)
  = 1 + \frac{e^2}{\epso m_e}\sum_{j}
  \frac{f_jN_j}{\omegao^2-\omega^2+i\gamma_j\omega}
  \label{brechungsindexallg}
\end{gather} 
wobei $f_j$\nomenclature{$f_j$}{Oszillatorstärke;
  $\eps(\omega)
  = 1 + \frac{e^2}{\epso m_e}\sum_{j}
  \frac{f_jN_j}{\omegao^2-\omega^2+i\gamma_j\omega}$}
die Oszillatorstärke\index{Oszillatorstärke},
$\gamma_j$\nomenclature{$\gamma_j$}{Dämpfung; 
  $\eps(\omega)
  = 1 + \frac{e^2}{\epso m_e}\sum_{j}
  \frac{f_jN_j}{\omegao^2-\omega^2+i\gamma_j\omega}$} die
Dämpfungskonstante,
$e$\nomenclature{$e$}{Elementarladung} die Elementarladung und
$m_e$\nomenclature{$m_e$}{Elektronenmasse} die Elektronenmasse ist.
Zur Berechnung des Brechungsindex verwendet man oft die 
\emph{Sellmeier-Beziehung}%
\index{Sellmeier-Beziehung}:
\begin{align*}
	n^2(\lambda)= A + \sum_{i=1}^{N}\frac{B_j}{\lambda^2-C_j^2}
\end{align*}
$A$, $B_j$, $C_j$ werden aus Messungen von $n(\lambda)$ bestimmt. Oft
genügt der erste Term (z.\,B. beim sichtbaren Bereich, für nicht
absorbierende Materialien).
\begin{itemize}
	\item Im sichtbaren (z.\,B. Gläser) nimmt $n(\omega)$ über weite
      Bereiche mit $\omega$ zu. Diese nennt man die 
      \enquote{\emph{positive} oder \emph{normale Dispersion}}%
      \index{Dispersion!positiv}\index{Dispersion!normal}
      mit $\frac{\dd n}{\dd \omega}>0$
      und $\frac{\dd n}{\dd \omega}>\ddif[\omega]{\lambda}$.
	\item Nahe bei den Resonanzen findet man $\ddif[n]{\lambda}>0$ und
      $\ddif[n]{\omega}<0$. Diese nennt man die 
      \enquote{\emph{negative} oder \emph{annormale Dispersion}}. 
      Der Imaginärteil ist hier wesentlich
      von nulll verschieden, deshalb absorbiert das Material. 
\end{itemize}

\paragraph{Transparente Materialien} Dispersion im sichtbaren Bereich
ist dominiert  von einer elektronischen Resonanz im UV-Bereich. Die
Dispersion bei transparenten Materialien ist ähnlich.

\section{Brechungsindex und Absorption von Metallen}
Metalle sind in erster Linie ein freies Elektronengas. Ein See von
Leitungselektronen führt zu hoher Leitfähigkeit. Eine
charakteristische Größe ist die 
Streuzeit $\tau$%
\nomenclature{$\tau$}{Streuzeit}%
\index{Streuzeit}
bzw. 
die Stoßfrequenz $\fracone{\tau}$%
\nomenclature{$\frac{1}{\tau}$}{Stoßfrequenz bei Streuzeit $\tau$;
  $\gamma\coloneqq \frac{1}{\tau}$}%
\index{Streuzeit} 
Für diese gilt
\begin{align*}
	\tau=\frac{\sigma_0m_e}{N e^2}
\end{align*}
wobei 
$N$\nomenclature{$N$}{Teilchenzahldichte} die Teilchenzahldichte ist.
Die Streuzeit liegt in einer Größenordnung von
ca. $10^{-14}$ bis $10^{-15}$. Für Frequenzen der elektromagnetischen Wellen
$\omega\gg\fracone{\tau}$ (Dämpfung vernachlässigbar) kann man im
einfachsten Modell schreiben
\begin{gather}
  \eps = 1 - \frac{\omega_p^2}{\omega^2} \label{Perm}
\end{gather}
mit \emph{Plasmafrequenz}
$\omega_p^2=\frac{e^2N}{\epso m_e}$%
\nomenclature{$\omega_p$}{Plasmafrequenz;
  $\omega_p^2=\frac{e^2N}{\epso m_e}$ im nahezu ungedämpften Fall}%
\index{Plasmafrequenz}.
Es gilt
\begin{description}
\item[$0 < \fracone{\tau}$] normaler Skineffekt
\item[$\fracone{\tau} < \omega_p$] annormaler Skineffekt
\item[$\omega_p < \omega$] normale Wellenausbreitung
\end{description}

Aus der Gleichung \ref{Perm} folgt für
\begin{description}
	\item[große Frequenzen ($\omega>\omega_p$):] Hier ist die
      Dielektrizitätskonstante positiv und somit der Brechungsindex
      reell. Daher gibt es keine Absorption. 
	\item[kleine Frequenzen ($\omega<\omega_p$):] Hier ist die
      Dielektrizitätskonstante negativ und somit der Brechungsindex
      rein imaginär. Also es gibt Absorption. 
\end{description}
Wenn man nun aber ein realistischeres Metall betrachtet, also ein Medium
mit Dämpfung, erhält man folgende Gleichungen
\begin{align*}
  \sigma
  &= \frac{Ne^2}{m_e}\tau
    = \epso\omega_p^2 \tau
    = \frac{\epso\omega_p^2}{\gamma}
  &\text{mit}\quad \tau=\fracone{\gamma}\\
  \eps(\omega)
  &= 1 + \frac{e^2N}{\epso m}\fracone{-\omega^2+i\gamma\omega}
    = 1 - \frac{\omega_p^2}{\omega^2-i\gamma\omega}
  &\text{aus \eqref{brechungsindexallg}}
\end{align*}%
\nomenclature{$\gamma$}{Stoßfrequenz bei Streuzeit $\tau$; 
  $\gamma\coloneqq \fracone{\tau}$}%
Anschaulich:
\begin{description}
  \item[$0 < \fracone{\tau}$ (normaler Skineffekt)]
    Die Eindringtiefe fällt mit wachsender Frequenz proportional zu
      $\sqrt{\omega}$ ab.
  \item[$\fracone{\tau}<\omega_p$ (annormaler Skineffekt)]
    Die Eindringtiefe ist konstant mit $\fracone{a}=\frac{c}{2\omega_p}$.
  \item[$\omega_p<\omega$ (normale Wellenausbreitung)]
    Das Material ist transparent.
\end{description}


%%% Local Variables:
%%% mode: latex
%%% TeX-master: "../OptikSkriptWS1516"
%%% End:
