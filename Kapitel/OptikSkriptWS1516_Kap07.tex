\chapter{Geometrische Optik}
Geometrische Optik
Die \emph{geometrische Optik}\index{geometrische Optik}
bzw. \emph{Strahlenoptik}\index{Strahlenoptik} befasst sich mit der
Darstellung komplexer optischer Instrumente.
%% IMAGE MISSING
Es wird der Grenzfall betrachtet, dass die Dimension des Problems sehr
viel größer sein muss, als die Wellenlänge.
Zur Vereinfachung wird die Wellenfront einer punktförmigen Lichtquelle
betrachtet und ein Lichtbündel bzw. ein Lichtstrahl durch Einfügen
einer Blende erzeugt. Beugungseffekte werden vernachlässigt.
%% IMAGE MISSING

\section[Fermat'sches Prinzip]
{Grundgesetze der geometrischen Optik/Fermat'sches Prinzip}
Für die geometrische Optik werden weiterhin folgende Annahmen gemacht:
\begin{itemize}
\item geradlinige Ausbreitung von Licht in einem homogenen Medium
\item Reflexionsgesetz und Snellius'sches Brechungsgesetz gelten
\item Strahlengänge sind umkehrbar
\item sich durchdringende Strahlen beeinflussen sich nicht
\end{itemize}
Dann gilt das \emph{Fermat'sche Prinzip}\index{Fermat'sche Prinzip}
(auch Variationsprinzip\index{Variationsprinzip}):
Lichtausbreitung erfolgt so, dass der optische Weg $W(s)$%
\nomenclature{$W(s)$}{optischer Weg im Fermat'schen Prinzip;
  $W(s) = \int_{s(Q\to P)}n(\vecx)\dd s$}
(Produkt aus Brechungungsindex $n(\vecr)$ und zurückgelegter Strecke
$s$) einen Extremwert besitzt:
\begin{gather*}
  W(s) = \int_{s(Q\to P)}n(\vecx)\dd s
  \qquad\text{und}\qquad
  \left( \dif[W(s)]{s} \right)_{s_0} = 0
\end{gather*}%
\nomenclature{$s$}{Strecke (Pfad) zwischen zwei Punkten}%
für eine Strecke (einen Pfad) $s$ vom Punkt $Q$ zum Punkt $P$.

Bemerkung dazu:
\begin{itemize}
\item Das Fermat'sche Prinzip lässt sich aus der Wellentheorie des
  Lichts herleiten.
\item Aus der obigen Gleichung folgt die Umkehrbarkeit des Lichtweges.
\item Es kann ein Minimum \emph{oder} ein Maximum sein (fast immer ein
  Minimum)
\end{itemize}
%% IMAGE MISSING



%%% Local Variables:
%%% mode: latex
%%% TeX-master: "../OptikSkriptWS1516"
%%% End:
