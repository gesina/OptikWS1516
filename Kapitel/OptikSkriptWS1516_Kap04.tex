%%-----------------------------------------

%% Vorlesungsmitschrift (Kapitel 4)

%% an der Uni Regensburg, gelesen von Christian Back

%%-----------------------------------------


\chapter{Elektromagnetische Wellen an Grenzflächen}
\section{Randbedingungen der elektromagnetischen Welle}
Wir wollen jetzt Wellenausbreitung in inhomogenen Medien beschreiben, z.B. den Übergang von Medium 1 nach Medium 2, also einer Grenzfläche.
\paragraph{Randbedingungen der MWGl.:} Die Tangentialkomponenten von $\vecE$ und $\vecH=\fracone{\muo\mu_r}\vecB$ sind stetig. Die Normalkomponenten von $\vecD=\epso\eps_r\vecE$ und $\vecB$ sind ebenfalls stetig. (isotrope, isolierende, nicht magnetische Medien $\mu=1$)
\paragraph{einfachster Fall:}zwei homogene Medien mit Brechungsindex $n_e$ (einfallender Strahl) und $n_t$ (transmittierte Welle).
%
%Skizze zu Reflexion
%
 Der Winkel $\alpha$ liegt zwischen $\veck_e$ und $\vece_y$ bzw. zwischen $\veck_t$ und $\vece_y$. Wir nehmen an, dass es eine fest vrogegebene einlaufende Welle ist.
 \begin{align*}
 	\vecE_e&=\vecE_{e_0}cor(\omega_et-\veck_e\vecr)=\vecE_{e_0}(\phi_e(\vecr,t))\\
	\vecE_r&=\vecE_{r_0}cor(\omega_rt-\veck_r\vecr+\varphi_r)=\vecE_{r_0}(\phi_r(\vecr,t))\\
	\vecE_t&=\vecE_{t_0}cor(\omega_tt-\veck_t\vecr+\varphi_t)=\vecE_{t_0}(\phi_t(\vecr,t))
 \end{align*}
 Die Wellenvektoren $\veck_e,\veck_r$ und $\veck_t$ müssen die Dispersionsrelationen im jeweiligen Medium erfüllen. Die Phasenfaktoren $\varphi_r,\varphi_t$ bestimmen die Phasenlage relativ zur einlaufenden Welle.