\chapter{Nichtlineare Optik}

Bisher haben wir nur den Zusammenhang
\begin{gather*}
  \vecP = \epso\chi\vecE
\end{gather*}
betrachtet mit $\chi=\eps(\omega)-1$, wobei $\eps$ nicht von $\vecE$
abhängig war, also ein linearer Zusammenhang zwischen $\vecP $ und
$\vec E$.
Bei großen Feldstärken wird der Zusammenhang zwischen $\vecE $ und
$\vecP $ aber nichtlinear (Sättigungsverhalten).

Der einfachste Ansatz, um $\vecP(\vecE)$ dann zu beschreiben, ist die
Entwicklung nach Ordnungen von $\vecE $:
\begin{align*}
  P &= \epso \left(
      \chi_1 E + \chi_2 E^2 + \chi_3 E^3 \dotsb
       \right) \\
    &= P_\text{lin} + P_\text{nicht lin}
\end{align*}
I.\,A. sind die $\chi_i$ klein, aber $E$ kann dafür groß werden
(z.\,B. bei starken Laserfeldern).
Außerdem sind die $\chi_i=\vec{\chi}_i$ i.\,A. Tensoren $(i+1)$ter
Ordnung.

Wir verwenden den Ansatz ebener Wellen:
\begin{align*}
  P = \epso \chi_1 E_0\cos(\omega t) 
  &+ \frac{1}{2}\epso\chi_2 E_0^2 \big[ 
  \underbrace{1}_{\mathclap{\substack{\text{opt. Gleich-}\\\text{richtung}}}} +
  \underbrace{\cos(2\omega t)}_{\substack{\text{doppelte}\\ \text{Frequenz}}}
  \big]\\
  &+ \frac{1}{4}\epso\chi_3 E_0^3 \big[
  3\cos(\omega t) + 
  \underbrace{\cos(3\omega t)}_{\substack{\text{dreifache}\\\text{Frequenz}}}
  \big]\\
  &+ \dotsb
\end{align*}
Zu beobachtende Konsequenzen sind
\begin{itemize}
\item nichtlineare Frequenzverdopplung
\item nichtlinearer Brechungsindex (z.\,B. Selbstfokussierung)
\item Selbstphasenmodulation
\item \dots
\end{itemize}



%%% Local Variables:
%%% mode: latex
%%% TeX-master: "../OptikSkriptWS1516"
%%% End:
